\documentclass{beamer}
\usetheme{Madrid}

\setbeamertemplate{navigation symbols}{}
%\setbeamertemplate{footline}[frame number]
%\setbeamertemplate{headline}{}

\usepackage[utf8]{inputenc}

\usepackage[backend=biber,
style=alphabetic,
]{biblatex}

\usepackage{graphicx}

\usepackage{minted}
\usemintedstyle{borland}

\usepackage{mathtools}

\usepackage{tikz}
\usetikzlibrary{positioning,shapes.multipart}

\usepackage{multicol}

\usepackage[ngerman]{babel}
\usepackage{csquotes}

\usepackage{qtree}
\noautomath %qtree parsing ist besser? vielleicht?

\usepackage{dirtree} %alternative bäume
%\DTsetlength{offset}{width}{sep}{rule-width}{dot-size} (siehe doku)
\DTsetlength{0.2em}{1em}{0.2em}{0.4pt}{0pt}

\usepackage{multirow} %mehrspaltige eintraege in tabelle

\usepackage{xcolor} %\textcolor{blue}{This is a sample text in blue.}



\title[Entwicklung eines Termersetzungssystems]{Entwicklung eines Termersetzungssystems für
assoziative und kommutative Ausdrücke}
\subtitle{Vortrag zur Bachelorarbeit}
\author{Bruno Borchardt}
\date{\today}

\addbibresource{quellen.bib}

\newcommand\blfootnote[1]{%
  \begingroup
  \renewcommand\thefootnote{}\footnote{#1}%
  \addtocounter{footnote}{-1}%
  \endgroup
}


%\paren*{a + b} skaliert automatisch klammern um a + b
\DeclarePairedDelimiter\paren{(}{)} 
\DeclarePairedDelimiter\curl{\{}{\}}

\begin{document}

%------------------------------------------------------------------------
%------------------------------------------------------------------------
%------------------------------------------------------------------------
\begin{frame}[noframenumbering, plain]
	%\titlepage
	\maketitle 
	\footnotesize
	\begin{center}
		\begin{tabular}[t]{l l}
			Betreuer: & PD Dr. Prashant Batra \\
					~ & Prof. Dr. Siegfried Rump
		\end{tabular}
	\end{center}
\end{frame}

%------------------------------------------------------------------------
%------------------------------------------------------------------------
%------------------------------------------------------------------------

\section{Motivation}
\begin{frame}{Motivation}
	\begin{itemize}
		\item{Ziel: vereinfache mathematische Ausdrücke automatisch, 
			etwa $\sin\paren{\frac{21}{2} \pi}$ zu (exakt) $1$ 
			oder $x^2 + 5 x y - 12 x y$ zu $x^2 - 7 x y$}
		\pause
		\item{Problem: Computer kennen keine mathematischen Gesetzmäßigkeiten}
		\pause
		\item{Idee: formuliere Vereinfachungsregeln}
		\pause
		\item{Mathematica ist Beispiel einer kommerziellen Umsetzung}
	\end{itemize}
\end{frame}

%------------------------------------------------------------------------
%------------------------------------------------------------------------
%------------------------------------------------------------------------

\section{Grundidee}
\begin{frame}[fragile]{Term}
	\begin{block}{Definition}
		Terme sind
		\begin{itemize}
			\item{\emph{Konstantensymbole}, etwa \verb~1~, \verb~-3i~, \verb~a~, \verb~sin~}
			\item{oder \emph{Funktionsanwendungen} eines Terms $f$ auf die Terme 
				$t_1, \dots, t_n$, geschrieben $f(t_1, \dots, t_n)$, etwa \verb~sin(3)~}
		\end{itemize}
	\end{block}		
	~\\~\\
	\begin{columns}[t] %t für align nach oben
        \begin{column}{0.5\textwidth}
			\pause
			Ist $t = 3 + 4 a$ ein Term?
			\pause
			\\~\\
			Ja, formal geschrieben \\
			$t = $ \verb~sum(3, prod(4, a))~
        \end{column}
        \begin{column}{0.5\textwidth}	
			\small
			\Tree [.{$t$} \verb~sum~ \verb~3~ [ \verb~prod~ \verb~4~ \verb~a~ ]!\qsetw{2cm} ]
        \end{column}
	\end{columns}
	~\\~\\
\end{frame}

%------------------------------------------------------------------------

\begin{frame}[fragile]{Ersetzungsregeln}
	
	\begin{itemize}
		\item{Paar $(l, r)$ aus Termen $l$ und $r$, geschrieben $l$\verb~ = ~$r$}
		\item{unterscheide \emph{Literale} (zu vereinfachende Terme) von \emph{Mustern} (Terme in Regeln)}
		\item{wird $l$ in einem Literal gefunden, erfolgt die Ersetzung durch $r$}
		\item{\emph{Mustervariablen} sind Platzhalter in Regeln, idetifiziert durch vorangestellten Unterstrich {\verb~_~}, etwa \verb~_x~}
	\end{itemize}	
\end{frame}

%------------------------------------------------------------------------

\begin{frame}[fragile]{Beispiel}
	
	Die Regel \verb~sum(_x, _x) = prod(2, _x)~ steht für die folgende Transformation eines Literals.
	~\\
	\begin{columns}[c] 
		\footnotesize
        \begin{column}{0.45\textwidth}			
			\Tree [.{\dots} [ \verb~sum~ \qroof{~~~\texttt{\char`_x}~~~}. \qroof{~~~\texttt{\char`_x}~~~}. ] ] 
        \end{column}
        \begin{column}{0.0001\textwidth}
			$\rightarrow$
        \end{column}
        \begin{column}{0.45\textwidth}	
			\Tree [.{\dots} [ \verb~prod~ \verb~2~ \qroof{~~~\texttt{\char`_x}~~~}. ] ]
        \end{column}
	\end{columns}
	~\\
	\pause
	Aus $t =$\verb~ sin(sum(a, a))~ wird $t' =$\verb~ sin(prod(2, a))~.	
	~\\
	\begin{columns}[c] 
		\footnotesize
        \begin{column}{0.45\textwidth}			
			\Tree [.{$t$} \verb~sin~ [ \verb~sum~ \verb~a~ \verb~a~ ] !{\qframesubtree} ] 
        \end{column}
        \begin{column}{0.0001\textwidth}
			$\rightarrow$
        \end{column}
        \begin{column}{0.45\textwidth}	
			\Tree [.{$t'$} \verb~sin~ [ \verb~prod~ \verb~2~ \verb~a~ ] !{\qframesubtree} ]
        \end{column}
	\end{columns}
	
\end{frame}

%------------------------------------------------------------------------

\begin{frame}[fragile]{Normalisierung}
	Wann sind zwei Terme identisch?
	\begin{enumerate}
		\item{\verb~sum(1, 1) ~ vs. \verb~ 2~}
		\item{\verb~sum(a, b) ~ vs. \verb~ sum(b, a)~}
		\item{\verb~sum(sum(a, b), 2) ~ vs. \verb~ sum(a, sum(b, 2))~}
	\end{enumerate}
	~\\
	\pause
	\begin{enumerate}
		\item{werte bekannte Funktionsanwendungen aus:\\
			\verb~  sum(1, 1)~ $\rightarrow$ \verb~2~}			
		\item{sortiere Argumente kommutativer Funktionsanwendungen:\\
			\verb~  sum(b, a)~ $\rightarrow$ \verb~sum(a, b)~}
		\item{erlaube keine geschachtelten Funktionsanwendungen derselben assoziativen Funktion:\\
			\verb~  sum(sum(a, b), 2)~ $\rightarrow$ \verb~sum(a, b, 2)~}
	\end{enumerate}	
\end{frame}

%------------------------------------------------------------------------

\begin{frame}[fragile]{Multi-Mustervariablen}
	\begin{itemize}
		\item{Bestimmte Muster sollen Funktionsanwendungen beliebiger Argumentanzahl darstellen können.}
		\item{Multi-Mustervariablen sind Platzhalter für beliebig viele Argumente, identifiziert durch einen Suffix aus drei Punkten, etwa \verb~xs...~}
	\end{itemize}
	
	\pause
	Regel \verb|sum(_x, _x) = prod(2, _x)| wird zu \verb|sum(_x, _x, ys...) = sum(prod(2, _x), ys...)|.	
	~\\
	\begin{columns}[c] 
		\scriptsize
        \begin{column}{0.49\textwidth}			
			\Tree [ \verb~sum~ \qroof{~~~\texttt{\char`_x}~~~}. \qroof{~~~\texttt{\char`_x}~~~}. {\texttt{ys...}}  ] 
        \end{column}
        \begin{column}{0.0001\textwidth}
			$\rightarrow$
        \end{column}
        \begin{column}{0.49\textwidth}	
			\Tree [ \verb~sum~ [ \verb~prod~ \verb~2~ \qroof{~~~\texttt{\char`_x}~~~}. ] {\texttt{ys...}} ]
        \end{column}
	\end{columns}
\end{frame}

%------------------------------------------------------------------------

\begin{frame}[fragile]{Termersetzungssystem}
	\begin{block}{Pseudocode Termersetzungssystem}
		Input: Regelmenge $R$, Literal $t$ \\
		
		\begin{enumerate}
			\item normalisiere $t$ \label{schritt:normalisieren}
			\item Wenn $\exists (l, r) \in R$ anwendbar auf $t$ oder Teil von $t$
			\item ~~~~ ersetze $l$ in $t$ durch $r$
			\item ~~~~ gehe zu \ref{schritt:normalisieren}
			\item gebe $t$ zurück
		\end{enumerate}
	\end{block}
\end{frame}

%------------------------------------------------------------------------

\begin{frame}[fragile]{Demo}
	\pause
	\begin{block}{Spickzettel}
		\footnotesize
		\begin{verbatim}
diff(3 + t + a, t)

:add diff(_u + vs..., _x) = diff(_u, _x) + diff(sum(vs...), _x)

:add diff(_y, _x) | !contains(_y, _x) = 0

:add diff(_x, _x) = 1
		\end{verbatim}
	\end{block}
\end{frame}

%------------------------------------------------------------------------
%------------------------------------------------------------------------
%------------------------------------------------------------------------

\section{Mustererkennung}
\begin{frame}[fragile]{Match}
	\begin{block}{Definition}
		Eine Funktion $v$, die Mustervariablen auf Literale abbildet, ist ein \emph{Match} des Musters $p$ mit dem Literal $t$, wenn die normalisierte Form der Ersetzung der Mustervariablen $x_i$ in $p$ durch ihre Funktionswerte $v(x_i)$ mit $t$ identisch ist.
	\end{block}
	
	\pause
	\begin{block}{Beispiel 1}
		Mit $v($\verb~_x~$) =$\verb~ sin(a)~ ist $v$ ein Match des
		Musters $p =$\verb~ sum(_x, _x)~ mit dem Literal $t =$\verb~ sum(sin(a), sin(a))~.
	\end{block}
	
	\pause
	\begin{block}{Beispiel 2}
		Mit $v($\verb~_x~$) =$\verb~ 3~ ist $v$ ein Match des
		Musters $p =$\verb~ sum(_x, _x)~ mit dem Literal $t =$\verb~ 6~.
	\end{block}
\end{frame}

%------------------------------------------------------------------------

\begin{frame}[fragile]{Vorgehen}
	\begin{itemize}
		\item{Muster und Literal werden parallel abgelaufen}
		\item{Mustervariablen werden in Entdeckungsreihenfolge der Tiefensuche an Teilterme des Literals gebunden}
		\item{Backtracking wenn Match aktuell unmöglich}
		\item{Algorithmus berücksichtigt Kommutativität entsprechender Funktionsanwendungen}
	\end{itemize}
\end{frame}

%------------------------------------------------------------------------



\begin{frame}[fragile]{Beispiel}
	Matche $p =$\verb~ prod(pow(_x, _y), _x, zs...)~ in 
	$t =$\verb~ prod(pow(a, 2), pow(sum(3, b), 4), sum(3, b))~
	~\\
	\begin{columns}[c] %t für align nach oben
		\scriptsize
        \begin{column}{0.3\textwidth}	
			\dirtree{%
			.1 \texttt{prod}.	
				.2 \texttt{pow}.
					.3 \texttt{\char`_x}.
					.3 \texttt{\char`_y}.
				.2 \texttt{\char`_x}.	
				.2 \texttt{zs...}.
			}
			~\\~\\
			\begin{tabular} {c|c}
			\multicolumn{2}{c}{Matchstatus}\\
			\hline
			\verb~_x~      & -            \\
			\verb~_y~      & -            \\
			\verb~zs...~   & -            
			\end{tabular}
        \end{column}
		
        \begin{column}{0.3\textwidth}	
			\dirtree{%
			.1 \texttt{prod}.		
				.2 \texttt{pow}.
					.3 \texttt{a}.
					.3 \texttt{2}.
				.2 \texttt{pow}.
					.3 \texttt{sum}.
						.4 \texttt{3}.
						.4 \texttt{b}.
					.3 \texttt{4}.
				.2 \texttt{sum}.
					.3 \texttt{3}.
					.3 \texttt{b}.
			}
        \end{column}
	\end{columns}
\end{frame}

%teste äußeres produkt
\begin{frame}[fragile, noframenumbering]{Beispiel}
	Matche $p =$\verb~ prod(pow(_x, _y), _x, zs...)~ in 
	$t =$\verb~ prod(pow(a, 2), pow(sum(3, b), 4), sum(3, b))~
	~\\
	\begin{columns}[c] %t für align nach oben
		\scriptsize
        \begin{column}{0.3\textwidth}	
			\dirtree{%
			.1 \textcolor{cyan}{\texttt{prod}}.
				.2 \texttt{pow}.
					.3 \texttt{\char`_x}.
					.3 \texttt{\char`_y}.
				.2 \texttt{\char`_x}.	
				.2 \texttt{zs...}.	
			}
			~\\~\\
			\begin{tabular} {c|c}
			\multicolumn{2}{c}{Matchstatus}\\
			\hline
			\verb~_x~      & -            \\
			\verb~_y~      & -            \\
			\verb~zs...~   & -            
			\end{tabular}
        \end{column}
		
        \begin{column}{0.3\textwidth}	
			\dirtree{%
			.1 \textcolor{cyan}{\texttt{prod}}.		
				.2 \texttt{pow}.
					.3 \texttt{a}.
					.3 \texttt{2}.
				.2 \texttt{pow}.
					.3 \texttt{sum}.
						.4 \texttt{3}.
						.4 \texttt{b}.
					.3 \texttt{4}.
				.2 \texttt{sum}.
					.3 \texttt{3}.
					.3 \texttt{b}.
			}
        \end{column}
	\end{columns}
\end{frame}

%teste erste pow
\begin{frame}[fragile, noframenumbering]{Beispiel}
	Matche $p =$\verb~ prod(pow(_x, _y), _x, zs...)~ in 
	$t =$\verb~ prod(pow(a, 2), pow(sum(3, b), 4), sum(3, b))~
	~\\
	\begin{columns}[c] %t für align nach oben
		\scriptsize
        \begin{column}{0.3\textwidth}	
			\dirtree{%
			.1 \textcolor{green}{\texttt{prod}}.
				.2 \textcolor{cyan}{\texttt{pow}}.
					.3 \texttt{\char`_x}.
					.3 \texttt{\char`_y}.
				.2 \texttt{\char`_x}.		
				.2 \texttt{zs...}.
			}
			~\\~\\
			\begin{tabular} {c|c}
			\multicolumn{2}{c}{Matchstatus}\\
			\hline
			\verb~_x~      & -            \\
			\verb~_y~      & -            \\
			\verb~zs...~   & -            
			\end{tabular}
        \end{column}
		
        \begin{column}{0.3\textwidth}	
			\dirtree{%
			.1 \textcolor{green}{\texttt{prod}}.		
				.2 \textcolor{cyan}{\texttt{pow}}.
					.3 \texttt{a}.
					.3 \texttt{2}.
				.2 \texttt{pow}.
					.3 \texttt{sum}.
						.4 \texttt{3}.
						.4 \texttt{b}.
					.3 \texttt{4}.
				.2 \texttt{sum}.
					.3 \texttt{3}.
					.3 \texttt{b}.
			}
        \end{column}
	\end{columns}
\end{frame}

%argument 1 von erstem pow
\begin{frame}[fragile, noframenumbering]{Beispiel}
	Matche $p =$\verb~ prod(pow(_x, _y), _x, zs...)~ in 
	$t =$\verb~ prod(pow(a, 2), pow(sum(3, b), 4), sum(3, b))~
	~\\
	\begin{columns}[c] %t für align nach oben
		\scriptsize
        \begin{column}{0.3\textwidth}	
			\dirtree{%
			.1 \textcolor{green}{\texttt{prod}}.
				.2 \textcolor{green}{\texttt{pow}}.
					.3 \textcolor{cyan}{\texttt{\char`_x}}.
					.3 \texttt{\char`_y}.
				.2 \texttt{\char`_x}.		
				.2 \texttt{zs...}.
			}
			~\\~\\
			\begin{tabular} {c|c}
			\multicolumn{2}{c}{Matchstatus}\\
			\hline
			\verb~_x~      & \textcolor{cyan}{a}            \\
			\verb~_y~      & -            \\
			\verb~zs...~   & -            
			\end{tabular}
        \end{column}
		
        \begin{column}{0.3\textwidth}	
			\dirtree{%
			.1 \textcolor{green}{\texttt{prod}}.		
				.2 \textcolor{green}{\texttt{pow}}.
					.3 \textcolor{cyan}{\texttt{a}}.
					.3 \texttt{2}.
				.2 \texttt{pow}.
					.3 \texttt{sum}.
						.4 \texttt{3}.
						.4 \texttt{b}.
					.3 \texttt{4}.
				.2 \texttt{sum}.
					.3 \texttt{3}.
					.3 \texttt{b}.
			}
        \end{column}
	\end{columns}
\end{frame}

%argument 2 von erstem pow
\begin{frame}[fragile, noframenumbering]{Beispiel}
	Matche $p =$\verb~ prod(pow(_x, _y), _x, zs...)~ in 
	$t =$\verb~ prod(pow(a, 2), pow(sum(3, b), 4), sum(3, b))~
	~\\
	\begin{columns}[c] %t für align nach oben
		\scriptsize
        \begin{column}{0.3\textwidth}	
			\dirtree{%
			.1 \textcolor{green}{\texttt{prod}}.
				.2 \textcolor{green}{\texttt{pow}}.
					.3 \textcolor{green}{\texttt{\char`_x}}.
					.3 \textcolor{cyan}{\texttt{\char`_y}}.
				.2 \texttt{\char`_x}.		
				.2 \texttt{zs...}.
			}
			~\\~\\
			\begin{tabular} {c|c}
			\multicolumn{2}{c}{Matchstatus}\\
			\hline
			\verb~_x~      & \textcolor{green}{a}            \\
			\verb~_y~      & \textcolor{cyan}{2}            \\
			\verb~zs...~   & -            
			\end{tabular}
        \end{column}
		
        \begin{column}{0.3\textwidth}	
			\dirtree{%
			.1 \textcolor{green}{\texttt{prod}}.		
				.2 \textcolor{green}{\texttt{pow}}.
					.3 \textcolor{green}{\texttt{a}}.
					.3 \textcolor{cyan}{\texttt{2}}.
				.2 \texttt{pow}.
					.3 \texttt{sum}.
						.4 \texttt{3}.
						.4 \texttt{b}.
					.3 \texttt{4}.
				.2 \texttt{sum}.
					.3 \texttt{3}.
					.3 \texttt{b}.
			}
        \end{column}
	\end{columns}
\end{frame}

%suche _x alleine mit erstem pow
\begin{frame}[fragile, noframenumbering]{Beispiel}
	Matche $p =$\verb~ prod(pow(_x, _y), _x, zs...)~ in 
	$t =$\verb~ prod(pow(a, 2), pow(sum(3, b), 4), sum(3, b))~
	~\\
	\begin{columns}[c] %t für align nach oben
		\scriptsize
        \begin{column}{0.3\textwidth}	
			\dirtree{%
			.1 \textcolor{green}{\texttt{prod}}.
				.2 \textcolor{green}{\texttt{pow}}.
					.3 \textcolor{green}{\texttt{\char`_x}}.
					.3 \textcolor{green}{\texttt{\char`_y}}.
				.2 \textcolor{cyan}{\texttt{\char`_x}}.		
				.2 \texttt{zs...}.
			}
			~\\~\\
			\begin{tabular} {c|c}
			\multicolumn{2}{c}{Matchstatus}\\
			\hline
			\verb~_x~      & \textcolor{green}{a}            \\
			\verb~_y~      & \textcolor{green}{2}            \\
			\verb~zs...~   & -            
			\end{tabular}
        \end{column}
		
        \begin{column}{0.3\textwidth}	
			\dirtree{%
			.1 \textcolor{green}{\texttt{prod}}.		
				.2 \textcolor{green}{\texttt{pow}}.
					.3 \textcolor{green}{\texttt{a}}.
					.3 \textcolor{green}{\texttt{2}}.
				.2 \textcolor{cyan}{\texttt{pow}}.
					.3 \textcolor{cyan}{\texttt{sum}}.
						.4 \textcolor{cyan}{\texttt{3}}.
						.4 \textcolor{cyan}{\texttt{b}}.
					.3 \textcolor{cyan}{\texttt{4}}.
				.2 \texttt{sum}.
					.3 \texttt{3}.
					.3 \texttt{b}.
			}
        \end{column}
	\end{columns}
\end{frame}

%suche _x alleine mit erstem pow
\begin{frame}[fragile, noframenumbering]{Beispiel}
	Matche $p =$\verb~ prod(pow(_x, _y), _x, zs...)~ in 
	$t =$\verb~ prod(pow(a, 2), pow(sum(3, b), 4), sum(3, b))~
	~\\
	\begin{columns}[c] %t für align nach oben
		\scriptsize
        \begin{column}{0.3\textwidth}	
			\dirtree{%
			.1 \textcolor{green}{\texttt{prod}}.
				.2 \textcolor{green}{\texttt{pow}}.
					.3 \textcolor{green}{\texttt{\char`_x}}.
					.3 \textcolor{green}{\texttt{\char`_y}}.
				.2 \textcolor{cyan}{\texttt{\char`_x}}.		
				.2 \texttt{zs...}.
			}
			~\\~\\
			\begin{tabular} {c|c}
			\multicolumn{2}{c}{Matchstatus}\\
			\hline
			\verb~_x~      & \textcolor{green}{a}            \\
			\verb~_y~      & \textcolor{green}{2}            \\
			\verb~zs...~   & -            
			\end{tabular}
        \end{column}
		
        \begin{column}{0.3\textwidth}	
			\dirtree{%
			.1 \textcolor{green}{\texttt{prod}}.		
				.2 \textcolor{green}{\texttt{pow}}.
					.3 \textcolor{green}{\texttt{a}}.
					.3 \textcolor{green}{\texttt{2}}.
				.2 \textcolor{black}{\texttt{pow}}.
					.3 \textcolor{black}{\texttt{sum}}.
						.4 \textcolor{black}{\texttt{3}}.
						.4 \textcolor{black}{\texttt{b}}.
					.3 \textcolor{black}{\texttt{4}}.
				.2 \textcolor{cyan}{\texttt{sum}}.
					.3 \textcolor{cyan}{\texttt{3}}.
					.3 \textcolor{cyan}{\texttt{b}}.
			}
        \end{column}
	\end{columns}
\end{frame}


%neues mach für pow musterfaktor
\begin{frame}[fragile, noframenumbering]{Beispiel}
	Matche $p =$\verb~ prod(pow(_x, _y), _x, zs...)~ in 
	$t =$\verb~ prod(pow(a, 2), pow(sum(3, b), 4), sum(3, b))~
	~\\
	\begin{columns}[c] %t für align nach oben
		\scriptsize
        \begin{column}{0.3\textwidth}	
			\dirtree{%
			.1 \textcolor{green}{\texttt{prod}}.
				.2 \textcolor{cyan}{\texttt{pow}}.
					.3 \textcolor{black}{\texttt{\char`_x}}.
					.3 \textcolor{black}{\texttt{\char`_y}}.
				.2 \textcolor{black}{\texttt{\char`_x}}.		
				.2 \texttt{zs...}.
			}
			~\\~\\
			\begin{tabular} {c|c}
			\multicolumn{2}{c}{Matchstatus}\\
			\hline
			\verb~_x~      & -            \\
			\verb~_y~      & -            \\
			\verb~zs...~   & -            
			\end{tabular}
        \end{column}
		
        \begin{column}{0.3\textwidth}	
			\dirtree{%
			.1 \textcolor{green}{\texttt{prod}}.		
				.2 \textcolor{black}{\texttt{pow}}.
					.3 \textcolor{black}{\texttt{a}}.
					.3 \textcolor{black}{\texttt{2}}.
				.2 \textcolor{cyan}{\texttt{pow}}.
					.3 \textcolor{black}{\texttt{sum}}.
						.4 \textcolor{black}{\texttt{3}}.
						.4 \textcolor{black}{\texttt{b}}.
					.3 \textcolor{black}{\texttt{4}}.
				.2 \textcolor{black}{\texttt{sum}}.
					.3 \textcolor{black}{\texttt{3}}.
					.3 \textcolor{black}{\texttt{b}}.
			}
        \end{column}
	\end{columns}
\end{frame}

%neues mach für pow musterfaktor, matche _x neu
\begin{frame}[fragile, noframenumbering]{Beispiel}
	Matche $p =$\verb~ prod(pow(_x, _y), _x, zs...)~ in 
	$t =$\verb~ prod(pow(a, 2), pow(sum(3, b), 4), sum(3, b))~
	~\\
	\begin{columns}[c] %t für align nach oben
		\scriptsize
        \begin{column}{0.3\textwidth}	
			\dirtree{%
			.1 \textcolor{green}{\texttt{prod}}.
				.2 \textcolor{green}{\texttt{pow}}.
					.3 \textcolor{cyan}{\texttt{\char`_x}}.
					.3 \textcolor{black}{\texttt{\char`_y}}.
				.2 \textcolor{black}{\texttt{\char`_x}}.		
				.2 \texttt{zs...}.
			}
			~\\~\\
			\begin{tabular} {c|c}
			\multicolumn{2}{c}{Matchstatus}\\
			\hline
			\verb~_x~      & \textcolor{cyan}{\verb~sum(3, b)~}            \\
			\verb~_y~      & -            \\
			\verb~zs...~   & -            
			\end{tabular}
        \end{column}
		
        \begin{column}{0.3\textwidth}	
			\dirtree{%
			.1 \textcolor{green}{\texttt{prod}}.		
				.2 \textcolor{black}{\texttt{pow}}.
					.3 \textcolor{black}{\texttt{a}}.
					.3 \textcolor{black}{\texttt{2}}.
				.2 \textcolor{green}{\texttt{pow}}.
					.3 \textcolor{cyan}{\texttt{sum}}.
						.4 \textcolor{cyan}{\texttt{3}}.
						.4 \textcolor{cyan}{\texttt{b}}.
					.3 \textcolor{black}{\texttt{4}}.
				.2 \textcolor{black}{\texttt{sum}}.
					.3 \textcolor{black}{\texttt{3}}.
					.3 \textcolor{black}{\texttt{b}}.
			}
        \end{column}
	\end{columns}
\end{frame}

%neues mach für pow musterfaktor, matche _y neu
\begin{frame}[fragile, noframenumbering]{Beispiel}
	Matche $p =$\verb~ prod(pow(_x, _y), _x, zs...)~ in 
	$t =$\verb~ prod(pow(a, 2), pow(sum(3, b), 4), sum(3, b))~
	~\\
	\begin{columns}[c] %t für align nach oben
		\scriptsize
        \begin{column}{0.3\textwidth}	
			\dirtree{%
			.1 \textcolor{green}{\texttt{prod}}.
				.2 \textcolor{green}{\texttt{pow}}.
					.3 \textcolor{green}{\texttt{\char`_x}}.
					.3 \textcolor{cyan}{\texttt{\char`_y}}.
				.2 \textcolor{black}{\texttt{\char`_x}}.		
				.2 \texttt{zs...}.
			}
			~\\~\\
			\begin{tabular} {c|c}
			\multicolumn{2}{c}{Matchstatus}\\
			\hline
			\verb~_x~      & \textcolor{green}{\verb~sum(3, b)~}            \\
			\verb~_y~      & \textcolor{cyan}{\verb~4~}            \\
			\verb~zs...~   & -            
			\end{tabular}
        \end{column}
		
        \begin{column}{0.3\textwidth}	
			\dirtree{%
			.1 \textcolor{green}{\texttt{prod}}.		
				.2 \textcolor{black}{\texttt{pow}}.
					.3 \textcolor{black}{\texttt{a}}.
					.3 \textcolor{black}{\texttt{2}}.
				.2 \textcolor{green}{\texttt{pow}}.
					.3 \textcolor{green}{\texttt{sum}}.
						.4 \textcolor{green}{\texttt{3}}.
						.4 \textcolor{green}{\texttt{b}}.
					.3 \textcolor{cyan}{\texttt{4}}.
				.2 \textcolor{black}{\texttt{sum}}.
					.3 \textcolor{black}{\texttt{3}}.
					.3 \textcolor{black}{\texttt{b}}.
			}
        \end{column}
	\end{columns}
\end{frame}


%neuer versuch matche faktor _x
\begin{frame}[fragile, noframenumbering]{Beispiel}
	Matche $p =$\verb~ prod(pow(_x, _y), _x, zs...)~ in 
	$t =$\verb~ prod(pow(a, 2), pow(sum(3, b), 4), sum(3, b))~
	~\\
	\begin{columns}[c] %t für align nach oben
		\scriptsize
        \begin{column}{0.3\textwidth}	
			\dirtree{%
			.1 \textcolor{green}{\texttt{prod}}.
				.2 \textcolor{green}{\texttt{pow}}.
					.3 \textcolor{green}{\texttt{\char`_x}}.
					.3 \textcolor{green}{\texttt{\char`_y}}.
				.2 \textcolor{cyan}{\texttt{\char`_x}}.		
				.2 \texttt{zs...}.
			}
			~\\~\\
			\begin{tabular} {c|c}
			\multicolumn{2}{c}{Matchstatus}\\
			\hline
			\verb~_x~      & \textcolor{green}{\verb~sum(3, b)~}            \\
			\verb~_y~      & \textcolor{green}{\verb~4~}            \\
			\verb~zs...~   & -            
			\end{tabular}
        \end{column}
		
        \begin{column}{0.3\textwidth}	
			\dirtree{%
			.1 \textcolor{green}{\texttt{prod}}.		
				.2 \textcolor{cyan}{\texttt{pow}}.
					.3 \textcolor{cyan}{\texttt{a}}.
					.3 \textcolor{cyan}{\texttt{2}}.
				.2 \textcolor{green}{\texttt{pow}}.
					.3 \textcolor{green}{\texttt{sum}}.
						.4 \textcolor{green}{\texttt{3}}.
						.4 \textcolor{green}{\texttt{b}}.
					.3 \textcolor{green}{\texttt{4}}.
				.2 \textcolor{black}{\texttt{sum}}.
					.3 \textcolor{black}{\texttt{3}}.
					.3 \textcolor{black}{\texttt{b}}.
			}
        \end{column}
	\end{columns}
\end{frame}


%neuer versuch matche faktor _x
\begin{frame}[fragile, noframenumbering]{Beispiel}
	Matche $p =$\verb~ prod(pow(_x, _y), _x, zs...)~ in 
	$t =$\verb~ prod(pow(a, 2), pow(sum(3, b), 4), sum(3, b))~
	~\\
	\begin{columns}[c] %t für align nach oben
		\scriptsize
        \begin{column}{0.3\textwidth}	
			\dirtree{%
			.1 \textcolor{green}{\texttt{prod}}.
				.2 \textcolor{green}{\texttt{pow}}.
					.3 \textcolor{green}{\texttt{\char`_x}}.
					.3 \textcolor{green}{\texttt{\char`_y}}.
				.2 \textcolor{cyan}{\texttt{\char`_x}}.		
				.2 \texttt{zs...}.
			}
			~\\~\\
			\begin{tabular} {c|c}
			\multicolumn{2}{c}{Matchstatus}\\
			\hline
			\verb~_x~      & \textcolor{green}{\verb~sum(3, b)~}            \\
			\verb~_y~      & \textcolor{green}{\verb~4~}            \\
			\verb~zs...~   & -            
			\end{tabular}
        \end{column}
		
        \begin{column}{0.3\textwidth}	
			\dirtree{%
			.1 \textcolor{green}{\texttt{prod}}.		
				.2 \textcolor{black}{\texttt{pow}}.
					.3 \textcolor{black}{\texttt{a}}.
					.3 \textcolor{black}{\texttt{2}}.
				.2 \textcolor{green}{\texttt{pow}}.
					.3 \textcolor{green}{\texttt{sum}}.
						.4 \textcolor{green}{\texttt{3}}.
						.4 \textcolor{green}{\texttt{b}}.
					.3 \textcolor{green}{\texttt{4}}.
				.2 \textcolor{cyan}{\texttt{sum}}.
					.3 \textcolor{cyan}{\texttt{3}}.
					.3 \textcolor{cyan}{\texttt{b}}.
			}
        \end{column}
	\end{columns}
\end{frame}

%erfolg!
\begin{frame}[fragile, noframenumbering]{Beispiel}
	Matche $p =$\verb~ prod(pow(_x, _y), _x, zs...)~ in 
	$t =$\verb~ prod(pow(a, 2), pow(sum(3, b), 4), sum(3, b))~
	~\\
	\begin{columns}[c] %t für align nach oben
		\scriptsize
        \begin{column}{0.3\textwidth}	
			\dirtree{%
			.1 \textcolor{green}{\texttt{prod}}.
				.2 \textcolor{green}{\texttt{pow}}.
					.3 \textcolor{green}{\texttt{\char`_x}}.
					.3 \textcolor{green}{\texttt{\char`_y}}.
				.2 \textcolor{green}{\texttt{\char`_x}}.		
				.2 \texttt{zs...}.
			}
			~\\~\\
			\begin{tabular} {c|c}
			\multicolumn{2}{c}{Matchstatus}\\
			\hline
			\verb~_x~      & \textcolor{green}{\verb~sum(3, b)~}            \\
			\verb~_y~      & \textcolor{green}{\verb~4~}            \\
			\verb~zs...~   & \verb~pow(a, 2)~            
			\end{tabular}
        \end{column}
		
        \begin{column}{0.3\textwidth}	
			\dirtree{%
			.1 \textcolor{green}{\texttt{prod}}.		
				.2 \textcolor{black}{\texttt{pow}}.
					.3 \textcolor{black}{\texttt{a}}.
					.3 \textcolor{black}{\texttt{2}}.
				.2 \textcolor{green}{\texttt{pow}}.
					.3 \textcolor{green}{\texttt{sum}}.
						.4 \textcolor{green}{\texttt{3}}.
						.4 \textcolor{green}{\texttt{b}}.
					.3 \textcolor{green}{\texttt{4}}.
				.2 \textcolor{green}{\texttt{sum}}.
					.3 \textcolor{green}{\texttt{3}}.
					.3 \textcolor{green}{\texttt{b}}.
			}
        \end{column}
	\end{columns}
\end{frame}







%------------------------------------------------------------------------

\begin{frame}{Demo}
\end{frame}

%------------------------------------------------------------------------
%------------------------------------------------------------------------
%------------------------------------------------------------------------
\section{Zusammenfassung}
\begin{frame}{Zusammenfassung}
	\begin{itemize}
		\item{Terme als Baumstrukturen}
		\item{Ersetzungsregeln als Paare von Mustertermen}
		\item{Mustererkennung für assoziative und kommutative Funktionen teilweise rechenintensiv}
		\begin{itemize}
			\item{Assoziativität indirekt über Multi-Mustervariablen ausgedrückt}
			\item{Kommutativität direkt berücksichtigt}
			\item{Backtracking}
		\end{itemize}
	\end{itemize}
\end{frame}

%------------------------------------------------------------------------

\begin{frame}[plain, noframenumbering]
	\begin{center}
		\Huge
		Fragen?
	\end{center}
\end{frame}

%------------------------------------------------------------------------

\end{document}




























\documentclass[aspectratio=169]{beamer}
\usetheme{Frankfurt}

\usepackage[backend=biber,
style=alphabetic,
]{biblatex}

\usepackage{graphicx}

\usepackage{minted}
\usemintedstyle{borland}

\usepackage{mathtools}

\usepackage{tikz}
\usetikzlibrary{positioning,shapes.multipart}

\usepackage{multicol}

\setbeamertemplate{navigation symbols}{}
\setbeamertemplate{footline}[frame number]
\setbeamertemplate{headline}{}

\usepackage[english]{babel}
\usepackage{csquotes}

\title{Template Metaprogramming in C\texttt{++}}
\author{Bruno Borchardt and Filip Nguyen
\small
\texorpdfstring{\\Seminar: Modern Programming Languages\\TUHH}{}}
\date{January 11, 2021}

\addbibresource{quellen.bib}

\newcommand\blfootnote[1]{%
  \begingroup
  \renewcommand\thefootnote{}\footnote{#1}%
  \addtocounter{footnote}{-1}%
  \endgroup
}

\begin{document}

%------------------------------------------------------------------------

\begin{frame}[noframenumbering, plain]
\titlepage 
\end{frame}

%------------------------------------------------------------------------
%------------------------------------------------------------------------

\section{Introduction to C++ Templates}

\begin{frame}[fragile]{Introduction to C++ Templates}

    \begin{columns}[t] 
        
        \begin{column}{0.5\textwidth}
        
            \scriptsize
            \begin{block}{Challenges}
                \begin{minted}{cpp}
int max(int x, int y)
{
    return x > y ? x : y;
}

double max(double x, double y)
{
    return x > y ? x : y;
}

#define MAX(x, y) ((x) > (y) ? (x) : (y))

void swap(int& x, int& y)
{
    int temp = x;
    x = y;
    y = temp;
}
                \end{minted}
            \end{block}
        \end{column}
    
        \begin{column}{0.5\textwidth}
        
            \begin{itemize}
                \item Clean code
                \item No duplicate
                \item Reusability
            \end{itemize}
        
            \scriptsize
            \begin{block}{Possible Solution}
                \begin{minted}{cpp}
template <typename T>
T max(T x, T y)
{
    return x > y ? x : y;
}
template <class T>
void swap(T& x, T& y)
{
    T temp = x;
    x = y;
    y = temp;
}
                \end{minted}
            \end{block}
        \end{column}
        
    \end{columns}
    
\end{frame}

%------------------------------------------------------------------------

\begin{frame}[fragile]{Introduction to C++ Templates}

    \begin{columns}[t] 
        \begin{column}{0.5\textwidth}
        
            \begin{itemize}
                \item Constant values can be passed 
                \item Exposing T as "type"
                \item Most specific Specialization is used
            \end{itemize}
        
            \scriptsize
            \begin{block}{Structs and Classes}
                \begin{minted}{cpp}
template <typename T, int Size>
class Array
{
public:
    // Obsolete: typedef T type;
    using type = T; 
    T * get(); 
    void set(T * value);
    
private:
    T buffer[Size];
};
                \end{minted}
            \end{block}
        \end{column}
        
        \begin{column}{0.5\textwidth}
        
            \scriptsize
            \begin{block}{User Specialization}
                \begin{minted}{cpp}
template <int Size>
class Array<bool, Size>
{
public:
    using type = bool; 
    ...
private:
    ...
};
Array<bool, 20> booleanArray;
                \end{minted}
            \end{block}
            
            \begin{block}{'type' in Use}
                \begin{minted}{cpp}
template<typename TArray>
void modifyArray(TArray& data) 
{   // modify data using localVariable ...
    Array::type localVariable = {}; 
}   
                \end{minted}
            \end{block}
        \end{column}
    \end{columns}
    
\end{frame}

%------------------------------------------------------------------------

\begin{frame}[fragile]{Introduction to C++ Templates}

    \scriptsize
    \begin{block}{Syntax Error}
        \begin{minted}{cpp}
std::cout << max(5, 5.5) << "\n";
        \end{minted}
    \end{block}

    \begin{block}{Poor Solution}
        \begin{minted}{cpp}
template <typename T1, typename T2>
T1 max(T1 x, T2 y)
{
    return x > y ? x : y;
}
std::cout << max(5, 5.5) << "\n";
        \end{minted}
    \end{block}

    \begin{block}{Better Solution}
        \begin{minted}{cpp}
template <typename TReturnValue, typename T1, typename T2>
TReturnValue max(T1 x, T2 y, TReturnValue)
{
    return x > y ? x : y;
}
std::cout << max(5, 5.5, 0.0) << "\n"; // Misleading, but works.
        \end{minted}
    \end{block}

\end{frame}

%------------------------------------------------------------------------

\begin{frame}[fragile]{Introduction to C++ Templates}

    \scriptsize
    \begin{block}{Better Solution}
        \begin{minted}{cpp}
template <typename TReturnValue, typename T1, typename T2>
TReturnValue max(T1 x, T2 y, TReturnValue)
{
    return x > y ? x : y;
}
std::cout << max(5, 5.5, 0.0) << "\n"; // Misleading, but works.
        \end{minted}
    \end{block}
    
    \begin{block}{Explicit Specialization}
        \begin{minted}{cpp}
std::cout << max<double>(5, 5.5) << "\n";
std::cout << max<double, int, double>(5, 5.5) << "\n";
        \end{minted}
    \end{block}

    \begin{block}{Best Solution}
        \begin{minted}{cpp}
template <typename T1, typename T2>
typename If<(sizeof(T1) > sizeof(T2)), T1, T2>::type max(T1 x, T2 y)
{
    return x > y ? x : y;
}
std::cout << max(5, 5.5) << "\n";
        \end{minted}
    \end{block}

\end{frame}

%------------------------------------------------------------------------

\section{Metaprogramming Basics}

\begin{frame}[fragile]{The First Metaprogram}
    
\begin{columns}[t]
        \begin{column}{0.5\textwidth}
            \scriptsize
            \begin{block}{Haskell}
                \begin{minted}{haskell}
square :: Int -> Int
square x = x * x

nine = square 3
                \end{minted}
            \end{block}
            
            \begin{block}{C\texttt{++}}
                \begin{minted}{cpp}
template<int x>
struct Square 
{ 
    enum { value = x * x };  
};

const int nine = Square<3>::value;
                \end{minted}
            \end{block}
        \end{column}
        
        \begin{column}{0.5\textwidth}
            \begin{itemize}
                \item Enum identifiers have their associated values evaluated at compile time
                \pause
                \item Today written as "static constexpr $<$type$>$"
            \end{itemize}
            \scriptsize
            
            \begin{block}{Modern C\texttt{++}}
                \begin{minted}{cpp}
template<int x>
struct Square 
{ 
    static constexpr int value = x * x;  
};

constexpr int nine = Square<3>::value;

constexpr int square(int x) { return x * x; }
constexpr int anotherNine = square(3);
                \end{minted}
            \end{block}
        \end{column}
\end{columns}
    
\end{frame}

%------------------------------------------------------------------------

\begin{frame}[fragile]{Types as Data: List}
    
\begin{columns}[t] %t für align nach oben
        \begin{column}{0.6\textwidth}
            \scriptsize
            \begin{block}{Haskell}
                \begin{minted}{haskell}
data List a = Nil 
            | Cons a (List a)

types = Cons "bool" (Cons "int" (Cons "double" Nil))
                \end{minted}
            \end{block}
            
            \begin{block}{C\texttt{++}}
                \begin{minted}{cpp}
struct Nil;

template<typename Value, typename Tail>
struct Cons;

using types = Cons<bool, Cons<int, Cons<double, Nil>>>;
                \end{minted}
            \end{block}
        \end{column}
        
        \begin{column}{0.4\textwidth}
            \begin{itemize}
                \item "Nil" and "Cons" are only ever declared
                \item A list of types is a type itself
                \pause
                \item Today variadic templates simplify the syntax
            \end{itemize}
            \scriptsize
            
            \begin{block}{Modern C\texttt{++}}
                \begin{minted}{cpp}
template<typename...>
struct List;

using types = List<bool, int, double>;
                \end{minted}
            \end{block}
        \end{column}
\end{columns}

\end{frame}

%------------------------------------------------------------------------

\begin{frame}[fragile]{Make Everything a Type!}
\begin{columns}[t] 
        \begin{column}{0.5\textwidth}
            \scriptsize
            \begin{block}{Haskell}
                \begin{minted}{haskell}
data Nat = Zero
         | Successor Nat 



two = Successor (Successor Zero)
                \end{minted}
            \end{block}
        \end{column}

        \begin{column}{0.5\textwidth}
            \scriptsize
            \begin{block}{C\texttt{++}}
                \begin{minted}{cpp}
struct Zero;

template<typename Natural>
struct Successor;

using two = Successor<Successor<Zero>>;
                \end{minted}
            \end{block}
            ~\\~\\~\\~\\~\\~\\~\\~\\~\\~
        \end{column}
\end{columns}
\scriptsize
\end{frame}

%------------------------------------------------------------------------

\begin{frame}[fragile]{Make Everything a Type!}
    
    \addtocounter{framenumber}{-1}
\begin{columns}[t] 
        \begin{column}{0.5\textwidth}
            \scriptsize
            \begin{block}{Haskell}
                \begin{minted}{haskell}
data Nat = Zero
         | Successor Nat 



two = Successor (Successor Zero)

--Word is an unsigned integral type
nat :: Word -> Nat 
nat value | value /= 0 = 
    Successor (
        nat (value - 1))
    

nat 0 = Zero

nine = nat 9
                \end{minted}
            \end{block}
        \end{column}

        \begin{column}{0.5\textwidth}
            \scriptsize
            \begin{block}{C\texttt{++}}
                \begin{minted}{cpp}
struct Zero;

template<typename Natural>
struct Successor;

using two = Successor<Successor<Zero>>;

template<unsigned Value>
struct Nat { 
    using type = Successor<
        typename Nat<Value - 1u>::type>; 
};

template<>
struct Nat<0u> { using type = Zero; };

using nine = Nat<9u>::type;
                \end{minted}
            \end{block}
        \end{column}
\end{columns}

\end{frame}

%------------------------------------------------------------------------

\begin{frame}[fragile]{A More Practical Version}
    
    \scriptsize
    \begin{block}{Integral Constant}
        \begin{minted}{cpp}
template<class T, T v>
struct integral_constant {
    static constexpr T value = v;
    using value_type = T;
    using type = integral_constant; // using injected-class-name
    constexpr operator value_type() const noexcept { return value; }
    constexpr value_type operator()() const noexcept { return value; } //since c++14
};

template <bool B>
using bool_constant = integral_constant<bool, B>;

//same as: using true_type = integral_constant<bool, true>;
using true_type =  bool_constant<true>;

//same as: using false_type = integral_constant<bool, false>;
using false_type = bool_constant<false>;
        \end{minted}
    \end{block}
    
    \blfootnote{integral\_constant example implementation taken from cppreference.com}
\end{frame}

%-----------------------------------------------------------------

\begin{frame}[fragile]{Solving the 'max' Problem}

    \scriptsize
    \begin{block}{Best Solution}
        \begin{minted}{cpp}
template <typename T1, typename T2>
typename If<(sizeof(T1) > sizeof(T2)), T1, T2>::type max(T1 x, T2 y)
{
    return x > y ? x : y;
}
std::cout << max(5, 5.5) << "\n";
        \end{minted}
    \end{block}

    \begin{block}{}
        \begin{minted}{cpp}
template <bool Condition, typename TThen, typename TElse>
struct If
{
    using type = TThen;
};

template <typename TThen, typename TElse>
struct If<false, TThen, TElse>
{
    using type = TElse;
};
        \end{minted}
    \end{block}

\end{frame}

%-----------------------------------------------------------------

\begin{frame}[fragile]{Limitations}
    
    \begin{itemize}
        \item Compile-time infinite recursion
        \item Maximum of 17 level of recursive template instantiations is required by C\texttt{++} standard
        \item Compiler g\texttt{++} 3.4 stops after 17 levels 
        \item Compiler MSVC 6 runs until the resources provided are exhausted
    \end{itemize}

    \scriptsize
    
    \begin{block}{Missing Specialization}
        \begin{minted}{cpp}
template <int N>
struct Factorial
{
    static constexpr int value = N * Factorial<N - 1>::value;
};
constexpr int six = Factorial<3>::value;
        \end{minted}
    \end{block}

\end{frame}

%-----------------------------------------------------------------

\begin{frame}[fragile]{Iterating over a List}
    \scriptsize
    \begin{block}{Is Type 'Needle' contained?}
        \begin{minted}{cpp}
struct Nil;

template<typename Value, typename Tail>
struct Cons;


template<typename, typename>
struct InList;

template<typename Needle>
struct InList<Needle, Nil> { static constexpr bool value = false; };

template<typename Needle, typename Tail>
struct InList<Needle, Cons<Needle, Tail>> { static constexpr bool value = true; };

template<typename Needle, typename Head, typename Tail>
struct InList<Needle, Cons<Head, Tail>> 
{
    static constexpr bool value = InList<Needle, Tail>::value;
};
        \end{minted}
    \end{block}

\end{frame}

%------------------------------------------------------------------------

\begin{frame}{Conclusion}
    \begin{itemize}
        \item C\texttt{++} templates are turing complete
        \item Templates where not designed as own language
        
        \begin{itemize}
            \item Initially hard to read
            \item Long compile times
        \end{itemize}
        
        \item Newer C\texttt{++} standards improve language facilities
            \begin{itemize}
                \item Constexpr functions allow compile-time value computations without templates
                \item Parameter packs ease variadic syntax
                \item Template template parameters ease higher order function syntax
                \item Concepts bring gradual typing to templates
            \end{itemize}
    \end{itemize}
\end{frame}

%-----------------------------------------------------------------

\begin{frame}[fragile]{Question}
    
    What is a more familiar name for the 'Mystery' function?
    \scriptsize
    \begin{block}{}
        \begin{minted}{cpp}
template <int N>
struct Mystery
{
    static constexpr int value = Mystery<N - 2>::value + Mystery<N - 1>::value;
};

template<>
struct Mystery<1>
{
    static constexpr int value = 1;
};

template<>
struct Mystery<0>
{
    static constexpr int value = 0;
};

constexpr int five = Mystery<5>::value;
        \end{minted}
    \end{block}
\end{frame}

%------------------------------------------------------------------------

\begin{frame}{References}

    \nocite{*}
    
    \printbibliography
\end{frame}

%------------------------------------------------------------------------

\begin{frame}[fragile]{Higher Order Functions}
    \addtocounter{framenumber}{-1}
    \begin{columns}[t] 
        \begin{column}{0.5\textwidth}
            \scriptsize
            \begin{block}{Haskell}
                \begin{minted}{haskell}
data List a = Nil 
            | Cons a (List a)
            
            
map :: (a -> b) -> (List a) -> (List b)





map f (Cons head tail) = 
    Cons 
        (f head) 
        (map f tail)
        


map f Nil = Nil
                \end{minted}
            \end{block}
        \end{column}

        \begin{column}{0.5\textwidth}
            \scriptsize
            \begin{block}{C\texttt{++}}
                \begin{minted}{cpp}
struct Nil;
template<typename Head, typename Tail>
struct Cons;

template<template<typename> class, typename> 
struct Map;

template<template<typename> class F, 
    typename Head, typename Tail>
struct Map<F, Cons<Head, Tail>> 
{
    using type = Cons<
        typename F<Head>::type, 
        typename Map<F, Tail>::type>;
};

template<template<typename> class F>
struct Map<F, Nil> { using type = Nil; };
                \end{minted}
            \end{block}
        \end{column}
    \end{columns}
\end{frame}


%------------------------------------------------------------------------

\end{document}


\chapter{Umsetzung in C\texttt{++}} \label{secKernUmsetzungInCpp}

%.........................................................................
%.........................................................................
%.........................................................................
\section{Konzeptionelle Unterschiede}
Die in diesem Kapitel vorgestellte Umsetzung implementiert nicht exakt die bisher beschriebenen Strukturen. Der erste Unterschied ist, dass die Konzepte \emph{Funktionssymbol} und \emph{Konstantensymbol} hier nicht unterschieden werden. Da der Zweck der Implementierung zudem alleine in der Vereinfachung von Ausdrücken über den Komplexen Zahlen $\mathbb C$ liegt, ist die Menge an Symbolen zudem im Code nicht generisch gehalten. Die beiden wichtigsten Arten von Symbolen für Literale sind Komplexe Zahlen $z \in \mathbb C$, sowie Zeichenketten $c_1 c_2\dots c_n$ beliebiger Länge $n$. Die einzelnen Zeichen $c_i$ stammen dabei aus dem Alphabet $\Sigma$, welches aus Klein-und Großbuchstaben des lateinischen Alphabetes, Ziffern von $0$ bis $9$, den Apostroph \verb|'| und dem Unterstrich \verb|_| besteht. Die Ausnahme bildet das erste Zeichen $c_1$, welches ein Klein-oder Großbuchstabe des Lateinischen Alphabetes sein muss. Zu den Zeichenketten gehören insbesondere auch die Funktionssymbole. Mit der dadurch entstehenden Möglichkeit Funktionssymbole als Werte zu behandeln, ergibt es Sinn auch Funktionsanwendungen von dynamisch bestimmten Funktionssymbolen zuzulassen. Das erste Element $f$ des Funktionsanwendungstupels $(f, \elems t 0 {n-1})$ ist in der Umsetzung also kein Symbol, sondern ein Term. Mit den bis hier diskutierten Änderungen und $\Sigma^+$ als Bezeichnung für die Menge von Zeichenketten über dem beschriebenen Alphabet $\Sigma$ sähe die idealisierte Menge aller Literale in der Umsetzung im Kontrast zu Definition \ref{defTerm} so aus:

$$T \coloneqq \Sigma^+ \cup \mathbb C \cup \curl*{(\elems t 0 n)~|~ \elems t 0 n \in T)}.$$

Die wichtigste Erweiterung der Implementierung ist allerdings die der anonymen Funktion bekannt als \emph{Lambda}. Das Lambdakalkül von Church \cite{ChurchLambda36} definiert Terme verwandt mit den in dieser Arbeit diskutierten. Der wichtigste Unterschied ist, dass keine externe Interpretation für einen gegebenen Lambdaausdruck notwendig ist. Anstatt Funktionssymbole über Namen zu identifizieren und die Abbildungsvorschrift getrennt anzugeben, ist eine Lambdafunktion $f \in \Lambda$ ausschließlich durch ihre Abbildungsvorschrift identifiziert. Church erlaubt neben der Funktionsanwendung als Term lediglich Variablensymbole  $v \in V$ und Lambdafunktionen. Soll $T$ also eine Obermenge aller Ausdrücke im Lambdakalkül werden, müsste prinzipiell nur die Lambdafunktion selbst hinzugefügt werden. Die Menge der Variablen $V$, hier als \emph{Lambdaparameter} bezeichnet wird allerdings von den bisher erlaubten Zeichenketten in $\Sigma^+$ getrennt.

Muster sind in der Umsetzung auch entsprechend flexibler, schließlich sind sie eine Obermenge der Literale. Dazu kommt die Fähigkeit, Bedingungen an Mustervariablen zu stellen, um mögliche Matches weiter einzuschränken. Eine spezielle Form dieser Einschränkung ist dabei die der \emph{Wert-Mustervariable} $w \in W$, welche versucht Matches zu finden, die in Unterkapitel \ref{subsecNormalKombinieren} erlaubt werden, also Komplexe Zahlen wieder in Rechenausdrücke zu dekonstruieren. Näher behandelt wird die Wert-Mustervariable noch im folgenden Abschnitt \ref{subsecMustervariablen}. 
Zuletzt kann die Multi-Mustervariable aus Abschnitt \ref{subsecMulti} von hier an nicht mehr nur als abstrakte Idee gehandelt werden. Die Menge der Multi-Mustervariablen wird $X^*$ genannt. 

\begin{definition} \label{defKnotentypenMathe}
Die Menge der Terme $T$ ist in diesem Kapitel definiert als
$$T \coloneqq \Sigma^+ \cup \mathbb C \cup X \cup X' \cup X^* \cup W \cup V \cup \Lambda \cup \curl*{(\elems t 0 n)~|~ \elems t 0 n \in T, n > 0)}$$
mit
\begin{align*}
    \Sigma^+  &\coloneqq \text{Zeichenketten}\\
    \mathbb C &\coloneqq \text{Komplexe Zahlen}\\
    X         &\coloneqq \text{Mustervariablen}\\
    X^*       &\coloneqq \text{Multi-Mustervariablen}\\
    W         &\coloneqq \text{Wert-Mustervariablen}\\
    V         &\coloneqq \text{Lambdaparameter}\\
    \Lambda   &\coloneqq \text{Lambdafunktionen}.
\end{align*}
\end{definition}




%.........................................................................
%.........................................................................
%.........................................................................
\section{Lambdafunktionen} \label{subsecLambdafunktionen}
Die umgesetzten Funktionen erweitern die Definition von Church, indem die selbe Lambdaabstraktion auch mehrere Parameter erlaubt. Während der Ausdruck $\lambda x y . x(y)$ für Church also nur eine vereinfachte Schreibweise\footnote{Die Klammern um $y$ werden in der Literatur oft weggelassen, für diese Arbeit sind sie allerdings notwendig.} für den Ausdruck $\lambda x .\lambda y .x(y)$ darstellt, handelt es sich für die hier beschriebene Umsetzung um zwei verschiedene Funktionen. 
Ein Lambdaparameter ist in der Syntax nicht von anderen Symbolen unterscheidbar\footnote{siehe \ref{subsubsecLambdaSyntax}}, wird intern allerdings durch einen Index dargestellt. Die Darstellung unterscheidet sich von De Bruijn Indizierung \cite{deBruijn} darin, dass jede Instanz der selben Variable immer den selben Index hat.
 Jeder neu gebundene Parameter bekommt als Index die Anzahl der weiter außen bzw. in der selben Abstraktion vor ihm gebundenen Parameter. Zur Veranschaulichung wird hier der Index im Tiefsatz mitgeschrieben. Wichtig ist hervorzuheben, dass die hier nach wie vor dargestellten Namen ausschließlich der Übersichtlichkeit dienen und in der Umsetzung nicht gespeichert sind.
 Als Beispiel dient $\lambda x_0 y_1 .\lambda z_2 .x_0 + y_1 + z_2$. Das Zeichen $x$ wird in der äußersten Abstraktion zuerst gebunden, hat also keine Vorgänger und dementsprechend Index $0$. In der selben Abstraktion wird als zweiter Parameter weiter $y$ gebunden. Als Nachfolger von $x$ wird $y$ Index $1$ zugewiesen. Die Bindung von $z$ liegt innerhalb der Abbildungsvorschrift des äußeren Lambdas, $x$ und $y$ können also referenziert werden. Dementsprechend hat $z$ zwei Vorgänger und Index $2$. 
 
 Die Auswertung der Funktionsanwendung einer Lambdafunktion $f$ entspricht der $\beta$-Reduktion (von Church \cite{ChurchLambda36} als \emph{operation II} bezeichnet). Parameter von $f$ werden dafür durch die übergebenen Argumente ersetzt. Enthält die Definition von $f$ selbst weitere Lambdaabstraktionen, so werden die Indices derer Parameter um die Stelligkeit von $f$ erniedrigt.
 Der Ausdruck $(\lambda x_0 y_1 .\lambda z_2 .x_0 + y_1 + z_2)(3, 6)$ wird damit zum Ausdruck $\lambda z_0 .3 + 6 + z_0$ ausgewertet.
 
 Die gewählte Indizierung hat ein Problem, welches die De Bruijn Indizierung nicht besitzt. Der Index eines Lambdaparameters ist abhängig vom Kontext der bindenden Abstraktion. Ändert sich dieser Kontext, stimmt die bisherige Indizierung möglicherweise nicht mehr:
 Enthält ein Argument $a$ der Funktionsanwendung einer Lambdafunktion $f$ selbst eine Lambdafunktion $g$, so reicht es für die Auswertung von $f$ mit bisher diskutierten Konzepten nicht aus, den entsprechenden Lambdaparameter von $f$ einfach durch $a$ zu ersetzen. Sollte diese Ersetzung innerhalb einer Lambdaabstraktion $f'$ stattfinden, müssen alle Lambdaparameter von $g$ im Index um die Anzahl der weiter außen gebundenen Parameter erhöht werden. Als Beispiel ist $a = g = \lambda x_0 .x_0$ die Identität, welche der Funktion $f = \lambda x_0 .\lambda y_1 .x_0(y_1)$ übergeben wird. Teil der Definition von $f$ ist $f' = \lambda y_1 .x_0(y_1)$.
 $$(\lambda x_0 .\lambda y_1 .x_0(y_1))(\lambda x_0 .x_0)$$
 Alternativ kann der selbe Audruck dargestellt werden, ohne Information zu benutzen
 An dieser Stelle wird zudem eine alternative Notation eingeführt, die nur Information benutzt, die der Implementierung tatsächlich bekannt ist. Ehrlicher ist die Darstellung einer Lambdaabstraktionen mit $n$ Parametern als $\lambda [n] .\textit{<Definition>}$. Mit den Parametern geschrieben als Prozentzeichen gefolgt von ihrem Index, ist der problematische Ausdruck geschrieben als
 
 $$(\lambda [1] .\lambda [1] .\%0(\%1))(\lambda [1] .\%0)$$.
 Die Naive Ersetzung von $x_0$ in $f$ durch $g$ würde im folgenden Ausdruck resultieren.
 $$\lambda y_0 .(\lambda x_0 .x_0)(y_0)$$
 Ohne Namen ist jetzt klar, dass in der Abbildungsvorschrift von $g$ nun fälschlicherweise auf den Parameter von $f'$ abgebildet wird:
 $$\lambda [1] .(\lambda [1] .\%0)(\%0)$$
 Es gibt verschiedene Möglichkeiten den Fehler zu beheben. Die wohl einfachste Lösung ist Indices von eingesetzten Argumenten in der Auswertung einer Lambdafunktion entsprechend anzupassen. Das würde allerdings bedeuten, dass die Komplexität der Auswertung von der Größe der Argumente abhängig ist. Ein weiterer Nachteil ist, dass veränderte Argumente kopiert werden müssen. Das macht \emph{lazy evaluation} \cite{EvalStrategien}, also die Idee jede Funktionsanwendung höchstens ein Mal auswerten zu müssen, unmöglich.
 
 Eine Alternative ist die Unterscheidung zwischen sogenannten \emph{transparenten} Lambdas und \emph{nicht-transparenten} Lambdas, zweitere dargestellt durch umschließende geschweifte Klammern: $\{\lambda \textit{<Parameter>}.\textit{<Definition>}\}$, bzw. $\{\lambda [n].\textit{<Definition>}\}$. Kommt ein transparentes Lambda $f'$ in der Definition eines Lambdas $f$ vor, wird bei der Auswertung von $f$ auch in $f'$ nach Lambdaparametern gesucht und diese ersetzt, bzw. deren Index erniedrigt. Transparente Lambdas verhalten sich damit nicht anders als die Lambdas bisher. Ist $f'$ allerdings nicht-transparent, lässt die Auswertung von $f$ die Definition von $f'$ unberührt. 
 Mit zwei weiteren Bedingungen wird die Überprüfung von Argumenten bei der Auswertung Lambdas dann obszolet: 
 Zum einen darf ein \emph{außenliegendes} Lambda $f$ nicht transparent sein. Ein Lambda ist außenliegend, wenn es nicht Teil der Definition einem weiteren Lambdas ist. Zum anderen darf die Funktionsanwendung eines Lambdas nur ausgewertet werden, wenn das Lambda außenliegend ist. Damit wird garantiert, dass während der Auswertung des Lambdas $f$ kein Argument $a$ direkt transparente Lambdas enthält, höchtens als Teil $g'$ (im Beispiel nicht enthalten) eines nicht-transparenten Lambdas $g$.

Im bereits behandelten Problemfall sind $f$ und $g$ damit nicht-transparent. Nur $f'$ ist zu Beginn transparent, wird allerdings nach Auswertung von $f$ ebendfalls zur äußersten Lambdaabstraktion, verliert also seine Transparenz. 
\begin{align*}
    ~           &~\{\lambda x_0 .\lambda y_1 .x_0(y_1)\}(\{\lambda x_0 .x_0\}) \\
    \rightarrow &~\{\lambda y_0 .\{\lambda x_0 .x_0\}(y_0)\}\\
\end{align*}
Alternativ ohne Namen:
\begin{align*}
    ~           &~\{\lambda [1] .\lambda [1] .\%0(\%1)\}(\{\lambda [1] .\%0\}) \\
    \rightarrow &~\{\lambda [1] .(\{\lambda [1] .\%0\})(\%0)\}\\
\end{align*}
Bemerkenswert ist, dass mit den soeben formulierten Bedingungen der Ausdruck nicht weiter transformiert werden darf. Das liegt daran, dass die Funktionsanwendung von $g$ selbst noch vom Lambda $f'$ umschlossen ist. Es ist also nicht ausgeschlossen, dass Argumente von $g$ (hier nur $\%0$) direkt transparente Lambdas enthalten\footnote{Eng verwandt ist der Grund, warum auch bei der Verwendung von De Bruijn Indices ohne weiteres nur die äußerste Funktionsanwendung abgebildet werden darf. In dem Fall könnte sich sonst die Anzahl der Abstraktionen zwischen einer Variable in einem Argument und ihrer eigenen Abstraktion ändern. Der alte De Bruijn Index würde die Variable dann der falschen Abstraktion zuordnen.}.



%.........................................................................
%.........................................................................
%.........................................................................
\section{Mustervariablen} \label{subsecMustervariablen}
In der Umsetzung wird zwischen drei Arten von Mustervariablen unterschieden. Die Sonderform der Multi-Mustervariable ist bereits aus Kapitel \ref{subsecMulti} bekannt, die Wert-Mustervariable ist allerdings neu. Als drittes ist die normale Mustervariable nach wie vor vorhanden. Wert-Mustervariable und normale Mustervariable können zudem durch Bedingungen eingeschränkt werden.

\subsection{Wert-Mustervariable} \label{subsubsecWertMustervariable}
Nach Definition \ref{defMatch} ist Match $v_p$ für ein Paar $(p, t)$ aus Muster $p$ und Literal $t$ dann gültig, wenn die Normalform von $p$ mit den Mustervariablen ersetzt duch die Funktionswerte von $v_p$ identisch zu $t$ ist. In Kapitel \ref{secPattermatching} nicht diskutiert wurde die Möglichkeit Zahlen mit Rechenausdrücken zu matchen. Ein Muster $p = (\texttt{prod}, 2, \mathbf k)$ würde für das Literal $t = 12$ etwa ein Match $v_p$ mit $v_p~\mathbf k = 6$ besitzen. Um die Matchsuche nicht zu kompliziert zu gestalten, besitzt die Umsetzung zwei Einschränkungen. Zum einen ist eine solche Dekonstruktion des Musters zu einem Wert nur dann möglich, wenn das entsprechende Teilmuster exakt eine Instanz $w$ einer Wert-Mustervariable enthält. Zum anderen werden nur bijektive Funktionen\footnote{bzw. sonst Mehrdeutigkeiten ignoriert} dekonstruiert, wie im Beispiel die Multiplikation mit $2$. 


Vor einem Matchversuch muss ein Muster mit Wert-Mustervariablen zuerst entsprechend transformiert werden. Ziel der Transformation ist, das für den Matchalgorithmus direkt klar ist, ab wo das Muster einen Wert repräsentieren soll. Der Teilbaum des Musters, der eine Wert-Mustervariable $w$ enthält und als ganzes einen Wert matchen soll, ist idealerweise also direkt gekennzeichnet. In der Umsetzung ist das realisiert, indem eine Wert-Mustervariable im zu matchenden Muster kein Blatt darstellt, sondern einen Teilbaum markiert. Dieser entspricht allerdings nicht der ursprünglichen Umgebung um die Wert-Mustervariable, sondern der Inversen. 

\begin{figure}
\begin{verbatim}
value_match__(_i, _dom, _match) + _a + cs... | _a :complex 
    = value_match__(_i, _dom, _match - _a) + cs...
        
value_match__(_i, _dom, _match) * _a * cs... | _a :complex 
    = value_match__(_i, _dom, _match / _a) * cs...
        
value_match__(_i, _dom, _match) ^ 2                        
    = value_match__(_i, _dom, sqrt(_match))
        
sqrt(value_match__(_i, _dom, _match))                           
    = value_match__(_i, _dom, _match ^ 2)
\end{verbatim}
\label{figWertMusterBau}
\caption{Transformationsregeln von Mustern mit Wert-Mustervariable}
\end{figure}

Die Transformation in diese Form erfolgt über Anwendung von Ersetzungsregeln. Die Wert-Mustervariable wird dazu als Funktionsanwendung des Symbols \verb|value_match__| gespeichert. Die Regeln sind in Abbildung \ref{figWertMusterBau} aufgeführt. Das Funktionssymbol erwartet drei Argumente, hier die Mustervariablen \verb|_i| für den identifikationsindex, \verb|_dom| für den erlauben Raum (kurz für \emph{domain}) und \verb|_match| für den Teilterm, der im Matchalgorithmus den Wert bestimmt. 


\subsection{Bedingungen} \label{subsubsecBedingungen}
\BFred{TODO}

%.........................................................................
%.........................................................................
%.........................................................................
\section{Syntax} \label{subsecSyntax}
In der Umsetzung wird der Termbaum immer aus einer ASCII Zeichenkette gebaut.
Solche Zeichenketten sind von hier an in \texttt{monospace} gesetzt. Auf eine formale Definition der genutzten Grammatik wird verzichtet.

\subsection{Funktionsanwendungen}
\begin{figure}
    \label{tabZucker}
    \centering
    \begin{tabular}{l l}
        \hline
        Normale Schreibweise & Alternative Syntax\\
        \hline \hline
        \verb|prod(-1, x)|         & \verb|-x|\\
        \verb|not(x)|              & \verb|!x|\\
        \verb|pow(x, y)|           & \verb|x^y|\\
        \verb|prod(x, y)|          & \verb|x * y|\\
        \verb|prod(x, y)|          & \verb|x y|\\
        \verb|prod(x, pow(y, -1))| & \verb|x / y|\\
        \verb|sum(x, y)|           & \verb|x + y|\\
        \verb|sum(x, prod(-1, y))| & \verb|x - y|\\
        \verb|cons(x, y)|          & \verb|x :: y|\\
        \verb|eq(x, y)|            & \verb|x == y|\\
        \verb|neq(x, y)|           & \verb|x != y|\\
        \verb|greater(x, y)|       & \verb|x > y|\\
        \verb|smaller(x, y)|       & \verb|x < y|\\
        \verb|greater_eq(x, y)|    & \verb|x >= y|\\
        \verb|smaller_eq(x, y)|    & \verb|x <= y|\\
        \verb|and(x, y)|           & \verb|x && y|\\
        \verb|or(x, y)|            & \verb!x || y!\\
        \verb|of_type__(x, y)|     & \verb|x :y|\\
        \hline
    \end{tabular}
    \caption{alternative Syntax für bestimmte Funktionssymbole}
\end{figure}

Die bisher genutzte Schreibweise $(f, x, y, z)$ für die Funktionsanwendung des Funktionssymbols $f$ auf die Parameter $x, y, z$ wurde in Abschnitt \ref{subsecTerm} eingeführt, um den Term syntaktisch von anderen Ideen zu differenzieren. Dies ist für die Umsetzung in C\texttt{++} nicht notwendig, da von vorne herein klar ist, ob eine Zeichenkette in einen Term übersetzt wird. Aus dem Grund werden Funktionsanwendungen in der Syntax \verb|f(x, y, z)| geparst. 
Weiter können Funktionsanwendungen bestimmter Funktionen auch mit fest definierten Infixoperatoren geschrieben werden, siehe Tabelle \ref{tabZucker}. Die Bidekraft der Operatoren nimmt in der Tabelle von oben nach unten ab.
Aus der Tabelle hervorzuheben sind zwei Dinge: Zum einen wird ein Leerzeichen zwischen zwei Termen als Multiplikation interpretiert, Funktionsanwendungen dürfen also kein Leerzeichen zwischen Funktion und Parametertupel setzen. \verb|f (x)| ist dementsprechend gleichbedeutend zu \verb|prod(f, x)|, während \verb|f (x, y)| einen Syntaxfehler darstellt. Die zweite Besonderheit ist das Fehlen von Funktionssymbolen für die Darstellung des additiven Inversen und des multiplikativen Inversen. Dies reduziert die Anzahl der Ersetzungsregeln, die benötigt werden um eine Gesetzmäßigkeit, die Summen oder Produkte involviert, abzubilden. 

\subsection{Lambdafunktionen} \label{subsubsecLambdaSyntax}
Da der ASCII Zeichensatz keine griechischen Buchstaben enthält, wird anstelle des kleinen Lambdas $\lambda$ der umgekehrte Schrägstrich \verb~\ ~genutzt. Die Identitätsfunktion $\lambda x.x$ wird dementsprechend \verb~\x .x~ geschrieben. Mehrere Parameter werden durch Leerzeichen getrennt: \verb~\x y .pow(x, y)~ ist eine Lambdafunktion mit identischem Verhalten zum Funktionssymbol \verb|pow|.

\subsection{Symbole}
Komplexe Zahlen auf der reellen Achse sind in der Syntax vergleichbar mit Darstellungen für Integer und Fließkommazahlen erlaubt in C. Möglich sind etwa \verb|3.1415|, \verb|42|, \verb|1.337e3| oder \verb|1.602e-19|. Komplexe Zahlen auf der imaginären Achse sind in der Struktur identisch, allerdings immer direkt gefolgt vom Zeichen \verb|i|. 

Die verschiedenen Mustervariablen werden auf unterschiedliche Weise identifiziert. Eine normale Mustervariable muss mit einem Unterstrich beginnen (\verb|_x|), eine Wert-Mustervariable mit einem Dollar (\verb|$x|) und eine Multi-Mustervariable endet in drei Punkten (\verb|xs...|). Besitzt ein Name keiner dieser besonderen Merkmale, wird daraus kontextabhängig ein Lambdaparameter oder ein Symbol aus $\Sigma^+$. Ist der umgebende Name bereits in einem umschließenden Lambda gebunden, wird die innerste solche Bindung gewählt. Als Beispiel ist im Ausdruck \verb|\x .x + y| das Symbol \verb|x| als Lambdaparameter interpretiert, während \verb|y| Als Zeichenkette erhalten bleibt.
Im Ausdruck \verb|\x .\x .2 - x| kann der Parameter \verb|x| des äußeren Lambdas in der Definition des inneren Lambdas nicht mehr referenziert werden, der Gesamtausdruck ist damit identisch zu \verb|(\x .(\y .2 - y))|.


\subsection{Ersetzungsregel}
Es gibt zwei Varianten Ersetzungsregeln zu schreiben.
\begin{align*}
	~&~&~&~&\textit{<linke Seite>}&~ ~                       &= \textit{<rechte Seite>}&~&~&~&~\\
	~&~&~&~&\textit{<linke Seite>}&~|~\textit{<Bedingungen>} &= \textit{<rechte Seite>}&~&~&~&~
\end{align*}
In der zweiten Variante ist \textit{<Bedingungen>} eine kommaseparierte Liste von Bedingungen and die vorkommenden Mustervariablen, wie in Abschnitt \ref{subsubsecBedingungen} diskutiert. 



%.........................................................................
%.........................................................................
%.........................................................................
\section{Datenstruktur}
Die Knoten des Termbaumes liegen in einem Array, dem sogenannten \verb|Store|. Während eine Komplexe Zahl oder ein Lambdaparameter immer ein Arrayelement besetzen, kann die Funktionsanwendung auch mehrere direkt hintereinanderliegende Arrayelemente nutzen, je nachdem, wie viele Argumente referenziert werden müssen. Referenziert wird ein Knoten aus einem Paar aus Arrayindex und Knotentyp, ebendfalls dargestellt als natürliche Zahl. Dieses Paar wird \verb|NodeIndex| genannt. Der Knotentyp wird als \verb|NodeType| bezeichnet und kann 12 verschiedene Werte annehmen, eng verwandt mit Definition \ref{defKnotentypenMathe}:
\begin{enumerate}
	\setcounter{enumi}{-1} %beginne aufzählung bei 0
	\item {$\Sigma^+$.}
	\item {Komplexe Fließkommazahl, zusammengesetzt aus zwei 64-bit Fließkommazahlen für Realteil und Imaginärteil}
	\item {$\Lambda$}
	\item {$V$}
	\item {Funktionsanwendung in einem Literal}
	\item {Funktionsanwendung in einem Muster}
	\item {Hilfsyp benötigt in der Konstruktion eines Musters mit Wert-Mustervariablen}
	\item {bindende Instanz einer Mustervariable $\mathbf x \in X$ an die im Muster Bedingungen geknüpft sind}
	\item {bindene Instanz einer Mustervariable $\mathbf x \in X$ ohne weitere Bedingungen}
	\item {gebundene Instanz einer Mustervariable $\mathbf x \in X$}
	\item {$X^*$}
	\item {$W$}
\end{enumerate}
Manche der hier aufgezählten Knoten müssen neben ihrem Typen nur eine einzige natürliche Zahl kennen. Bereits diekutiert wurde das für Knoten von Typ $3$, den Lambdaparametern. Für ein Knoten dieser Art ist es nicht notwendig, die Information als eigenens Arrayelement des \verb|Store| zu speichern, der Index des Paares aus Index und Typ von \verb|NodeType| enthält direkt den erforderlichen Wert. Alle Knoten dieser Art

\BFred{TODO}

%.........................................................................
%.........................................................................
%.........................................................................
\section{Algorithmen} \label{subsecCppAlgos}
\BFred{TODO}

















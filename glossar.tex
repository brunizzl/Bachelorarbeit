

\makeglossaries

\newglossaryentry{Funktionsanwendung}
{
    name = {Funktionsanwendung},
    description = {Art von Term: Tupel $(f, \elems t 0 {n-1})$ aus Funktionssymbol ${f \in F}$ und Argumenten ${\elems t 0 {n-1} \in T}$. die Stelligkeit von $f$ beträgt $n$ oder $f$ ist variadisch.}
}

\newglossaryentry{Parameter}
{
    name = Parameter,
    description = {Platzhalter für eine Koordinate des Definitionsbereiches einer Funktion}
}

\newglossaryentry{Argument}
{
    name = Argument,
    plural = Argumente,
    description = {Koordinate des Definitionsbereiches einer Funktion}
}

\newglossaryentry{Interpretation}
{
    name = Interpretation,
    description = {Funktion, die Termen oder Teilen von Termen eine Bedeutung zuweist}
}

\newglossaryentry{Konstantensymbol}
{
    name = Konstantensymbol,
    description = {(nicht-rekursiver) Term, Element in $C$}
}

\newglossaryentry{Funktionssymbol}
{
    name = Funktionssymbol,
    description = {Tag, welcher die Funktion in einer Funktionsanwendung repräsentiert, Element in $F$}
}

\newglossaryentry{Term}
{
    name = Term,
    description = {Baumstruktur mit Funktionsanwendungen als innere Knoten und Konstantensymbolen als Blätter}
}

\newglossaryentry{Auswertung}
{
    name = Auswertung,
    description = {Funktionswert einer Interpretation}
}

\newglossaryentry{Stelligkeitsfunktion}
{
    name = Stelligkeitsfunktion,
    description = {$\Const{arity} \colon F \rightarrow \mathbb N \cup \omega$ bildet ein Funktionssymbol $f$ auf die erwartete Anzahl seiner Argmente ab},
}

\newglossaryentry{variadisch}
{
    name = variadisch,
    description = {Ist das Funktionssymbol $f$ variadisch, kann es auf beliebig viele Argumente angewendet werden. Schreibe dann $\Const{arity}~f = \omega$.},
}

\newglossaryentry{Nachkomme}
{
    name = Nachkomme,
    description = {Kind des aktuellen Terms oder Kind eines Nachkommen des aktuellen Terms},
}

\newglossaryentry{Kind}
{
    name = Kind,
    description = {Argument der aktuellen Funktionsanwendung},
}


\newglossaryentry{Vater}
{
    name = Vater,
    description = {Funktionsanwendung, die den aktuellen Term als Argument enthält},
}

\newglossaryentry{Ahne}
{
    name = Ahne,
    description = {Vater des aktuellen Terms oder Ahne des Vaters des aktuellen Terms},
}

\newglossaryentry{Konstruktor}
{
    name = Konstruktor,
    description = {Interpretation $u_c$ eines Funktionssymbols als Tag des Argumenttupels},
}


\newglossaryentry{Literal}
{
    name = Literal,
    description = {zu transformierender Term oder Teil davon},
}

\newglossaryentry{Muster}
{
    name = Muster,
    description = {Term in $M(F, C)$ als Teil einer Ersetzungsregel},
}

\newglossaryentry{Mustervariable}
{
    name = Mustervariable,
    plural = Mustervariablen,
    description = {Konstantensymbol in $X$ mit Platzhalterrolle in einem Muster},
}

\newglossaryentry{Ersetzungsregel}
{
    name = Ersetzungsregel,
    description = {Paar von Mustern. Das linke Muster soll in einem Literal durch das rechte Muster ersetzt werden},
}

\newglossaryentry{Match}
{
    name = Match,
    description = {Interpretation $v_p \colon X \rightarrow T$ von Mustervariablen als Literale mit $\Const{lit}(p, v_p) = t$ für Muster $p$ und Literal $t$},
}

\newglossaryentry{Musterinterpretation}
{
    name = Musterinterpretation,
    description = {Interpretation $\Const{lit} \colon \paren*{X \rightarrow T} \times M \rightarrow T$ von Mustern als Literale via Match $v_p$},
}

\newglossaryentry{Matchalgorithmus}
{
    name = Matchalgorithmus,
    description = {Verfahren um ein Match zu finden},
}

\newglossaryentry{Normalform}
{
    name = Normalform,
    description = {Element in Bild von $\Const{normalize}$ oder Literal, auf das keine Ersetzungsregel mehr angewendet werden kann},
}

\newglossaryentry{konfluent}
{
    name = konfluent,
    description = {keine zwei linke Seiten unterschiedlicher Regeln können das selbe Literal matchen, wenn die rechten Seiten ungleich sind},
}

\newglossaryentry{Lambda}
{
    name = Lambda,
    description = {auch Lambdafunktion oder Lambdaabstraktion, Funktion mit lokal definierter Abbildungsvorschrift},
}

\newglossaryentry{Lambdaparameter}
{
    name = Lambdaparameter,
    description = {Parameter einer Lambdafunktion, Element in $V$},
}

\newglossaryentry{Wert-Mustervariable}
{
    name = Wert-Mustervariable,
    description = {spezielle Mustervariable, die erlaubt Funktionsanwendungen als Muster mit Konstantensymbolen zu matchen},
}

\newglossaryentry{Multi-Mustervariable}
{
    name = Multi-Mustervariable,
    description = {spezielle Mustervariable, die mehrere Argumente einer Funktionsanwendung matchen kann},
}

\newglossaryentry{transparent}
{
    name = transparent,
    plural = transparenten,
    description = {Lambdafunktion hat möglicherweise nicht-eigene Lambdaparameter in Abbildungsvorschrift},
}





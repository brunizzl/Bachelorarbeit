\documentclass[
%%%%% Styles and Sizes
%10pt,
%11pt,
%12pt,
fancyheadings, % headings with seplines and logo
%
%%%%% Printing, Color and Binding
%a4paper, 
%a5paper,
%twoside, % duplex printout (default)
oneside, % single sided printout
%% binding correction is used to compensate for the paper lost during binding
%% of the document
%BCOR=0.7cm, % binding correction
%nobcorignoretitle, % do not ignore BCOR for title page
%% the following two options only concern the graphics included by the document
%% class
%grayscaletitle, % keep the title in grayscale
%grayscalebody, % keep the rest of the document in grayscale
%
%%%%% expert options: your mileage may vary
%baseclass=..., % special option to use a different document baseclass
bibliography=totoc %inhaltsverzeichnis enthaelt literatur
]{stsreprt}

\usepackage[utf8]{inputenc}

% Information for the Titlepage
\subject{Bachelorarbeit}
\professor{PD Dr. Prashant Batra}
\advisor{Prof. Dr. Siegfried Rump}
\title{Entwicklung eines Termersetzungssystems für assoziative und kommutative Ausdrücke}
\author{Bruno Borchardt}
\date{\today}

% Font and Fontencoding Magic
% FAQ: 
% http://tex.stackexchange.com/questions/664/why-should-i-use-usepackaget1fontenc
% http://en.wikipedia.org/wiki/Computer_Modern
% http://tex.stackexchange.com/questions/1390/latin-modern-vs-cm-super
\usepackage[T1]{fontenc}
\usepackage{lmodern}
%\usepackage{fix-cm}

\usepackage{csquotes}
\usepackage[ngerman]{babel}

\usepackage{graphicx}
\usepackage{mathtools}
\usepackage{amssymb}
\usepackage{amsmath} %  \begin{cases}  Wert 1 & Bedingung 1 \\ Wert 2 & Bedingung 2 \\ \end{cases} 
\numberwithin{figure}{section} %label nr. beinhaltet section

\usepackage{qtree}

 %option linesnumbered schlägt sich mit goto labeln
\usepackage[german, ruled, vlined]{algorithm2e} 

\usepackage[backend=biber, style=alphabetic]{biblatex}
\addbibresource{quellen.bib}
%\bibliographystyle{plain}
%\bibliography{references}

\usepackage{xcolor} %\textcolor{blue}{This is a sample text in blue.}

\usepackage{minted} %code
\usemintedstyle{friendly}
\renewcommand{\listingscaption}{Quelltext}


%toc fügt eintrag in inhaltsverzeichnis ein
%nopostdot setzt keinen Punkt nach beschreibung
%altlist setzt \n nach name
\usepackage[style = list, toc, nopostdot]{glossaries}
\renewcommand*\glspostdescription{\hfill} %seitenzahlen rechtsbuendig


\makeglossaries

\newglossaryentry{Funktionsanwendung}
{
    name = {Funktionsanwendung},
    description = {Art von Term: Tupel $(f, \elems t 0 {n-1})$ aus Funktionssymbol ${f \in F}$ und Argumenten ${\elems t 0 {n-1} \in T}$. die Stelligkeit von $f$ beträgt $n$ oder $f$ ist variadisch.}
}

\newglossaryentry{Parameter}
{
    name = Parameter,
    description = {Platzhalter für eine Koordinate des Definitionsbereiches einer Funktion}
}

\newglossaryentry{Argument}
{
    name = Argument,
    description = {Koordinate des Definitionsbereiches einer Funktion}
}

\newglossaryentry{Interpretation}
{
    name = Interpretation,
    description = {Funktion, die Termen oder Teilen von Termen eine Bedeutung zuweist}
}

\newglossaryentry{Konstantensymbol}
{
    name = Konstantensymbol,
    description = {(nicht-rekursiver) Term, Element in $C$}
}

\newglossaryentry{Funktionssymbol}
{
    name = Funktionssymbol,
    description = {Tag welches die Funktion in einer Funktionsanwendung repräsentiert, Element in $F$}
}

\newglossaryentry{Term}
{
    name = Term,
    description = {Baumstruktur mit Funktionsanwendungen als innere Knoten und Konstantensymbolen als Blätter}
}

\newglossaryentry{Auswertung}
{
    name = Auswertung,
    description = {Funktionswert einer Interpretation}
}






\setlength{\parindent}{0pt} %keine Einrückung nach absatz

\usepackage{amsthm}
\theoremstyle{definition} %keine kursiven theoreme

%subsections nicht im Inhaltsverzeichnis
\setcounter{tocdepth}{1}
%\setcounter{secnumdepth}{2}




%.........................................................................
%................................ Macros .................................
%.........................................................................

% baut teile eines tupels: "t_1, ..., t_n"
\newcommand{\elems}[3]{{#1}_{#2}, \dots, {#1}_{#3}}
\newcommand{\tOneN}{\elems t 1 n}

% stapelt zweiten parameter auf ersten
\newcommand{\stapel}[2]{\mathrel{\overset{\makebox[0pt]{\mbox{\normalfont\tiny\sffamily {#2}}}}{#1}}}

%gibt text in fett und rot aus
\newcommand{\BFred}[1]{\textbf{\textcolor{red}{#1}}}

%\paren*{a + b} skaliert automatisch klammern um a + b
\DeclarePairedDelimiter\paren{(}{)} 
\DeclarePairedDelimiter\curl{\{}{\}}

%mathekonstanten wie funktionsnahmen sollen so gedruckt werden
%\newcommand{\Const}[1]{\mathrm{\underline{#1}}}
\newcommand{\Const}[1]{\mathit{#1}}

%befehle für algorithm2e

%in algorithm folgt nach "if" kein "then", nach "for" kein "do" ...
\SetKwIF{If}{ElseIf}{Else}{if}{ }{else if}{else}{}
\SetKwFor{For}{for}{}{}
\SetKwFor{While}{while}{}{}
\SetKwFor{Loop}{loop}{}{}
\SetKwRepeat{DoWhile}{do}{while}
\SetKwSwitch{Switch}{Case}{Other}{switch}{}{case}{otherwise}{endcase}{endsw}
\SetKw{Goto}{goto}
\SetKw{Let}{let}
\SetArgSty{text} %statt default emph
\newcommand{\KwAnd}{\textbf{and }}
\newcommand{\KwOr}{\textbf{or }}


%theorem umgebungen
\newtheorem{beispiel}{Beispiel}[chapter] %zähler zweistellig mit chapter
\newtheorem{definition}[beispiel]{Definition} %selber zähler wie beispiel
\newtheorem{lemma}[beispiel]{Lemma}           %selber zähler wie beispiel

%setze emph nicht kursiv, da so schon funktionssymbole aussehen (Const)
\newcommand{\Emph}[1]{\underline{#1}}


%eignet sich gut, um verbatim zu wrappen
\newenvironment{unbreakable}{\samepage}{\pagebreak[0]}



%.........................................................................
%................................ Silben .................................
%.........................................................................

\hyphenation{Addi-tion}
\hyphenation{Funk-tions-an-wen-dung}
\hyphenation{Funk-tions-an-wen-dung-en}
\hyphenation{Funk-tions-sym-bol}
\hyphenation{Funk-tions-sym-bole}
\hyphenation{Funk-tions-sym-bols}
\hyphenation{Kon-stan-ten-sym-bol}
\hyphenation{Kon-stan-ten-sym-bole}
\hyphenation{Kon-stan-ten-sym-bols}


%.........................................................................
%................................ Body ...................................
%.........................................................................
\begin{document}
\emergencystretch 2em %not go into margin as often

\maketitle

%subsections nicht im Inhaltsverzeichnis
\setcounter{tocdepth}{1}
\tableofcontents

\clearpage
%\cleardoublepage <- für andere Dokumenttypen



\chapter{Einleitung} \label{secEinleitung}

Die Manipulation symbolischer Ausdrücke ist in der Mathematik allgegenwärtig. Ist bekannt, dass ein Ausdruck $A$ gleichbedeutend zu einem zweiten Ausdruck $B$ ist, so kann in einem dritten Ausdruck $C$ jedes Vorkommen von $A$ durch $B$ ersetzt werden. 
Wenn bekannt ist, dass $A = 4$ und $B = 2 \cdot 2$ die gleiche Bedeutung haben, kann beispielsweise der Ausdruck $C = \frac{4}{2}$ auch als $C' = \frac{2 \cdot 2}{2}$ geschrieben werden.

Oft ist für zwei Ausdrücke bekannt, dass sie gleichbedeutend sind, wenn beide jeweils einer bestimmten Struktur folgen. Im Beispiel kann $C'$ auch als $C'' = 2$ geschrieben werden, da unabhängig konkreter Werte von $x$ und $y$ feststeht, dass der Ausdruck $\frac{x \cdot y}{x}$ auch als $y$ geschrieben werden kann.

Solche Regeln von Hand anzuwenden, ist sowohl zeitaufwändig als auch fehleranfällig. 
Die Idee, Computer zu nutzen, um symbolische Ausdrücke zu manipulieren, ist deshalb fast so alt wie der Computer selbst.  LISP ist als eine der ersten höheren Programmiersprachen bereits für diesen Zweck geschaffen worden \cite{lisp}. Rückblickend war die symbolische Berechnung sogar vor der Erfindung des Computers ein wichtiger Teil dessen, was heute theoretische Informatik genannt wird. Herausstechend ist die Idee des Lambdakalküls von Church \cite{ChurchLambda36}. 

Gegenwärtig gibt es eine Reihe etablierter Systeme zum symbolischen Rechnen. Wichtiger Bestandteil sind sie etwa in Mathematica \cite{MathematicaSymbolic}, Maple \cite{MapleSymbolic} oder Matlab \cite{MatlabSymbolic}.
Konzeptionelle Grundlage dieser Anwendungen ist die des \Emph{Termersetzungssystems}: Mit einer vom Nutzer oder Bibliotheksautor bestimmten Menge von Ersetzungsregeln wird ein gegebener Ausdruck so lange transformiert, bis keine Regel mehr anwendbar ist.


\begin{beispiel}~\\
Es werden vier Ersetzungsregeln definiert:
\begin{alignat*}{4}
    ~& x \cdot y + x \cdot z   & &= x \cdot (y + z) &~~~& (1) \\
    ~& \sqrt{x}                & &= x^{\frac 1 2}   &~~~& (2) \\
    ~& \paren{x^y}^z           & &= x^{y \cdot z}   &~~~& (3) \\
    ~& (\sin x)^2 + (\cos x)^2 & &= 1               &~~~& (4)
\end{alignat*}
Der folgende Ausdruck kann durch Ersetzung der Struktur der linken Seite einer Vereinfachungsregel durch die Struktur der rechten Seite vereinfacht werden. Weiter werden Ausdrücke ohne Unbekannte ausgewertet.
\begin{equation*}
    \begin{split}
	3 \cdot (\sin(a + b))^2 + 3 \cdot \sqrt{(\cos(a + b))^4}
	&\stapel = {(1)} 3 \cdot \paren*{(\sin(a + b))^2 + \sqrt{(\cos(a + b))^4}} \\
	&\stapel = {(2)} 3 \cdot \paren*{(\sin(a + b))^2 + \paren*{{(\cos(a + b))^4}}^{\frac 1 2}}\\
	&\stapel = {(3)} 3 \cdot \paren*{(\sin(a + b))^2 + {(\cos(a + b))^{4 \cdot \frac 1 2}}}\\
	& =              3 \cdot \paren*{(\sin(a + b))^2 + {(\cos(a + b))^2}}\\
	&\stapel = {(4)} 3 \cdot 1\\
    & = 3
    \end{split}
\end{equation*}
In der einzelnen Ersetzung wird eine Variable der Ersetzungsregel dabei an einen Teil des zu ersetzenden Ausdrucks gebunden, welcher dann im Einsetzen der rechten Regelseite im transformierten Ausdruck wieder an Stelle der Variablen gesetzt wird.
 Formalisiert wird dieser Unterschied in Abschnitt \ref{subsecMuster}. Als Beispiel gilt für die erste Umformung $y = (\sin(a + b))^2$ mit $y$ aus Regel $(1)$.
\end{beispiel}

~\\
Der Einfluss von Termersetzungssystemen geht weit über die Anwendung durch Mathematiker oder Ingenieure mit direktem Interesse am vereinfachten Ausdruck hinaus. Interessant ist etwa die Formulierung von Ersetzungsregeln im Kontext eines optimierenden Compilers \cite{HaskellCustomRewriteRules, HaskellCoreOptimizer}. Termersetzungssysteme können weiter direkt die Grundlage der Auswertung eines funktionalen Programms sein \cite{Jones1987JanRewritingMiranda}.


\section{Zielsetzung}
Ziel der Arbeit ist Design und Umsetzung eines Termersetzungssystems zur Vereinfachung algebraischer Ausdrücke. Der Kern des Termersetzungssystems ist ein Algorithmus zur Erkennung eines bestimmten Musters in einem Term. 
Der Algorithmus soll ein Muster dabei möglichst nicht nur erkennen, wenn der Term die exakt identische Struktur aufweist. Bestimmte Äquivalenzklassen, wie etwa alle Permutationen der Parameter in einer kommutativen Funktion, sollen bereits auf der Mustererkennungsebene berücksichtigt werden. Die Formulierung einer Menge von Ersetzungsregeln für das Termersetzungssystem muss also möglichst kompakt möglich sein. 
Damit ist das Ziel, die Mustererkennung möglichst schnell durchführen zu können, beim Treffen von Designentscheidungen in der Musterstruktur nicht ausschlaggebend, eine polynomielle Laufzeitkomplexität ist allerdings angestrebt.\\
Das Leistungsvermögen des entwickelten Termersetzungssystems wird in einer Anwendung zur Vereinfachung algebraischer Ausdrücke über den Komplexen Zahlen getestet. 

\section{Aufbau der Arbeit}
In Abschnitt \ref{secGrundlegendeDefinitionen} werden die Begriffe eingeführt, die zur Beschreibung eines zu transformierenden Ausdrucks, aber auch zur Beschreibung der Transformation selbst notwendig sind. Mit den dann etablierten Begriffen werden die Algorithmen zur Normalisierung ohne Mustererkennung aus Kapitel \ref{secErsteNormalform} und die Algorithmen zur Mustererkennung in Abschnitt \ref{secPattermatching} erläutert. 
Die tatsächliche Umsetzung und ihre Abweichungen von vorhergehenden Ideen ist in Kapitel \ref{secKernUmsetzungInCpp} behandelt. 
Kapitel \ref{secZusammenfassung} fasst die Arbeit zusammen.







\chapter{Grundlegende Definitionen} \label{secGrundlegendeDefinitionen}

\section{Term} \label{subsecTerm}
Eine Menge von Termen $T$ ist in dieser Arbeit immer  in Abhängigkeit der nicht-leeren Mengen $F$ und $C$, sowie der \emph{Stelligkeitsfunktion} $\Const{arity} \colon F \rightarrow \mathbb{N} \cup \{\omega\}$ definiert, ähnlich der Notation von Benanav et. al. in \cite{NPHardMatching}. $F$ enthält die sogenannten \emph{Funktionssymbole}. Beispiele für mögliche Elemente in $F$ sind \texttt{sin} und \texttt{sqrt}, zudem auch Operatoren wie die Division, etwa geschrieben als \texttt{divide}. Die Stelligkeitsfunktion $\Const{arity}$ gibt für jedes Funktionssymbol an, wie viele Argumente erwartet werden. Eine mögliche Stelligkeit der genannten Beispielsymbole ist die folgende.

$$\Const{arity} f = \begin{cases}
2 & f  = \texttt{divide}\\
1 & f \in \{\texttt{sin}, \texttt{sqrt}\}\\
\end{cases}$$

Kann eine Funktionssymbol $f$ beliebig viele Argumente entgegennehmen, wird gesagt, dass $f$ \emph{variadische} Stelligkeit hat oder \emph{variadisch} ist. Die Stelligkeitsfunktion bildet $f$ dann auf $\omega$ ab. 

Die Menge $C$ enthält die \emph{Konstantensymbole}. Mit den genannten Beispielen für Funktionssymbole, ergibt etwa $C = \mathbb R$ Sinn. Wichtig ist allerdings, dass im folgenden nicht vorausgesetzt wird, dass zwangsweise jedem Konstantensymbol ein eindeutiger numerischer Wert zugeordnet werden kann\footnote{Die Symbole unbekannten Wertes werden häufig von den Konstantensymbolen getrennt und Variablensymbole genannt. Diese Unterscheidung wird hier nicht getroffen, primär um die Definitionen einfach zu halten.}.




\begin{definition}\label{defTerm}~\\
Ein \Gls{Term} $t \in T(F, C)$ ist dann  {
\begin{itemize}
	\item{ein \Gls{Konstantensymbol}, also $t \in C$}
	\item{oder eine \emph{\Gls{Funktionsanwendung}} des Funktionssymbols $f \in F$ mit $\Const{arity}~f \in \{n, \omega\}$ 
		auf die Terme ${\tOneN \in T(F, C)}$, geschrieben ${t = (f, \tOneN)}$}
\end{itemize}}
In Mengenschreibweise:
$$T(F, C) \coloneqq C \cup \curl*{
(f, \tOneN)~|
~f\in F,~\Const{arity}~f~ \in \{n, \omega\},~ \tOneN \in T(F, C)
}$$ 
\end{definition}

Eine \Gls{Funktionsanwendung} wird in der Literatur oft mit dem Funktionssymbol außerhalb des Tupels geschrieben (\cite{buch1977}, \cite{NPHardMatching}), also $f(\tOneN)$ statt $(f, \tOneN)$. Zum deutlicheren Abheben von Funktionen die auf Termen operieren zu Termen selbst, wird diese Schreibweise hier keine Verwendung finden. 


\begin{figure}
\Tree [.\texttt{divide} 3 [.\texttt{sin} 1 ] ]
\label{ersterBeispielBaum}
\caption{Baumdarstellung des Terms $(\texttt{divide}, 3, (\texttt{sin}, 1))$ }
\end{figure}

\begin{beispiel}~\\
Als Beispiel lässt sich der Ausdruck $\frac 3 {\sin 1}$ in der formalen Schreibweise als Term mittels der Funktionssymbole $\texttt{sin}$ und $\texttt{divide}$, sowie den Konstantensymbolen $3$ und $1$ darstellen als $(\texttt{divide}, 3, (\texttt{sin}, 1))$. Ein Term kann dabei auch immer als Baum\footnote{In der theoretischen Informatik auch Syntaxbaum oder AST (englisch für \textit{Abstract Syntax Tree})} aufgefasst werden, etwa das aktuelle Beispiel in in Abb. \ref{ersterBeispielBaum} .
\end{beispiel}

Mit dem Kontext der Baumdarstellung lassen sich nun die folgenden Begriffe auf Terme übertragen. In der Funktionsanwendung $t = (f, \tOneN)$ sind $\tOneN$ die \emph{Kinder} ihres \emph{Vaters} $t$. Kinder sind allgemeiner \emph{Nachkommen}. Nachkommen verhalten sich transitiv, also ein Nachkomme $z$ des Nachkommen $y$ von $x$ ist auch ein Nachkomme von $x$. Umgekehrt ist $x$ \emph{Ahne} von $y$ und $z$. \\




\section{Funktionsauswertung}
Die Erweiterung des Funktionssymbols zur Funktion, die von einem Raum $Y^n$ nach $Y$ abbildet, folgt mittels der $\Const{eval}$ Funktion frei nach O'Donnell \cite{buch1977}.

\begin{equation*}
    \begin{split}
	\Const{eval} &\colon \paren*{F \rightarrow \bigcup_{n \in \mathbb{N}} Y^n \rightarrow Y} \times (C \rightarrow Y) \rightarrow T \rightarrow Y\\
	\Const{eval} &(u, v)~t = \begin{cases}
		u~f~(\elems {\Const{eval}(u, v)~t} 1 n) & t = (f, \tOneN)\\
		v~t                                      & t \in C\\
		\end{cases}
    \end{split}
\end{equation*}
Gilt $\Const{arity}~f = n \in N$ für ein $f \in F$, ist zudem die Funktion $u~f \colon Y^n \rightarrow Y$ in ihrer Definitionsmenge auf Dimension $n$ eingeschränkt. 
Die Funktion $u$ wird als \emph{\Gls{Interpretation}} der Funktionssymbole $F$, die Funktion $v$ als Interpretation der Konstantensymbole $C$ und die Funktion $\Const{eval}(u, v) \colon T \rightarrow Y$ als Interpretation der Terme $T$ bezeichnet. 
Ein Element des Bildes einer Interpretation, bzw. die Abbildung dahin, wird als \emph{\Gls{Auswertung}} bezeichnet.
\\~\\

\begin{beispiel} \label{bEval}
Sei $F = \{\texttt{sum}, \texttt{prod}, \texttt{neg} \}$ und $C = \mathbb{N}$ mit $\Const{arity}~ \texttt{sum} = \Const{arity}~ \texttt{prod} = \omega$ und $\Const{arity}~ \texttt{neg} = 1$.
Für die Interpretation $\Const{eval}(u, v)$ können $u$ und $v$ so gewählt werden, dass jeder Term in $T$ zu einer ganzen Zahl $n \in \mathbb{Z}$ auswertbar ist.

\begin{equation*}
    \begin{split}
    u~\texttt{sum}  ~(\elems y 1 n) &= \Sigma_{k = 1}^n y_k\\
    u~\texttt{prod} ~(\elems y 1 n) &=    \Pi_{k = 1}^n y_k\\
    u~\texttt{neg}~y &= -y\\
    &\\
    v~y &= y
    \end{split}
\end{equation*}

Hervorzuheben ist dabei, dass $u~\texttt{neg} \colon \mathbb Z \rightarrow \mathbb Z$ nur eine ganze Zahl als Parameter erwartet, während $u~\texttt{sum}$ und $u~\texttt{prod}$ Tupel ganzer Zahlen beliebiger Länge abbilden können.
Der Term $t = (\texttt{sum}, 3, (\texttt{prod}, 2, 4), (\texttt{neg}, 1))$ kann dann ausgewertet werden zu 
\begin{equation*}
    \begin{split}
    \Const{eval}(u, v)~t &= \Const{eval}(u, v) (\texttt{sum}, 3, (\texttt{prod}, 2, 4), (\texttt{neg}, 1)) \\
    &= u~\texttt{sum}~(\Const{eval}(u, v)~3, \Const{eval}(u, v)(\texttt{prod}, 2, 4),  \Const{eval}(u, v) (\texttt{neg}, 1)) \\
    &= u~\texttt{sum}~(v~3, u~\texttt{prod}~(\Const{eval}(u, v)~2, \Const{eval}(u, v)~4), u~\texttt{neg}~ (v~1)) \\
    &= u~\texttt{sum}~(3, u~\texttt{prod}~(v~2, v~4), u~\texttt{neg}~ 1) \\
    &= u~\texttt{sum}~(3, u~\texttt{prod}~(2, 4), -1) \\
    &= u~\texttt{sum}~( 3, 8, -1) \\
    &= 10 \\
    \end{split}
\end{equation*}
\end{beispiel}


\subsection{Konstruktor}
Eine direkt aus der Struktur des Terms folgende Interpretation $u_c$ für Funktionssymbole ist die des \emph{Konstruktors}. Als Konstruktor eines Typen $A$ wird allgemein eine Funktion bezeichnet, die nach $A$ abbildet. Für typlose Terme ist diese Definition deshalb schwierig. Insbesondere in funktionalen Sprachen wird das Konzept allerdings noch verfeinert. In Haskell transformiert ein Konstruktor Daten nicht im eigentlichen Sinne, sondern bündelt lediglich ein Tag, das festhält welcher Konstruktor genutzt wurde, mit den $n$ übergebenen Argumenten zu einem Tupel der Größe $n+1$ \cite{haskellConstructor}. Das entspricht exakt der Funktionsanwendung in einem Term, mit dem Funktionssymbol als Tag. Dementsprechend folgt die Definition des Konstruktors.

\begin{definition}~\\
Die Interpretation $u_c$ angewendet auf ein Funktionssymbol $f$ bildet das $n$ Tupel von Argumenten auf eine Funktionsanwendung von $f$ mit den selben $n$ Argumenten ab. 
 Mit $f \in F$ und $\Const{arity} f = n \in \mathbb N$ 
gilt 
$$u_c~f \colon T^n \rightarrow T, ~(\tOneN) \mapsto (f, \tOneN)$$
\end{definition}

Mit einem beliebigen $v \colon C \rightarrow C'$ ändert die Interpretation $\Const{eval}(u_c, v) \colon T(F, C) \rightarrow T(F, C')$ damit nur die Konstantensymbole eines Terms, lässt aber die sonstige Struktur unverändert. Insbesondere ist $\Const{eval}(u_c, v) \colon T \rightarrow T$ mit $v~y = y$ die Identität.
Die Interpretation $u_c$ reicht für bestimmte Funktionssymbole aus, etwa kann so das Funktionssymbol $\texttt{pair}$ ein Paar als Term darstellen.
$$u_c~\texttt{pair} \colon T^2 \rightarrow T, ~(a, b) \mapsto (\texttt{pair}, a, b)$$
Äquivalent ist die Darstellung endlicher Mengen und Tupel mit variadischen Funktionssymbolen \texttt{set} und \texttt{tup} möglich\footnote{Da eine Menge ihren Elementen keine Reihenfolge gibt, muss $u_c~\texttt{set}$ im Unterschied zu $u_c~\texttt{tup}$ prinzipiell nicht die ursprüngliche Argumentreihenfolge erhalten. In Kapitel \ref{subsecNormalSortieren} wird eine Größenrelation zur möglichen Umordung diskutiert.}. 
Hoffmann und O'Donnell \cite{hoffmann1982programming} definieren den Konstruktor im selben Kontext.


\begin{beispiel}~\\
Ein Graph $G = (V, E)$ wird als Paar der Menge von Knoten $V$ und Menge der Kanten $E$ definiert. Eine Kante ist dabei eine zweielementige Menge von Knoten. Der vollständige Graph auf drei Knoten ist $K_3 = \paren*{\{1, 2, 3\}, \curl*{\{1, 2\}, \{2, 3\}, \{1, 3\}}}$. Mit der Interpretation der Funktionssymbole $\texttt{pair}$ und $\texttt{set}$ als Konstruktoren lässt sich $K_3$ auch als Term darstellen:
$$(\texttt{pair}, (\texttt{set}, 1, 2, 3), (\texttt{set}, (\texttt{set}, 1, 2), (\texttt{set}, 2, 3), (\texttt{set}, 1, 3))).$$
\end{beispiel}

\section{Muster} \label{subsecMuster}

Bisher wurden die Objekte beschrieben, die in dieser Arbeit transformiert werden sollen. Die Transformationsregeln selbst lassen sich allerdings auch als Paare von bestimmten Termen darstellen. Zur Abgrenzung beider Konzepte werden die zu transformierenden Terme $t\in T(F, C)$ von hier an \emph{Literal} genannt, Terme die  Teil einer Regeldefinition sind werden \emph{Muster} genannt. Die Menge der Muster $M(F, C)$ ist dabei eine Obermenge der Literale, da sie deren Konstantensymbole um die Menge der \emph{Mustervariablen} $X$ erweitert\footnote{Die Ergänzung der Funktionssymbole um Mustervariablen ist genau so möglich, wird aber vor allem um die Notation verdaubar zu halten in den folgenden Kapiteln außen vor gelassen.}. Konkrete Elemente $\mathbf x \in X$ werden im folgenden \textbf{fett} geschrieben.
$$M(F, C) \coloneqq T(F, C \cup X)$$

Eine \emph{Ersetzungsregel} für Literale $t \in T(F, C)$ hat die Form $(l, r) \in M(F, C) \times M(F, C)$. Die linke Seite $l$ steht für das Muster, dass im Literal durch einen Ausdruck der Form der rechten Seite $r$ ersetzt werden soll. Für die bessere Lesbarkeit wird statt $(l, r)$ auch $l \mapsto r$ geschrieben.

\begin{beispiel} \label{bMuster}
Die Regel, die die Summe zweier identischer Terme $x$ als Produkt von $2$ und $x$ transformiert wird geschrieben als
$$(\texttt{sum}, \mathbf x, \mathbf x) \mapsto (\texttt{prod}, 2, \mathbf x)$$
Wird die Regel jetzt auf das Literal 
$t = (\texttt{sum}, (\texttt{sin}, 3), (\texttt{sin}, 3))$ angewandt, kann man $t$ zu $t' = (\texttt{prod}, 2, (\texttt{sin}, 3))$ transformieren. 
Hervorzuheben ist dabei, dass die Mustervariable $\mathbf x$ selbst nicht mehr im Ergebnisterm vorkommt. Sie wurde stattdessen durch den Teilterm ersetzt, der im Ursprungsliteral an der Stelle von $\mathbf x$ stand, nämlich $(\texttt{sin}, 3)$.
\end{beispiel}

\begin{definition} \label{defMatch}
Für ein Paar $(p, t) \in M(F, C) \times T(F, C)$ ist eine Funktion $v_p \colon X \rightarrow T(F, C)$ ein \emph{Match}, wenn folgendes gilt:
$$\Const{eval}(u_c, \tilde v_p)~ p = t$$
$$\tilde v_p~ c = \begin{cases}
	v_p~ c & c \in X\\
	c      & c \in C \setminus X
\end{cases}$$
$v_p$ muss die Mustervariablen in $p$ so durch Literale ersetzen, dass ein Term identisch zu $t$ entsteht. 
Im vorangegangenen Beispiel \ref{bMuster} gilt damit $v_p~ \mathbf a = (\texttt{sin}, 3)$.

Im Folgenden wird der Begriff des Matches noch etwas weiter gefasst. Es werden nach wie vor die Mustervariablen durch Literale ersetzt, allerdings muss nicht direkt das Ergebnis der Ersetzung, sondern nur eine normalisierte Form des Ergebnisses mit dem Literal $t$ übereinstimmen:
$$\Const{lit}(v_p, p)  = t$$
mit 
$$\Const{lit}(v_p, p) \coloneqq \Const{normalize}~(\Const{eval}(u_c, \tilde v_p)~ p).$$
Die Funktion $\Const{lit} \colon \paren*{X \rightarrow T} \times M \rightarrow T$ wird als \emph{Musterinterpretation} bezeichnet.
Aus der Definition ist bereits klar, dass $\Const{normalize} \colon T \rightarrow T$ Terme auf Terme abbildet und ein Match $v_p$ nur dann gefunden werden kann, wenn $t$ im Bild von $\Const{normalize}$ liegt. 
Gedacht ist $\Const{normalize}$ als Mittel, um Unterschiede zwischen Termen zu reduzieren, die die Auswertung für eine gegebene Interpretationen $u$ und $v$ nicht ändern. Da das Ergebnis von $\Const{normalize}$ aber nur von dem übergebenen Term abhängig ist, können nie direkt Unterschiede zwischen mehreren Termen verglichen und beseitigt werden. Ist die normalisierte Form $t'$ eines Terms $t$ ein Fixpunkt von $\Const{normalize}$, ist in dem Kontext klar, dass $t$ und $t'$ die gleiche Auswertung mit der Interpretation $(u, v)$ besitzen, da auch $\Const{normalize}~t' = \Const{normalize}~t$ gilt. Wäre $\Const{normalize}~t' \neq t'$, würde die Funktion nicht die ihr zugedachte Aufgabe erfüllen. Deshalb muss $\Const{normalize}$ eine Projektion sein, also eine Funktion, für die jeder Bildpunkt gleichzeitig ein Fixpunkt ist. 

Welche Unterschiede $\Const{normalize}$ beseitigt soll hier nicht festgelegt werden. Klar ist aber, dass je nach Wahl von $\Const{normalize}$ zwar ein einzelnes Muster sehr mächtig werden kann, also ein Match mit sehr vielen Termen möglich ist, das Finden des Matches dann im allgemeinen Fall allerdings immer aufwändiger wird. 

\end{definition}

Ein \emph{Matchalgorithmus} ist eine Vorgehensweise für ein gegebenes Paar $(p, t) \in M(F, C) \times T(F, C)$ ein gültiges Match zu finden. Perfekt wird ein solcher Algorithmus dann genannt, wenn jedes mögliche Match gefunden werden kann.







\chapter {Normalform} \label{secErsteNormalform}

Das Kernthema dieser Arbeit ist die Vereinfachung von Termen. Eine Vereinfachung ist nur gültig, wenn sich die Bedeutung des vereinfachten Terms gegenüber der des ursprünglichen Terms nicht geändert hat. Da ein Term in sich keine Bedeutung trägt, muss eine Vereinfachung immer in Bezug auf eine Interpretation $\Const{eval}(u, v)$ gesehen werden. Etwa kann der Ausdruck $X A X^{-1}$ zu $A$ vereinfacht werden, wenn $X, A \in \mathbb{C} \setminus \{0\}$, allerdings ist die Vereinfachung allgemein nicht möglich, sollten die Symbole $X$ und $A$ für Matrizen stehen. \\
Im Folgenden wird von der Assoziativität oder Kommutativität bestimmter Funktionssymbole gesprochen. Diese ist immer im Kontext der Interpretation $\Const{eval}(u, v)$ zu sehen. Gleichzeitig ist aber auch klar, dass unabhängig von der Interpretation verschiedene Funktionssymbole die Rolle der Multiplikation übernehmen müssen, sollte sowohl skalare Multiplikation als auch Matrixmultiplikation im selben Term möglich sein. $X A X^{-1}$ als Matrixmultiplikation könnte der Term $(\texttt{prod'}, X, A, (\texttt{pow}, X, -1))$ darstellen. Sind $A$ und $X$ Skalare, wäre der Ausdruck als $(\texttt{prod}, X, A, (\texttt{pow}, X, -1))$ schreibbar. Das Funktionssymbol $\texttt{prod'}$ steht dann für ein nicht-kommutatives Produkt, während die Reihenfolge der Parameter in einer Funktionsanwendung von $\texttt{prod}$ keine Rolle spielt.\\

In diesem Abschnitt werden einfache Termumformungen beschrieben, die isolierte Eingenschaften einzelner Funktionen ausnutzen. Ziel ist es, Äquivalenzklassen für die Erkennung von Mustern zu schaffen, die über die Austauschbarkeit jeder Mustervariable mit einem beliebigen Literal hinausgehen. Als Beispiel dient die Regel der Faktorisierung, normal geschrieben als $x \cdot y + x \cdot z = x \cdot (y + z)$. In der in Unterkapitel \ref{subsecMuster} etablierten Musterschreibweise, mit Mustervariablen \textbf{fett} geschrieben, wird daraus:
$$(\texttt{sum}, (\texttt{prod}, \mathbf x, \mathbf y), (\texttt{prod}, \mathbf x, \mathbf z)) \mapsto (\texttt{prod}, \mathbf x, (\texttt{sum}, \mathbf y, \mathbf z))$$
Ziel ist, die Regel auf das Literal $(\texttt{sum}, (\texttt{prod}, a, b), (\texttt{prod}, a, c, d))$ anwendbar zu machen bzw. eine Regel formulieren zu können, die eine ähnliche Struktur hat und diesen Fall mit abdeckt. 
Würde das Literal geschrieben sein als $$(\texttt{sum}, (\texttt{prod}, a, b), (\texttt{prod}, a, (\texttt{prod}, c, d))),$$ gäbe es ein Match $v_p$ der linken Regelseite mit dem Literal, mit $v_p~\mathbf x = a$, $v_p~\mathbf y = b$ und $v_p~\mathbf z = (\texttt{prod}, c, d)$. Ergebnis dieses Kapitels wird eine in Abschnitt \ref{subsecMuster} genutzte Projektion ${\Const{normalize} \colon T \rightarrow T}$ sein, welche die Beispielregel auf das Beispielliteral in seiner ursprünglichen Form anwendbar macht.\\

Weiteres Ziel dieses Kapitels ist, dass möglichst viele Literale mit identischer Auswertung identische normalisierter Terme besitzen. So soll etwa die Normalisierung von $t_1 = (\texttt{sum}, a, b, c)$ identisch zur Normalisierung von $t_2 = (\texttt{sum}, b, a, c)$ identisch zur Normalisierung von $t_3 = (\texttt{sum}, (\texttt{sum}, b, a), c)$ sein. Je mehr Literale identischen Wertes auch zu identischen Termen normalisiert werden, desto besser können Muster erkannt werden, in denen dieselbe Mustervariable mehrfach vorkommt. Interessant ist dabei, dass derselbe Effekt auch erreicht werden würde, wenn man eine Menge von Ersetzungsregeln um entsprechende normalisierende Ersetzungsregeln ergänzt. Wo die Grenze in der Arbeitsteilung von einer fest implementierten $\Const{normalize}$ Funktion zu den Regeln in einem Termersetzungssystem liegt, ist prinzipiell fast beliebig und in erster Linie eine Frage des Aufwandes, sowohl in Programmierung als auch Laufzeit. Etwa würde eine Darstellung natürlicher Zahlen ähnlich der Church-Numerale, wie sie in der Fachliteratur, etwa bei Baader und Nipkow \cite{baader_nipkow_1998}, üblich ist, erlauben, Rechenoperationen auf den natürlichen Zahlen komplett mit einer endlichen Menge von Mustern auszuwerten. Nachteile dieser Vorgehensweise wären allerdings eine langsamere Auswertung, mehr Speicherbedarf und nur sehr schwierig zu lesende Ergebnisse. Andersherum wäre etwa die Anwendung der ersten binomischen Formel prinzipiell auch in der $\Const{normalize}$ Funktion möglich, allerdings steht der Aufwand, manuell auf das Muster zu testen, nur möglicherweise minimalen Geschwindigkeitsvorteilen des Gesamtsystems gegenüber. Die Transformationen, die in diesem Kapitel der $\Const{normalize}$ Funktion zugewiesen werden, sollen also idealerweise nicht einfacher mit Mustern implementierbar sein.\\

In diesem Kapitel werden häufig Abschnitte der Parameter einer Funktionsanwendung beliebiger Länge der Form $\elems t i k$ vorkommen. Kompakt wird $ts...$ für den (möglicher\-weise leeren) Abschnitt des Funktionsanwendungstupels geschrieben. Das $s$ in $ts...$ ist dann nicht als einzelnes Symbol zu lesen, sondern als Suffix um $t$ in den Plural zu setzen. \\$(f, \elems t 1 k, a, \elems t {k+2} n)$ kann also äquivalent $(f, ts..., a, rs...)$ geschrieben werden, mit $(\elems t 1 k) = (ts...)$ und $(\elems t {k+2} n) = (rs...)$.\\

\section {Assoziative Funktionsanwendungen}
Die geschachtelte Anwendung einer assoziativen Funktion führt je nach Klammersetzung zu verschiedenen mathematisch äquivalenten Termen. Als Beispiel dient hier die Addition, dargestellt als Anwendung des Funktionssymbols $\texttt{sum}$. Die folgenden Ausdrücke sind paarweise verschiedene Terme, jedoch in ihrer Interpretation als Summe von $a$, $b$, $c$ und $d$ alle mathematisch äquivalent.
\begin{equation*}
	\begin{split}
	   (\texttt{sum}, (\texttt{sum}, (\texttt{sum}, a, b), c), d) 
    &= (\texttt{sum}, (\texttt{sum}, a, (\texttt{sum}, b, c)), d)\\
	&= (\texttt{sum}, (\texttt{sum}, a, b), (\texttt{sum}, c, d))\\
	&= (\texttt{sum}, a, (\texttt{sum}, b, (\texttt{sum}, c, d)))\\
	&= \dots \\
	\end{split}
\end{equation*}
Es gibt mehrere Optionen eine solche Schachtelung in einem Term zu normalisieren, also in eine eindeutige Form zu bringen. Die erste ist, festzulegen, dass in der normalisierten Form höchstens eines der beiden Argumente einer binären assoziativen Funktion wieder Anwendung desselben Funktionssymbols sein darf. Wählt man das zweite Argument dafür aus, wird die Summe in der Normalform dargestellt als $(\texttt{sum}, a, (\texttt{sum}, b, (\texttt{sum}, c, d)))$. Ein Problem der Methode ist, dass nicht immer alle Argumente eines assoziativen Funktionssymbols direkt vorliegen.\\
Alternativ kann man die Summe von zwei Argumenten auch als Spezialfall einer Summe von $n \in \mathbb{N}$ Argumenten auffassen, gewohnt geschrieben als $\Sigma_{x \in \{a, b, c, d\}} x$. Dieser Weg wird im Folgenden gewählt, wobei die Darstellung als Term dann $(\texttt{sum}, a, b, c, d)$ ist. Assoziative Funktionen sind in der gewählten Darstellung damit variadisch. Eker (\cite{BipartiteGraphMatching}), Benanav (\cite{NPHardMatching}) und Kounalis (\cite{ACPatternCompiler}) wählen ebenfalls diese Darstellung. 
Die Normalisierung von Funktionsanwendungen des assoziativen Funktionssymbols $f$ bedeutet dann, geschachtelte Funktionsanwendungen in eine einzelne Funktionsanwendung zu übersetzen. 
$$(f, as..., f(bs...), cs...) \mapsto (f, as..., bs..., cs...)$$
Der Spezialfall ist eine assoziative Funktionsanwendung mit nur einem Parameter. Diese kann immer zu dem Parameter selbst normalisiert werden. 

Als Algorithmus dargestellt sind die Überlegungen in Algorithmus \ref{flatten}.
Die Funktion $\tilde u$ kann hier und in den weiteren Algorithmen dieses Kapitels als \glqq{natürliche}\grqq{} Interpretation der Menge von Funktionssymbolen gesehen werden, ähnlich $u$ in Beispiel \ref{bEval}. Prinzipiell ist für die Gültigkeit des Kapitel aber egal, in welchem Kontext die Abbildungsvorschrift von $\tilde u$ Sinn ergibt oder ob ein solcher Kontext überhaupt existiert. Wichtig ist nur, dass $\tilde u$ über das gesamte Kapitel hinweg eine einheitliche Abbildungsvorschrift hat.

\begin{algorithm}
\DontPrintSemicolon
\caption{$\Const{flatten} \colon T \rightarrow T$}\label{flatten}
\KwIn{$t \in T(F, C)$}

\If{$t = (f, t_1)$ mit $\tilde u~f$ assoziativ}{
    \Return {$t_1$}
}
\ElseIf{$t = (f, \elems t 0 {n-1})$ mit $\tilde u~f$ assoziativ}{
    \While{$t = (f, xs..., (f, ys...), zs...)$}{
        $t \leftarrow (f, xs..., ys..., zs...)$\;
    }
}
\Return {$t$}
\end{algorithm}

\section{Kommutative Funktionsanwendungen} \label{subsecNormalSortieren}
Eine Normalform für kommutative Funktionsanwendungen erfordert eine totale Ordnung auf der Menge aller Terme $T(F, C)$. 

\begin{definition} \label{defOrdnungKleiner}
Aufbauend auf einer totalen Ordnung von $F$ sowie $C$, kann eine lexikographische Ordnung $<$ von $T$ frei nach \cite{LexikografischeOrdnung} wie folgt definiert werden: 
\begin{enumerate}
	\item{sind $c, \tilde{c} \in T$ Konstantensymbole, so ist die Ordnung identisch zu der Ordnung in $C$}
	\item{sind $c, a, \in T$ sowie $c$ ein Konstantensymbol und $a$ eine Funktionsanwendung, gilt $c < a$ }
	\item{sind $a = (f, ts...), b = (g, rs...) \in T$ Funktionsanwendungen und ist $f < g$, gilt $a < b$}
	\item{sind $a = (f, t_1, \dots, t_n), b = (f, r_1, \dots, r_m) \in T$ Funktionsanwendungen, ist die Ordnung wie folgt}
	\begin{enumerate}
		\item{wenn $\exists k \leq \min{(n, m)} \colon \forall i < k ~ t_i = r_i ,~ t_k \neq r_k $, gilt ${a < b \iff t_k < r_k}$}
		\item{ist $n < m$ und $\forall i < n\colon t_i = r_i$, gilt $a < b$}
		\item{ist $n = m$ und $\forall i \leq n\colon t_i = r_i$, gilt $a = b$}
	\end{enumerate}
\end{enumerate}

\end{definition}

\begin{lemma} \label{lemMinMax}
$T$ hat genau dann ein Minimum, wenn $C$ ein Minimum hat $(1)$. $T$ hat kein endlich großes Maximum $(2)$.
\end{lemma}

\textbf{Beweis.}
Teil $(1)$ folgt daraus, dass die Konstantensymbole selbst bereits Terme kleiner als jede Funktionsanwendung sind. Ein Minimum von $C$ ist damit gleichzeitig Minimum von $T$.

Teil $(2)$ folgt aus einem Widerspruch. Angenommen es gäbe einen größten endlichen Term $t$. Der Term $t'$ ist identisch zu $t$, nur die Konstantensymbole von $t$ werden durch beliebige Funktionsanwendungen ersetzt. Der neue Term $t'$ ist größer als $t$.
\hfill $\square$\\

Zur Normalisierung einer kommutativen Funktionsanwendung werden zuerst alle Parameter normalisiert, dann können die Parameter nach der lexikographischen Ordnung $<$ von $T$ sortiert werden. 

\section{Teilweise Auswertung} \label{subsecNormalKombinieren}

\begin{algorithm}
\DontPrintSemicolon
\caption{$\Const{combine} \colon T \rightarrow T$}\label{combine}
\KwIn{$t \in T(F, C)$}

\If{$\Const{eval}(\tilde u, \Const{id})~t = c \in C$}{
    \Return {$c$}
}
\ElseIf{$t = (f, \elems t 0 {n-1})$ und $\tilde u~f$ assoziativ}{
    \If{$\tilde u~f$ kommutativ}{
        \While{$t = (f, us..., x, vs..., y, ws...) $ und $ \tilde u~f~(x, y) = z \in C$}{
            $t \leftarrow (f, z, us..., vs..., ws...)$\;
        }
    }
    \Else{
        \While{$t = (f, us..., x, y, vs...) $ und $ \tilde u~f~(x, y) = z \in C$}{
            $t \leftarrow (f, us..., z, vs...)$\;
        }
    }
}
\end{algorithm}

Mit der Darstellung einer assoziativen Funktion mit einem variadischen Funktionssymbol $f \in F$, kann eine Funktionsanwendung von $f$ in bestimmen Fällen teilweise ausgewertet werden. Als Beispiel kann die Summe der Symbole $1$, $3$ und $\texttt{a}$ geschrieben als $(\texttt{sum}, 1, 3, \texttt{a})$ zur Summe $(\texttt{sum}, 4, \texttt{a})$ transformiert werden. 
Gilt allgemein für ein Funktionssymbol $f \in F$, dass $\tilde u~f$ assoziativ ist, reicht es aus, zwei aufeinander folgende Argumente $x$ und $y$ in einer Funktionsanwendung von $f$ zu finden, mit denen die Funktionsanwendung $(f, x, y)$ auswertbar wäre. $x$ und $y$ können dann entsprechend ersetzt werden.
$$\Const{eval}(\tilde u, \Const{id})~(f, x, y) = z \in C \implies (f, us..., x, y, vs...) \mapsto (f, us..., z, vs...)$$

Ist $\tilde u~f$ zudem kommutativ, müssen $x$ und $y$ nicht notwendigerweise direkt aufeinander folgen.
$$\Const{eval}(\tilde u, \Const{id})~(f, x, y) = z \in C \implies (f, us..., x, vs..., y, ws...) \mapsto (f, z, us..., vs..., ws...)$$

Eine normalisierte Funktionsanwendung enthält keine zwei auf diese Art ersetzbare Argumente $x$ und $y$ mehr. Weiter ist jede Funktionsanwendung, die als ganzes zu einer Konstante $z \in C$ auswertbar ist, ausgewertet.\\
Zusammengefasst sind die Überlegungen in Algorithmus \ref{combine}. Die Verfahren zur Normalisierung assoziativer und kommutativer Funktionssymbole werden auch von Eker \cite{BipartiteGraphMatching} beschrieben.

\section{Kombination der einzelnen Vereinfachungen} \label{subsecKomboNormal}

\begin{algorithm}
\DontPrintSemicolon
\caption{$\Const{normalize} \colon T \rightarrow T$}\label{normalize}
\KwIn{$t \in T(F, C)$}

\If {$t = (f, t_1, \dots, t_n)$}{
	\For {$i \in \{1, \dots, n\}$}{
		$t_i \leftarrow \Const{normalize}~t_i$\;
	}
}
$t \leftarrow \Const{flatten}~t$\;
$t \leftarrow \Const{combine}~t$\;
\If {$t = (f, t_1, \dots, t_n)$ mit $\tilde u~f$ kommutativ}{
	sortiere $t_1, \dots, t_n$ lexikographisch nach Ordnung $<$\;
}
\Return $t$ 
\end{algorithm}
Algorithmus \ref{normalize} kombiniert die einzelnen Überlegungen dieses Kapitels: Zunächst werden alle Argumente einer Funktionsanwendung normalisiert, anschließend die Funktionsanwendung selbst.








\chapter{Mustererkennung} \label{secPattermatching}

In Kapitel \ref{subsecMuster} wurden die Konzepte des Musters und des Matches eingeführt, zweiteres insbesondere in einer weiter gefassten Form, was auch erlaubt, Muster mit strukturell nicht exakt identischen Literalen zu assoziieren, sofern die Unterschiede mit der Projektion $\Const{normalize} \colon T \rightarrow T$ beseitigt werden können. Die in Kapitel \ref{secErsteNormalform} beschriebene Funktion $\Const{normalize}$ ist als solche Projektion nutzbar.  

In diesem Kapitel wird ein Algorithmus entwickelt, der die Äquivalenzklassen der verschieden geschachtelten Funktionsanwendungen eines assoziativen Funktionssymbols mit den selben Argumenten sowie die Äquivalenzklassen der verschieden permutierten Argumente in der Funktionsanwendung eines kommutativen Funktionssymbols beim Finden eines Matches berücksichtigt. Die teilweise Auswertung von $\Const{normalize}$ aus Kapitel \ref{subsecNormalKombinieren} wird in diesem Kapitel nicht verfolgt.


%.........................................................................
%.........................................................................
%.........................................................................
\section{Grundstruktur} \label{subsecPatternmatchingGrundstruktur}

Dem Ergebnis eines Matchalgorithmus müssen zwei Dinge entnehmbar sein. Zum ersten muss klar sein, ob ein Match $v_p \colon X \rightarrow T$ gefunden wurde. Wurde ein Match gefunden, muss zudem dessen Abbildungsvorschrift zurückgegeben werden. Der Rückgabetyp von Algorithmus \ref{simpleMatchAlgorithmShell} ist deswegen nicht nur das finale Match, sondern auch ein Wahrheitswert $b \in \mathit{Bool} \coloneqq \{\Const{false}, \Const{true}\}$. Alternativ kann die Menge aller möglichen Matches zurückgegeben werden. Diese Idee wird im Folgenden nicht weiter verfolgt, da sie mit den Anforderungen an hier behandelte Muster auch im besten Fall schnell exponentielle Laufzeiten produziert\footnote{siehe Lemma \ref{lemNrAssocMatches}}. Sind aber Mehrfachnennungen einer Mustervariable in einem Muster nicht erlaubt, haben Hoffman und O'Donnell in \cite{patternMatchingInTrees} gezeigt, dass sehr effiziente Algorithmen zum gleichzeitigen Finden von Matches einer ganzen Menge von Mustern in allen Teiltermen eines Literals mit dieser Grundidee möglich sind.\\

Da eine Mustervariable in dieser Arbeit mehrfach in einem Muster vorkommen darf, muss ein Algorithmus beim Suchen nach einem Match $v_p \colon X \rightarrow T$ zu jedem Zeitpunkt wissen, für welche $x \in X$ das Match $v_p~x$ bereits feststeht. $v_p$ ist also nicht nur Rückgabewert eines Matchalgorithmus, sondern muss mit den Funktionswerten für bereits besuchte Mustervariablen auch Eingabe in den Algorithmus sein. In Algorithmus \ref{simpleMatchAlgorithmShell} wird $v_p$ deswegen als partielle Funktion definiert, welche zu Beginn keine einzige Mustervariable nach $T$ abbilden kann. \\

\begin{algorithm}
\DontPrintSemicolon
\caption{$\Const{simpleMatchAlgorithmShell} \colon M \times T \rightarrow (\mathit{Bool}, X \rightharpoonup T)$}\label{simpleMatchAlgorithmShell}
\KwIn{$p \in M$, $t \in T$}

\textbf{let} $v_p \colon X \rightharpoonup T,~ x \mapsto \bot$\;
\Return {$\Const{simpleMatchAlgorithm}(p, t, v_p)$}
\end{algorithm}

\begin{algorithm}
\DontPrintSemicolon
\caption{$\Const{simpleMatchAlgorithm} \colon M \times T \times (X \rightharpoonup T) \rightarrow (\mathit{Bool}, X \rightharpoonup T)$}\label{simpleMatchAlgorithm}
\KwIn {$p \in M$, $t \in T$, $v_p \colon X \rightharpoonup T$}

\If {$p \in X$ \KwAnd $v_p~p = \bot$} {
	$(v_p~p) \leftarrow t$\;
	\Return {$(\Const{true}, v_p)$}
}
\ElseIf {$p \in X$ \KwAnd $v_p~p \neq \bot$}{
	\Return {$(v_p~p = t, v_p)$}
}
\ElseIf {$p \in C \setminus X$} {
	\Return {$(p = t, v_p)$}
}
\ElseIf {$p = (f, \elems p 0 {m-1})$ \KwAnd $t = (f, \elems t 0 {n-1})$}{
	\For {$k \in \{0, \dots, {n-1}\}$}{
		$(\mathit{success}_k, v_p) \leftarrow \Const{simpleMatchAlgorithm}(m_k, t_k, v_p)$\;
		\If {$\Const{not}~\mathit{success}_k$}{
			\Return {$(\Const{false}, v_p)$}
		}
	}
	\Return {$(\Const{true}, v_p)$}  
}
\Else {
	\Return {$(\Const{false}, v_p)$}  
}
\end{algorithm}


Wenn das Match streng definiert ist, also der Unterschied zwischen einem Muster $p$ und einem Literal $t$ für die Existenz eines Matches $v_p$ ausschließlich darin bestehen darf, dass Teilterme von $t$ in $p$ durch eine Mustervariable repräsentiert werden, ist ein einfacher Matchalgorithmus fast trivial. Auf der Idee von $\Const{simpleMatchAlgorithm}$ basieren auch die späteren Algorithmen dieses Kapitels. Diese ist, dass mit einer Tiefensuche, die parallel durch Muster und Literal läuft, nach einem Unterschied zwischen beiden gesucht wird. Mustervariablen funktionieren dabei als Wildcard, wenn eine identische Mustervariable in der Tiefensuche vorher noch nicht gefunden wurde. Andernfalls vergleichen sie identisch zu dem Teilbaum, der mit dem ersten Vorkommen der Mustervariable verglichen wurde. Die Aufgabe, diese vorher begegneten Teilbäume zu speichern, übernimmt $v_p$, was erklärt, warum $v_p$ auch als Parameter für Algorithmus \ref{simpleMatchAlgorithm} notwendig ist. Ist das gesamte Muster durchlaufen worden ohne einen strukturellen Unterschied zum Literal zu finden, ist $v_p$ das resultierende Match.\\

\begin{lemma}~\\
Die Laufzeit von Algorithmus \ref{simpleMatchAlgorithmShell} ist linear abhängig von der Anzahl der Funktionssymbole und Konstantensymbole des Literals.
\end{lemma}

\textbf{Beweis}.\\
Gibt es ein Match, wird jedes Funktionssymbol und Konstantensymbol des Literals höchstens ein Mal in $\Const{simpleMatchAlgorithm}$ abgelaufen. Wird eine Funktionsanwendung $t$ im Literal parallel zu einer Mustervariable $\mathbf x$ im Muster abgelaufen, bleiben die Nachkommen von $t$ unbesucht, wenn $\mathbf x$ noch nicht gematcht wurde. Andernfalls wird jeder Nachkomme von $t$ höchstens ein Mal abgelaufen, um Gleichheit zu $v_p~\mathbf x$ zu testen.
Gibt es kein Match, wird das Literal so lange identisch zum anderen Fall abgelaufen, bis ein struktureller Unterschied festgestellt wurde. Dann bricht der Algorithmus ab.
\hfill $\square$\\


\begin{definition}~\\
Die Instanz einer Mustervariable $\mathbf x$ wird als \emph{bindend}  bezeichnet, wenn sie in einer Tiefensuche durch das gesamte Muster als erste Instanz abgelaufen wird. Da Algorithmus \ref{simpleMatchAlgorithm} das Muster in einer Tiefensuche abläuft, ist die Bedingung $p \in X$ \KwAnd $v_p~p = \bot$ in der ersten Zeile damit genau dann wahr, wenn $p$ bindend ist\footnote{Der Begriff \emph{bindend} ist so zu verstehen, dass $\mathbf x$ nach Ablauf der ersten Instanz in Algorithmus \ref{simpleMatchAlgorithm} einen festen Wert $v_p~\mathbf x$ hat, also für den spätere Teile des Musters an diesen Wert gebunden ist.}. Weitere Instanzen von $\mathbf x$ im selben Muster werden als \emph{gebunden} bezeichnet.
\end{definition}


%.........................................................................
%.........................................................................
%.........................................................................
\section{Multi-Mustervariablen} \label{subsecMulti}

Von Anfang an werden Funktionssymbole in dieser Arbeit als möglicherweise variadisch definiert. Das ist insofern ein Problem, als dass Muster bisher immer nur eine feste Anzahl an Argumenten für jede Funktionsanwendung angeben können. Ist ein variadisches Funktionssymbol zudem assoziativ, ließe sich dieses Problem prinzipiell beheben, wenn Assoziativität im Matchalgorithmus berücksichtigt würde. Das Muster $\tilde p = (f, \mathbf x, \mathbf y)$ würde für ein assoziatives Funktionssymbol $f$ dann auch Literale wie $\tilde t = (f, a, b, c, d)$ matchen, mit verschiedenen Optionen für $v_p$, etwa $v_p~\mathbf x = (f, a, b)$ und $v_p~\mathbf y = (f, c, d)$. Ist auch die leere Funktionsanwendung $(f)$ von $f$ erlaubt \footnote{Das ergibt dann Sinn, wenn $f$ ein neutrales Element $e \in T$ besitzt, da $(f, as..., (f), bs...)$ mit $\Const{normalize}$ zu $(f, as..., bs...) = (f, as..., e, bs...)$ umgeformt wird. Im Folgenden wird von der Existenz eines Neutralen Elementes ausgegangen.}, gäbe es fünf verschiedene Matches $v_p$ für $\paren*{\tilde p, \tilde t}$ mit nicht-kommutativem $f$.

\begin{lemma}~\\ \label{lemNrAssocMatches}
Das Muster $p = (f, \elems {\mathbf x} 1 m)$ hat mit dem Literal $t = (f, \elems a 1 n)$ genau ${m + n - 1}\choose n$ mögliche Matches, wenn $f$ assoziativ aber nicht kommutativ ist $(1)$. Ist $f$ assoziativ und kommutativ, gibt es $m^n$ mögliche Matches $(2)$.\\
\end{lemma}

\textbf{Beweis}.\\
$(1)$: Es gibt $m$ möglicherweise leere Abschnitte in den $n$ Argumenten von $t$, welche jeweils eine Mustervariable $\mathbf x_i$ matchen. Stellt man eine Abschnittsgrenze mit einem Strich $~|~$ und ein Argument von $t$ mit einem Stern $~*~$ dar, kann die Aufteilung der Argumente von $t$ über ein String aus $m - 1$ Strichen und $n$ Sternen dargestellt werden. 
Als Beispiel ist $~**|**~$ der String zur Aufteilung von $\tilde t$ aus dem Anfang des Abschnittes zum beschriebenen Match $v_p$.
Es gibt ${m + n - 1}\choose n$ Möglichkeiten die $n$ Sterne auf die ${m + n - 1}$ möglichen Plätze zu verteilen.\\

$(2)$: Jedes der $n$ Argumente von $t$ kann unabhängig der restlichen Argumenten zu einer der $m$ Mustervariablen gematcht werden. Insgesamt ergeben sich so $m^n$ Kombinationen.
\hfill $\square$\\

Schon für nicht-kommutative aber assoziative Funktionssymbole $f$ gibt es somit Muster $p$ mit einer Anzahl möglicher Matches, die exponentiell mit der Größe des Literals steigt. Ist ein solches Muster $p$ Teil eines größeren Musters $p'$ und kommen Mustervariablen von $p$ auch in anderen Teilen von $p'$ vor, so ist nicht direkt ersichtlich, wie ein Algorithmus aussehen würde, der in $P$ liegt und bestimmen kann, dass es kein Match für $p'$ mit einem entsprechenden Literal gibt, bzw. das Match findet. Die Existenz eines solchen Algorithmus ist unwarscheinlich: Benanav hat 1987 gezeigt, dass das Problem NP-vollständig ist \cite{NPHardMatching}.
Von dem perfekten Matchalgorithmus wird aus diesem Grund abgesehen. Für viele Spezialfälle sind bessere Algorithmen möglich. Eine wichtige Klasse solcher Spezialfälle ist die, wo von vorne herein klar ist, welche Mustervariable möglicherweise mehrere Parameter des Literals matchen soll. Würde man etwa bei der Ersetzung der ersten Binomischen Formel eine weitere Mustervariable $\mathbf c$ hinzufügen, um die Binomische Formel auch in einer Summe mit mehr als drei Summanden zu erkennen, kann die Ersetzungsregel geschrieben werden als
$$(\texttt{sum}, (\texttt{pow}, \mathbf a, 2), (\texttt{prod}, 2, \mathbf a, \mathbf b), (\texttt{pow}, \mathbf b, 2), \mathbf c) \mapsto (\texttt{sum}, (\texttt{pow}, (\texttt{sum}, \mathbf a, \mathbf b), 2), \mathbf c).$$
$\mathbf c$ ist damit vom Autor des Musters ausschließlich dazu gedacht überbleibende Summanden \glqq aufzusaugen\grqq{}. Dieser Gedanke bleibt aber bisher dem Algorithmus verborgen.
Die in dieser Arbeit gewählte Lösung zur Beschreibung von beliebig vielen Argumenten in einem Muster ist im Prinzip schon in Kapitel \ref{secErsteNormalform} eingeführt worden. Die Schreibweise $(f, ts...)$ als kompakte Alternative zu $(f, t_1, \dots, t_n)$ hat viele der zur Beschreibung von Assoziativität gewünschten Eigenschaften. Ferner können so auch Muster mit nicht assoziativen variadischen Funktionssymbolen dargestellt werden. Eine \emph{\Gls{Multi-Mustervariable}} der Form $\mathbf{xs...}$ kann also nicht nur genau ein Argument in einer Funktionsanwendung matchen, sondern beliebig viele, auch keins. Um den Matchalgorithmus nicht zu kompliziert zu gestalten, darf jede Multi-Mustervariable auf der linken Seite einer Ersetzungsregel nur höchstens ein Mal vorkommen\footnote{Das macht zudem mehrere Multi-Mustervariablen in der selben Funktionsanwendung eines kommutativen Funktionssymbols auf der linken Seite einer Ersetzungsregel überflüssig. Diese Konstellation ist dementsprechend im Folgenden nicht berücksichtigt.}. Die rigorose Beschreibung des Konzeptes gestaltet sich allerdings mit den bisher eingeführten Ideen schwierig, da eine Multi-Mustervariable nur Teil einer Funktionsanwendung ist und damit auch alleine keinen vollständigen Term repräsentiert. Konnte eine Matchfunktion $v_p \colon X \rightarrow T$ vorher einfach auf die Menge aller Terme abbilden, wäre dies nach Hinzufügen der Multi-Mustervariablen nicht mehr möglich. Entsprechend umständlicher würde auch die Beschreibung der Auswertung eines Musters werden. \\

Formal wird die Multi-Mustervariable damit nicht als neues Symbol in die Menge der Muster aufgenommen, sondern ist lediglich eine vereinfachende Schreibweise, die wie auch vorher immer für eine beliebige Anzahl an Teiltermen steht, in diesem Fall Mustervariablen. Ein Muster mit einer Multi-Mustervariable $\mathbf{xs...}$ repräsentiert formal unendlich viele konkrete Muster mit konkreten Mustervariablen $\mathbf{x_i}$:
\begin{equation*}
	\begin{split}
			(f, \mathbf{ts...}) = \{&(f), \\
			&(f, \mathbf{x_1}),\\
			&(f, \mathbf{x_1}, \mathbf{x_2}), \\
			&(f, \mathbf{x_1}, \mathbf{x_2}, \mathbf{x_3}), \\
			&\dots \}    		
	\end{split}
\end{equation*}
Für die folgenden Algorithmen dieses Kapitels, sowie der echten Umsetzung, ist es allerdings nicht praktikabel diese Definition anzuwenden. Mit der Restriktion, dass jede Multi-Mustervariable auf der linken Seite einer Ersetzungsregel höchstens ein Mal vorkommen darf, ist eine sehr einfache Verwaltung möglich. Für Funktionsanwendungen kommutativer Funktionssymbole in einem Muster muss lediglich zwischen \textit{enthält eine Multi-Mustervariable} und \textit{enthält keine Multi-Mustervariable} unterschieden werden. Multi-Mustervariablen in Funktionsanwendungen nicht-kommutativer Funktionssymbole haben allerdings eine eindeutige Position. Im Folgenden werden aber auch hier keine weiteren Parameter für Multi-Mustervariablen hinzugefügt. Alternativ wird für jeden tatsächlichen Term in den Parametern einer solchen Funktionsanwendung festgehalten, ob er Nachfolger einer Multi-Mustervariable ist und weiter, ob an dem letzten Parameter noch eine Multi-Mustervariable anschließt. 
Die Ersetzungsregel für die erste Binomische Formel anwendbar auf Summen beliebiger Länge wird demgemäß  geschrieben als:
$$(\texttt{sum}, (\texttt{pow}, \mathbf a, 2), (\texttt{prod}, 2, \mathbf a, \mathbf b), (\texttt{pow}, \mathbf b, 2), \mathbf{cs...}) \mapsto (\texttt{sum}, (\texttt{pow}, (\texttt{sum}, \mathbf a, \mathbf b), 2), \mathbf{cs...}).$$
Die Summe der linken Seite setzt sich für die folgenden Algorithmen dieses Kapitels dennoch nur aus drei Summanden zusammen. Die syntaktisch als Parameter geschriebenen $\mathbf{cs...}$ kommen hier in der linken Seite der Regel nur als Wahrheitswert vor, denn es gilt \glqq Die Summe \textit{enthält eine Multi-Mustervariable}\grqq{}. Lediglich auf der rechten Seite ist relevant, um welche Multi-Mustervariable es sich handelt, sollte es mehrere geben. Dieses Kapitel befasst sich allerdings fast ausschließlich mit der linken Seite einer Regel.



%.........................................................................
%.........................................................................
%.........................................................................
\section{Kommutative Muster} \label{subsecACMuster}

Die Algorithmen \ref{findMatch}, \ref{rematch}, \ref{findPermutation}, \ref{findDilation} und \ref{findIdentic} bilden zusammen die Grundlage des finalen Matchalgorithmus dieser Arbeit. Der Startpunkt einer Matchsuche ist der Aufruf von $\Const{findMatch}$. Hier wird allerdings im Kontrast zu $\Const{simpleMatchAlgorithm}$ nicht direkt ein Rekursionsaufruf durchgeführt, sondern abhändig von der Form der vorgefundenen Funktionsanwendung eine entsprechende Strategie für die Suche eines Matches gewählt. Die Algorithmen, die die entsprechenden Strategien implementieren sind $\Const{findPermutation}$, $\Const{findDilation}$ und $\Const{findIdentic}$. Alle hier vorgestellten Strategien nutzen dabei Backtracking um die verschiedenen Möglichkeiten zu testen. Sollte dabei die Notwendigkeit auftreten, für einen bereits gematchten Teil des Musters ein neues Match mit dem selben Literal zu finden, wird in allen drei Strategien $\Const{rematch}$ aufgerufen. Dieser Algorithmus ist ähnlich zu $\Const{findMatch}$, erwartet dementsprechend allerdings, dass das übergebende Muster $p$ bereits mit dem übergebenen Literal $t$ gematcht ist. Die eigentliche Arbeit wird bei $\Const{rematch}$ allerdings erneut an die Suchstrategien abgegeben. Da der Startpunkt dort allerdings davon abhängig ist, ob bereits eine bestimmte Zuordnung der Argumente als erfolgreich matchend festgehalten ist oder nicht, wird diese Information als letzter Parameter jeder Strategie mit übergeben.


Im Grundaufbau funktionieren alle Strategien gleich. Die Argumente $\elems p 0 {m-1}$ des Musters $p$ werden in der vorliegenden Reihenfolge mit den Argumenten $\elems t 0 {n-1}$ des Literals $t$ gematcht. Kann für das aktuelle Argument $p_i$ kein Match mehr gefunden werden, wird probiert die vorhergehenden Argumente $\elems p 0 {i-1}$ neu zu matchen, beginnend mit $p_{i-1}$. Mit welchen Argumenten $t_k$ ein Match dabei erlaubt ist, ist nach Strategie unterschiedlich. Am stärksten eingeschränkt ist $\Const{findIdentic}$. Musterargument $p_i$ kann dort nur mit $t_k$ gematcht werden, wenn $k = i$ gilt\footnote{Algorithmus \ref{simpleMatchAlgorithm} hat ausschließlich auf diese Weise nach einem Match gesucht}. Das andere Extrem stellt $\Const{findPermutation}$ da. Hier kann jedes $p_i$ mit jedem $t_k$ gematcht werden, vorausgesetzt $t_k$ ist noch nicht mit einem Argument aus $\elems p 0 {i-1}$ gematcht. In der Freiheit dazwischen steht $\Const{findDilation}$, welche die Reihenfolge der $p_i$ untereinander gleich halten muss, jedoch eine Lücke beliebiger Länge zwischen $p_{i-1}$ und $p_i$ erlaubt, sofern im Muster an dieser Stelle eine Multi-Mustervariable steht\footnote{Wie in Abschnitt \ref{subsecMulti} erörtert, treten diese hier nicht als echte Argumente auf.}.

\begin{algorithm}
\DontPrintSemicolon
\caption{$\Const{findMatch} \colon M \times T \rightarrow \mathit{Bool}$}\label{findMatch}
\KwIn {$p \in M$, $t \in T$}

\If {$p \in X$ \KwAnd $p$ bindend}{
	merke: $v_p~p = t$\;
	\Return {$\Const{true}$}
}
\ElseIf {$p \in X$ \KwAnd $p$ gebunden}{
	\Return {$v_p~p = t$}
}
\ElseIf {$p \in C \setminus X$} {
	\Return {$p = t$}
}
\ElseIf {$p = (f, \elems p 0 {m-1})$ \KwAnd $t = (f, \elems t 0 {n-1})$}{
	\If {$u~f$ kommutativ} {
		\Return {$\Const{findPermutation}(p, t, \Const{false})$}
	}
	\ElseIf {$\elems t 1 n$ enthalten Multi-Mustervariablen} {
		\Return {$\Const{findDilation}(p, t, \Const{false})$}
	}
	\ElseIf {$m = n$} {
		\Return {$\Const{findIdentic}(p, t, \Const{false})$}
	}
}
\Return{$\Const{false}$}  
\end{algorithm}

\begin{algorithm}
\DontPrintSemicolon
\caption{$\Const{rematch} \colon M \times T \rightarrow \mathit{Bool}$}\label{rematch}
\KwIn {$p \in M$, $t \in T$}
\If {$p = (f, \elems p 0 {m-1})$ \KwAnd $t = (f, \elems t 0 {n-1})$} {
	\If {$u~f$ kommutativ} {
		\Return {$\Const{findPermutation}(p, t, \Const{true})$}
	}
	\ElseIf {$\elems t 1 n$ enthalten Multi-Mustervariablen} {
		\Return {$\Const{findDilation}(p, t, \Const{true})$}
	}
	\ElseIf {$m = n$} {
		\Return {$\Const{findIdentic}(p, t, \Const{true})$}        
	}
}        
\Return {$\Const{false}$}  
\end{algorithm}

Abweichend von bisherigen Algorithmen wird von hier an im Pseudocode nicht mehr jede tatsächlich notwendige Information explizit übergeben. Als Beispiel muss $v_p$ nach wie vor von jedem Funktionsaufruf aktualisiert werden, ist aber im Pseudocode der Algorithmen \ref{findMatch}, \ref{findPermutation}, \ref{findDilation}, etc. nicht länger explizit in Parameterliste oder als Rückgabewert erwähnt.
Anstelle der konkreten Zuweisung eines Wertes zu einem Namen, dargestellt duch den Pfeil nach links \glqq $\leftarrow$\grqq{}, wird die Veränderung einer solchen nicht explizit erwähnten Datenstruktur nur mit dem Wort \glqq merke\grqq{} dargestellt.

\subsection{findMatch und rematch}
Algorithmus \ref{findMatch} ist in der Struktur ähnlich zu $\Const{simpleMatchAlgorithm}$. Neben dem Auslagern der Rekursionsaufrufe in die verschiedenen Matchstrategien, besteht ein Unterschied im Umgang mit Mustervariablen. Für $\Const{simpleMatchAlgorithm}$ wird getestet, ob der Funktionswert $v_p~\mathbf x$ für eine Mustervariable $\mathbf x$ bereits definiert ist, wenn $\mathbf x$ angetroffen wird und herausgefunden werden muss ob die Instanz bindend oder gebunden ist. Das reicht für $\Const{findMatch}$ nicht, da dieser Algorithmus auch funktionieren muss, wenn die verschiedenen Matchstrategien ein Backtracking beinhalten, d.h., $\Const{rematch}$ aufrufen.
Der Algorithmus $\Const{rematch}$ ist fast identisch zur unteren Hälfte von $\Const{findMatch}$, ruft die verschiedenen Matchstrategien allerdings mit $\Const{true}$ als letzen Parameter auf, was bedeutet, dass direkt zum Backtracking gesprungen wird.



\subsubsection {findIdentic}
\begin{algorithm}
\DontPrintSemicolon
\caption{$\Const{findIdentic} \colon M \times T \times \mathit{Bool} \rightarrow \mathit{Bool}$}\label{findIdentic}
\KwIn {$p = (f, \elems p 0 {n-1}) \in M$, $t = (f, \elems t 0 {n-1}) \in T$, $\mathit{starteGematcht} \in \mathit{Bool}$}
\Let {$i \leftarrow 0$}\;
\If {$\mathit{starteGematcht}$} {
	$i \leftarrow n$\;
	\Goto \texttt{\ref{backtrackRematchMuster}}\;
}
\Loop {} {
	\nlset{matche $p_i$} \label{backtrackMatchMuster}
	\While {$\Const{findMatch}(p_i, t_i)$} {
		$i \leftarrow i + 1$\;
		\lIf {i = n} {\Return {$\Const{true}$}}    
	}
	\nlset{zurück} \label{backtrackRematchMuster}
	\DoWhile {$\Const{not}$ $\Const{rematch}(p_i, t_i)$} { 
		\lIf {i = 0} {\Return {$\Const{false}$}}
		$i \leftarrow i - 1$\;
	}    
	$i \leftarrow i + 1$\;
}
\end{algorithm}


Als Strategie mit den wenigsten Freiheiten ist die Umsetzung von $\Const{findIdentic}$ die kürzeste. Die Laufvariable $i$ steht gleichzeitig als Index für die Argumente von Muster und Literal. Wenn der Aufruf aus $\Const{findMatch}$ erfolgt, wird mit $i = 0$ gestartet und in Abschnitt \texttt{\ref{backtrackMatchMuster}} versucht für alle $i$ bis $n-1$ $p_i$ mit $t_i$ zu matchen. Sollte das  für ein $i$ fehlschlagen, besteht die Hoffnung, dass einer der Argumente $p_j \in \{\elems p 0 {i-1}\}$ anders als bisher mit $t_j$ gematcht werden kann, wodurch dann das Match von $p_i$ mit $t_i$ ermöglicht wird. Gefunden wird $p_j$ in Abschnitt \texttt{\ref{backtrackRematchMuster}}. 
Soll das gesamte Muster neu gematcht werden, startet $\Const{findIdentic}$ bei einem Aufruf durch $\Const{rematch}$ deswegen bei \texttt{\ref{backtrackRematchMuster}} und mit $i = n$.

\begin{lemma}\label{lemKomplexitaetFindPermutation}~\\
Die Laufzeitkomplexität von Algorithmus \ref{findIdentic} bei der Suche eines Matches für ein Muster $p = (f, \elems {\mathbf x} 0 {n-1})$ mit einem Literal $t = (f, \elems t 0 {n-1})$ ist in $\mathcal O(n)$, wenn jeder Parameter $t_i$ von $t$ nur $\mathcal O(1)$ Konstantensymbole und Funktionssymbole besitzt.
\end{lemma}

\textbf{Beweis}.\\
Der Ausdruck $\Const{not}$ $\Const{rematch}(pi , ti)$ ist für kein $i$ wahr, da $\Const{rematch}$ für Mustervariablen immer $\Const{false}$ zurückgibt. Die äußere Schleife wird also nur exakt ein Mal durchlaufen. Sowohl ein Durchlauf der \textbf{while}-Schleife, als auch ein Durchlauf der \textbf{do-while}-Schleife ist in $\mathcal O(1)$, da jedes $t_i$ nur $\mathcal O(1)$ Teilterme hat, bzw. $\Const{rematch}(pi , ti)$ für Mustervariblen $p_i$ direkt $\Const{false}$ zurückgibt. Entweder wird der Abschnitt \texttt{\ref{backtrackMatchMuster}} exakt $n$ Mal abgelaufen und $\Const{true}$ zurückgegeben oder $n' < n$ Mal abgelaufen, woraufhin auch Abschnitt \texttt{\ref{backtrackRematchMuster}} $n'$ Mal abgelaufen wird, bis $\Const{false}$ zurückgegeben wird. 
\hfill $\square$\\



\subsubsection {findPermutation}
\begin{algorithm}
\DontPrintSemicolon
\caption{$\Const{findPermutation} \colon M \times T \times \mathit{Bool} \rightarrow \mathit{Bool}$}\label{findPermutation}
\KwIn {$p = (f, \elems p 0 {m-1}) \in M$, $t = (f, \elems t 0 {n-1}) \in T$, $\mathit{starteGematcht} \in \mathit{Bool}$}
\Let {$i \leftarrow 0$, $k \leftarrow 0$}\;
\If {$\mathit{starteGematcht}$} {
	$i \leftarrow m$\;
	\Goto \texttt{\ref{permutationRematchMuster}}\;
}
 \If {$m > n$} {
	\Return {$\Const{false}$}
 }
 \nlset{matche $p_i$}\label{PermutationHauptschleifenbeginn}
 \While {$i < m$} {
	\While {$k < n$} {
		\If {$t_k$ ist mit keinem Parameter von $p$ gematcht} {
			\If {$\Const{findMatch}(p_i, t_k)$} {
				\Goto \texttt{\ref{permutationNaechstesMuster}}\;
			}
		}
		$k \leftarrow k + 1$\;
	}
	\nlset{zurück}\label{permutationRematchMuster}
	\If {$i = 0$} {
		\Return {$\Const{false}$}
	}
	$i \leftarrow i - 1$\;
	{$k \leftarrow k'$ aus \glqq $p_{i}$ ist mit $t_{k'}$ gematcht\grqq{}}\;
	\If {not $\Const{rematch}(p_{i}, t_{k})$} {
		merke: $p_{i}$ ist nicht mehr mit $t_{k}$ gematcht\;
		$k \leftarrow k + 1$\;
		\Goto \texttt{\ref{PermutationHauptschleifenbeginn}}\;
	} 
	\nlset{weiter}\label{permutationNaechstesMuster}
	merke: $p_i$ ist mit $t_k$ gematcht\;
	$i \leftarrow i + 1$\;
	$k \leftarrow 0$\;    
 }
 \Return {$p$ enthält eine Multi-Mustervariable \KwOr alle $t_k$ wurden gematcht}
\end{algorithm}

Algorithmus \ref{findPermutation} probiert das Muster einer kommutativen Funktionsanwendung $p$ auf ein Literal $t$ der gleichen Form zu matchen. Die beiden Laufvariablen $i$ und $k$ sind Index für die Musterargumente $\elems p 0 {m-1}$, bzw. der Argumente des Literals $\elems t 0 {n-1}$. Für die Suche eines Matches wird zuerst versucht $p_0$ mit $t_0$ zu matchen. Schlägt das fehl, wird $k$ hochgezählt, bis $p_0$ ein $t_k$ matchen kann oder jedes $t_k$ getestet wurde. Im erfolglosen Fall wird die Suche beendet, da für $p_0$ alle verfügbaren Freiheitsgerade getestet wurden. Wurde $p_0$ erfolgreich mit $t_k$ gematcht, wiederholt sich der Prozess für $p_1$, mit der Ausname, dass $t_k$ jetzt nicht mehr als Matchkandidat zur Verfügung steht. Wird ein Argument gefunden, das $p_1$ matcht, wird $i = 3$ gesetzt und der Prozess wiederholt sich für $p_3$. Falls die Suche für $p_1$ in dem Durchlauf erfolglos war, heißt es allerdings nicht, dass kein Match von $p$ und $t$ möglich ist. Es besteht die Option, dass $p_0$ mit $t_k$ noch auf eine andere Weise als die bisherige gematcht werden kann. Beinhaltet $p_1$ Mustervariablen, die in $p_0$ bindend vorkommen, eröffnet ein ändern der Bindung möglicherweise neue Matchmöglichkeiten für $p_1$. Aus dem Grund wird im Abschnitt \texttt{\ref{permutationRematchMuster}} zuerst versucht $p_0$ mit $t_k$ zu rematchen. Sollte das fehlschlagen, ist es möglich, dass $p_0$ noch mit Parametern von $t$ gematcht werden kann, die nach $t_k$ aufgelistet sind, was wiederum $p_1$ erlauben würde, ein Match mit $t_k$ zu testen. Auch diese Option wird ausprobiert.
Wurde für alle $p_i$ ein Match gefunden, so ist ganz $p$ mit ganz $t$ gematcht, falls gleichzeitig alle $t_k$ gematcht sind oder $p$ eine Multi-Mustervariable enthält. Sollte keiner der beiden Fälle eintreten, ist an dieser Stelle kein Match von $p$ und $t$ möglich. 


\begin{lemma}\label{lemKomplexitaetFindPermutation}~\\
Die Laufzeitkomplexität von Algorithmus \ref{findPermutation} bei der Suche eines Matches für ein Muster $p = (f, \elems {\mathbf x} 0 {m-1})$ mit einem Literal $t = (f, \elems t 0 {n-1})$ ist in $\mathcal O(n^m)$, wenn jedes Argument $t_k$ von $t$ nur $\mathcal O(1)$ Konstantensymbole und Funktionssymbole besitzt.
\end{lemma}

\textbf{Beweis}.\\
Der Beweis erfolgt als Induktion über $m$.
Für den Induktionsanfang mit $m = 1$ muss $\mathbf x_0$ höchstens mit allen $n$ Argumenten des Literals verglichen werden. Unabhängig davon, ob die Instanz von $\mathbf x_0$ bindend oder gebunden ist, terminiert $\Const{findMatch}(\mathbf x_0, t_k)$ in $\mathcal O(1)$, da alle $t_k$ in ihrer Größe beschränkt sind. \\
Im allgemeinen Fall kann die Anwendung von $\Const{findPermutation}$ mit $p$ und $t$ auf höchstens $m$ Anwendungen des Algorithmus mit Mustergröße $m-1$ zurückgeführt werden. Erneut kann $\mathbf x_0$ potenziell mit jedem $t_k$ matchen. Das Match mit $t_k$ erfolgt wie im Indunktionsanfang beschrieben in $\mathcal O (1)$. Die anschließende Suche nach Matches für $x_i$ mit $i > 0$ ist in der Komplexität äquivalent zu einem neuen Aufruf von $\Const{findPermutation}$ mit dem Muster $p' = (f, \elems {\mathbf x} 1 {m-1})$ und dem Literal $t' = (f, \elems t 0 {k-1}, t_k', \elems t {k+1} {n-1})$, wobei $t_k'$ ein spezieller Wert ist, mit dem ein Match zu jedem Muster in $\mathcal O (1)$ abgelehnt wird \footnote{Der Wert von $t_k'$ ist sonst nicht notwendig, da in Bereich \texttt{\ref{PermutationHauptschleifenbeginn}} durch die if-Abfrage der problematische Matchversuch mit $t_k$ umgangen wird.}. Gibt dieser Aufruf $\Const{false}$ zurück, gibt auch $\Const{rematch}(\mathbf x_0, t_k)$ in $\mathcal O (1)$ $\Const{false}$ zurück. Der Übergang von $t_k$ zu $t_{k-1}$ erfolgt ebenfalls in $\mathcal O (1)$. Für jedes der bis zu $n$ Matches von $\mathbf x_0$ mit einem $t_k$ treten damit Laufzeitkosten von $\mathcal O (1)$ außerhalb der Rekursion auf. Jeder der $n$ Rekursionsaufrufe hat nach Induktionshypothese eine Laufzeit in $\mathcal O (n^{m-1})$. Insgesamt ergibt sich so also eine Laufzeit in $n \cdot \mathcal O (1) \cdot \mathcal O (n^{m-1}) = \mathcal O (n^m)$.
\hfill $\square$\\


\subsubsection {findDilation}
\begin{algorithm}
\DontPrintSemicolon
\caption{$\Const{findDilation} \colon M \times T \times \mathit{Bool} \rightarrow \mathit{Bool}$}\label{findDilation}
\KwIn {$p = (f, \elems p 0 {m-1}) \in M$, $t = (f, \elems t 0 {n-1}) \in T$}
\Let $i \leftarrow 0$, $k \leftarrow 0$\;
\If {$\mathit{starteGematcht}$} {
	$i \leftarrow m$\;
	\Goto \texttt{\ref{dilationRematchLastNeedle}}\;
}
\If {$m = 0$} {
	\Return {$\Const{true}$}
}
\If {$n = 0$} {
	\Return {$\Const{false}$}
}
\nlset{matche $p_i$} \label{dilationMatchCurrentNeedle} 
\If {$k < n$} {
	\DoWhile {$p_i$ ist Nachfolger einer Multi-Mustervariable \KwAnd $k < n$} { 
		\If {$\Const{findMatch}(p_i, t_k)$} {
			\Goto \texttt{\ref{dilationPrepareNextNeedle}}\;
		}
		$k \leftarrow k + 1$\;
	}
}
\nlset{zurück} \label{dilationRematchLastNeedle} 
\While {$i > 0$} {
	$i \leftarrow i - 1$\;
	{$k \leftarrow k'$ aus \glqq $p_{i}$ ist mit $t_{k'}$ gematcht\grqq{}}\;
	\If {$\Const{rematch}(p_i, t_k)$} {
		\Goto \texttt{\ref{dilationPrepareNextNeedle}}\;
	}
	\ElseIf {$p_i$ ist Nachfolger einer Multi-Mustervariable} {
		$k \leftarrow k + 1$\;
	\Goto \texttt{\ref{dilationMatchCurrentNeedle}}\;
	}
}
\Return{$\Const{false}$}\;
\nlset{weiter} \label{dilationPrepareNextNeedle} 
merke: $p_i$ ist mit $t_k$ gematcht\;
$i \leftarrow i + 1$\;
$k \leftarrow k + 1$\;
\If {$i < m$} {
	\Goto \texttt{\ref{dilationMatchCurrentNeedle}}\;
}
\ElseIf {$k < n$ \KwAnd nach $p_{m-1}$ folgt keine Multi-Mustervariable} {
	\Goto \texttt{\ref{dilationRematchLastNeedle}}\;
}
\Return {$\Const{true}$}
\end{algorithm}


Algorithmus \ref{findDilation} teilt die Musterargumente $\elems p 0 {m-1}$ in Blöcke der Form $\elems p i j$ ein. Ein solcher Block muss Elementweise einen Block $\elems t k {k + j - i}$ der Argumente des Literals matchen. Die Aufteilung der Musterargumente in Blöcke ist dabei fest: Wenn zwischen $p_{i-1}$ und $p_{i}$ eine Multi-Mustervariable liegt, ist dort eine Blockgrenze. Aus dem Grund wird in \texttt{\ref{dilationMatchCurrentNeedle}} nur mehr als ein Schleifendurchlauf erlaubt, wenn $p_i$ einen neuen Block beginnt, $t_k$ also nicht durch den bereits gematchen Beginn des Blockes festgelegt ist. 
Konnte $p_i$ nicht gematcht werden, wird  $i$ im Bereich \texttt{\ref{dilationRematchLastNeedle}} so lange heruntergezählt, bis entweder $\Const{rematch}$ erfolgreich ist oder $p_i$ das erste Element des aktuellen Blocks ist. Besonderes Verhalten tritt erneut für $i = 0$ auf. War jeder Matchversuch für $p_0$ erfolglos, gibt es keine Möglichkeit mehr die Teilterme von $p$ so zu \glqq strecken\grqq{}\footnote{daher auch der Name $\Const{findDilation}$}, dass ein Match mit $t$ gefunden werden kann, ähnlich der Situation für $p_0$ in $\Const{findPermutation}$. Anders als bei $\Const{findPermutation}$ wird $k$ im Bereich \texttt{\ref{dilationPrepareNextNeedle}} allerdings hochgezählt, da $p_i$ nie ein Literal $t_k$ matchen darf, wenn $p_{i-1}$ bereits mit $t_l$ gematcht ist und $l > k$.


\begin{lemma}\label{lemKomplexitaetDilation}~\\
Die Laufzeitkomplexität von Algorithmus \ref{findDilation} bei der Suche eines Matches für ein Muster $p$  mit einem Literal $t = (f, \elems t 0 {n-1})$ ist in $\mathcal O(n^m)$, wenn jedes Argument $t_k$ von $t$ nur $\mathcal O(1)$ Konstantensymbole und Funktionssymbole besitzt und $p$ eine Funktionsanwendung von $f$ auf $m$ Mustervariablen $\elems {\mathbf x} 0 {m-1}$ ist, 
wobei jede Mustervariable $\mathbf x_i$ Nachfolger einer Multi-Mustervariable ist und auf $x_{m-1}$ eine Multi-Mustervariable folgt. Für $m = 3$ gilt also $p = (f, \mathbf{as...}, \mathbf x_0, \mathbf{bs...}, \mathbf x_1, \mathbf{cs...}, \mathbf x_2, \mathbf{ds...})$.
\end{lemma}

\textbf{Beweis}.\\
Sei $D(m, n)$ die asymptotische Laufzeit von Algorithmus \ref{findDilation}. Erneut wird ein Induktionsbeweis über $m$ beschrieben.
Mit $m = 1$ ist der Fall identisch zu dem Indunktionsanfang des Beweises von Lemma \ref{lemKomplexitaetFindPermutation}, es gilt $D(1, n) = \mathcal O(n)$. 

Für $m > 1$ wird $D(m, n)$ auf $D(m-1, n)$ zurückgeführt. Für jedes $k$ muss nach erfolgreichem Match von $\mathbf x_0$ mit $t_k$ versucht werden, die restlichen Mustervariablen $\elems {\mathbf x} 1 {m-1}$ in den restlichen Argumenten $\elems {t} {k+1} {n-1}$ des Literals $t$ zu matchen. Das entspricht einem erneuten Aufruf von $\Const{findDilation}$ mit neuem Muster $p'$ ohne $\mathbf x_0$ und neuem Liteal $t' = (f, \elems t {k+1}, {n-1})$.
Im rechenaufwändigsten Fall passiert das für jedes $k$.
$$D(m, n) = \sum_{k = 0}^{n-1} \paren*{\mathcal O(1) + D(m-1, n - k - 1)}$$
Trotz der vorgegebenden Reihenfolge der Matches von Musterargumenten im Literal, folgt die selbe Komplexitätsabschätzung wie für $\Const{findPermutation}$.
\begin{equation*}
	\begin{split} 
		D(m, n) 
		&= \sum_{k = 0}^{n-1} \paren*{\mathcal O(1) + D(m-1, n - k - 1)}\\ 
		&< \sum_{k = 0}^{n-1} \paren*{\mathcal O(1) + D(m-1, n)}\\
		&= \mathcal O(n) + \mathcal O(n) \cdot D(m-1, n)\\
		&= \mathcal O(n) \cdot D(m-1, n)\\
		&= \mathcal O(n^m)
	\end{split}
\end{equation*}
\hfill $\square$\\


%.........................................................................
%.........................................................................
%.........................................................................
\section{Bessere Laufzeit für kommutative Muster} \label{subsecCMuster}

Benanav zeigte 1987, dass auch das Matchproblem mit einem kommutativen Funktionssymbol NP-vollständig ist \cite{NPHardMatching}. Dennoch kann die Laufzeit von $\Const{findPermutation}$ für bestimmte Arten von Mustern verbessert werden. In diesem Abschnitt wird das versucht, indem a priori ausgeschlossen wird, dass bestimmte Reihenfolgen von Musterparametern erfolgreich matchen können. Diese Reihenfolgen müssen dann im Algorithmus nicht mehr geprüft werden.
Vorraussetzung dafür wird sein, dass sowohl Muster als auch Literal mit $\Const{normalize}$ aus Kapitel \ref{secErsteNormalform} normalisiert werden. Insbesondere die Sortierung nach der Relation $<$ aus Definition \ref{defOrdnungKleiner} ist relevant.
Bestimmte Muster mit kommutativen Funktionssymbolen können mit dieser einfachen Maßname bereits in linearer Zeit mit einem Literal abgeglichen werden. Besteht ein Muster etwa aus der Anwendung eines komutativen Funktionssymbols $f$ auf $m$ Argumente $\elems p 1 m$ und sind die Argumente $p_i$ ausschließlich Funktionsanwendungen paarweise verschiedener Funktionssymbole, ist garantiert, dass die Argumente eines normalisierten Literals für ein Match in der selben Reihenfolge liegen müssen. Wenn auch alle Teilmuster $p_i$ dieser Struktur folgen, bzw. nicht kommutativ sind, wird ein Match zuverlässig bereits mit $\Const{findIdentic}$ (Algorithmus \ref{findIdentic}) gefunden\footnote{ohne Berücksichtigung von Matches, welche durch Assoziativität ermöglicht würden}. Das liegt daran, dass für alle $i, j \in \{1, \dots, m\}, i < j$ gilt, dass $p_i$ ausschließlich Literale matchen kann, die vor jedes Literal sortiert werden, welches mit $p_j$ gematcht werden könnte. Diese Idee der Ordnung von Mustern wird im Folgenden ausgeführt.


\begin{definition}~\\
Man sagt dass Muster $p_1$ ist \emph{stark kleiner} als das Muster $p_2$ oder $p_1 \prec p_2$, wenn für alle Matchfunktionen $v_p$ gilt, dass $\Const{lit}(p_1, v_p) < \Const{lit}(p_2, v_p)$. Gilt immer $\Const{lit}(p_1, v_p) \leq \Const{lit}(p_2, v_p)$, sagt man $p_1$ ist \emph{stark kleiner-gleich} als $p_2$ oder $p_1 \preceq p_2$.
\end{definition}

\begin{beispiel}~\\
Wenn $\texttt{sin} < \texttt{cos}$, gilt $p_1 \prec p_2$ für $p_1 = (\texttt{pow}, (\texttt{sin}, \mathbf x), 2)$ und $p_2 = (\texttt{pow}, (\texttt{cos}, \mathbf y), 2)$. Im Kontrast sind die Muster $\hat p_1 = (\texttt{pow}, \mathbf x, 2)$ und $\hat p_2 = (\texttt{pow}, \mathbf y, 3)$ zueinander nicht stark geordnet. Gezeigt werden kann das mit Literalten $t_1$, $t_2$ und $t_3$, wenn ${t_1 < t_2 < t_3}$ gilt, aber $\hat p_1$ sowohl $t_1$ als auch $t_3$ matchen kann und $\hat p_2$ das Literal $t_2$ matchen kann, da sie die Mustervariablen von $\hat p_1$ nicht mit denen von $\hat p_2$ überschneiden. Mit $1 < 2 < 3$ erfüllen $t_1 = (\texttt{pow}, 1, 2)$, $t_2 = (\texttt{pow}, 2, 3)$ und $t_3 = (\texttt{pow}, 3, 2)$ die Bedingungen.
\end{beispiel}

\begin{lemma}~\\
Sei $p$ ein beliebiges Muster ohne die Mustervariable $\mathbf x$, $q$ ein Muster, welches nur aus der Mustervariable $\mathbf x$ besteht und sei die Menge der Konstantensymbole $C$ ohne Minimum.
Es gilt $p \not\preceq q$, und $q \not\preceq p$.
\end{lemma}

\textbf{Beweis}.\\
Mit beliebiger Matchfunktion $v_p$ sei $t_2 = \Const{lit}~(p, v_p)$. Nach Lemma \ref{lemMinMax} gibt es die Literale $t_1 < t_2$ und $t_3 > t_2$. 
Mit $v_p'$ identisch zu $v_p$, nur $v_p'~\Const x = t_1$, gilt 
$$\Const{lit}(p, v_p') = t_2 > t_1 = \Const{lit}(q, v_p') \implies p \not\preceq q.$$
Mit $v_p''$ identisch zu $v_p$, nur $v_p''~\Const x = t_3$, gilt 
$$\Const{lit}(p, v_p'') = t_2 < t_3 = \Const{lit}(q, v_p'') \implies q \not\preceq p.$$
\hfill $\square$\\


Einfach zu sehen ist, dass nur sehr wenige Muster zueinander stark geordnet sind, wenn die allgemeinste Form des Matches erlaubt ist. Problematisch ist dabei vor allem $\Const{combine}$ (Algorithmus \ref{combine}) als Teil von $\Const{normalize}$. Dadurch wird es möglich Funktionsanwendungen als Muster mit Konstantensymbolen als Literal zu matchen. Da es aufwändig ist, überhaupt zu bestimmen, welche Muster Matches dieser Art erlauben, wird $\Const{normalize}$ hier abweichend von Kapitel \ref{secErsteNormalform} ohne $\Const{combine}$ angenommen. Nur Assoziativität und Kommutativität werden berücksichtigt.

\begin{lemma}~\\ \label{lemStarkKleinerFaelle}
Für folgende Formen von normalisierten Mustern $p$ und $q$ gilt $p \prec q$:
\begin{enumerate}
	\item{$p$ und $q$ enthalten keine Mustervariablen und $p < q$} \label{itemStarkKleiner1}
	
	\item{$p$ ist ein Konstantensymbol aber keine Mustervariable und $q$ ist eine Funktionsanwendung}  \label{itemStarkKleiner2}
	
	\item{$p$ und $q$ sind Funktionsanwendungen verschiedener Funktionssymbole $f$ und $g$ mit $f < g$}  \label{itemStarkKleiner3}
		
	\item{$p = (f, \elems p 0 {m-1})$ und $q = (f, \elems q 0 {n-1})$ sind Funktionsanwendungen des selben nicht-kommutativen Funktionssymbols $f$ und einer der folgenden Punkte trifft zu.
	\begin{enumerate}
		\item{$m < n$, $\forall j \in \{0, \dots, m-1\} \colon p_j = q_j$, $p$ hat keine Multi-Mustervariablen in seinen Argumenten und wenn $q$, dann erst nach $q_m$}
		\item{$\exists i < min\{n, m\} \colon p_i \prec q_i$, $\forall j \in \{0, \dots, i  - 1\} \colon p_j = q_j$ und weder $p$ noch $q$ haben Multi-Mustervariablen in ihren Argumenten vor $p_i$ bzw. $q_i$}
	\end{enumerate}
	} \label{itemStarkKleiner4}
	
\end{enumerate}
\end{lemma}

\textbf{Beweis}~\\
In allen Fällen wird eine Mustervariable nur an Punkten erlaubt, die nicht zur Bestimmung der Ordnung von $p$ und $q$ unter der Relation $<$ beitragen, bzw die für kein Literal anstelle der Mustervariable die Ordnung der normalisierten Terme beeinflussen würden. Trivial ist das für Fall \ref{itemStarkKleiner1}.

Für die restlichen Fälle muss klar sein, dass ein Literal der Form $(f, xs...)$ auch nach Normalisierung diese Form behält. Die Reihenfolge der Argumente und damit auch ob $f$ kommutativ ist, spielen keine Rolle. Das Normalisieren mehrerer geschachtelter Anwendungen des selben assoziativen Funktionssymbols $f$ entfernt nie die äußerste Funktionsanwendung, die Struktur bleibt also auch so erhalten.
Fall \ref{itemStarkKleiner2} und Fall \ref{itemStarkKleiner3} sind ist damit bewiesen.

Mit der selben Argumentation bleibt $f$ auch in Fall \ref{itemStarkKleiner4} immer das äußerste Funktionssymbol erhalten. 
Da $\Const{normalize}$ in Fall \ref{itemStarkKleiner4} die Argumentreihenfolge nicht ändern darf, können Mustervariablen auch nach Auswertung der Musterinterpretation nicht den vorderen Bereich der Funktionsanwendungen beeinflussen.  

\hfill $\square$\\



%.........................................................................
%.........................................................................
%.........................................................................
\section{Termersetzungssystem} \label{subsecTermersetzungssystem}

Das Ziel dieses Abschnittes ist erneut die Normalisierung eines Terms. Im Unterschied zu Kapitel \ref{secErsteNormalform} werden die Ersetzungsregeln hier nicht im Algorithmus festgelegt, sondern erst als Parameter mit übergeben. Die Thematik soll in dieser Arbeit allerdings nur angerissen werden.

\begin{algorithm}
\DontPrintSemicolon
\caption{$\Const{applyRuleset} \colon \mathit{Regelmenge} \times T \rightarrow T$ }\label{algoTES}
\KwIn {$R \in \{M \times M\}$, $t \in T$}
$t \leftarrow \Const{normalize}~t$\;
\While {$\exists (p, p') \in R$, $r$ Nachkomme von $t$, $v_p$ Match von $p$ und $r$} {
    ersetze $r$ in $t$ durch $r' = \Const{lit}(p', v_p)$\;
    $t \leftarrow \Const{normalize}~t$\;
}
\Return {$t$}
\end{algorithm}

Ist eine Ersetzungsregel auf das übergebende Literal $t = t^{(0)}$ oder ein Teil dessen anwendbar, so wird das Ergebnis der Ersetzung $t^{(1)}$ genannt. Auf dem selben Weg kann aus $t^{(1)}$ der Term $t^{(2)}$ erzeugt werden oder allgemeiner aus $t^{(i)}$ der Term $t^{(i+1)}$. Ist auf keinen Teil von $t^{(n)}$ mehr eine Regel anwendbar, wird $t^{(n)}$ als \emph{\Gls{Normalform}} von $t$ zu den übergebenden Ersetzungsregeln bezeichnet. 



\subsection {Konfluenz} \label{subsubsecKonfluenz}
Es ist möglich, dass ein Literal $t$ mit einer bestimmten Regelmenge mehr als nur eine Normalform besitzt. Eine einfache Regelmenge mit dieser Eigenschaft besteht aus zwei Regeln mit identischer linker Seite aber unterschiedlicher rechter Seite. Bestimmte Regeln können allerdings auch in Isolation mehrere Normalformen produzieren. Ein Beispiel ist die folgende Regel, welche $\mathbf x$ ausklammert:
$$(\texttt{sum}, \mathbf x, (\texttt{prod}, \mathbf x, \mathbf{ys...}), \mathbf{zs...}) 
\mapsto (\texttt{sum}, (\texttt{prod}, \mathbf x, (\texttt{sum}, 1, (\texttt{prod}, \mathbf{ys...})) , \mathbf{zs...})$$

Für das Literal $t = (\texttt{sum}, a, b, (\texttt{prod}, a, b))$ existiert sowohl die Normalform 
$$t' = (\texttt{sum}, b, (\texttt{prod}, a, (\texttt{sum}, 1, b))),$$ 
als auch 
$$t'' = (\texttt{sum}, a, (\texttt{prod}, b, (\texttt{sum}, 1, a))).$$

Das ist an dieser Stelle allerdings nicht neu: Algorithmus \ref{rematch} ($\Const{rematch}$) ist eine direkte Antwort auf Muster dieser Art. Die Fragestellung welche Mengen von Ersetzungsregeln eindeutige Normalformen unabhängig von der Reihenfolge ihrer Anwendung produzieren (\emph{\gls{konfluent}} sind) ist im allgemeinen Fall nicht entscheidbar \cite{KonfluenzUnentscheidbar}.
Einschränkungen einer Regelmenge die Konfluenz implizieren, werden etwa von Hoffmann und O'Donnell diskutiert \cite{hoffmann1982programming}. Diese sind hier allerdings nur sehr beschränkt anwendbar. Zum einen findet der Matchbegriff dort ausschließlich in seiner strengen Form Anwendung, zum anderen beinhalten die Einschränkungen etwa die Restriktion, dass jede Mustervariable höchstens ein Mal in der linken Seite einer Ersetzungsregel vorkommen darf.

Ein Weg die Problematik teilweise zu umgehen ist, parallel alle möglichen Transformationen anzuwenden und erst am Ende mit einer entsprechenden Gewichtsfunktion die gewünschte Normalform zu wählen. Aufgrund der damit verbundenen hohen Laufzeitkosten wird dieser Ansatz in den meisten Fällen von Anfang ausgeschlossen.


\subsection{Ersetzungsreihenfolge}
Wird eine gegebene Regelmenge als nicht konfluent angenommen, kann die  Normalform eines Literals $t$ nicht nur davon abhängen, welche Regel zuerst auf ihre Anwendbarkeit getestet wird, sondern auch an welcher Stelle eine Ersetzung priorisiert ist. Weiter ist es möglich, dass für bestimmte Literale manche solche Strategien eine Normalform erzeugen, während andere nicht konvergieren.

\begin{beispiel} \label{bspFakutaetTerminiert}
Die Fakultätsfunktion, repräsentiert durch das Funktionssymbol $\texttt{fact}$, kann durch die folgende Menge von Ersetzungsregeln beschrieben werden, wenn die Funktionssymbole $\texttt{eq}$, $\texttt{prod}$ und $\texttt{sub}$ mit $\Const{normalize}$ ausgewertet werden. 
\begin{align*}
    (\texttt{cond}, \texttt{true}, \mathbf x, \mathbf y) 
    &\mapsto \mathbf x 
    &(1)\\        
    (\texttt{cond}, \texttt{false}, \mathbf x, \mathbf y) 
    &\mapsto \mathbf y 
    &(2)\\        
    (\texttt{fact}, \mathbf x) 
    &\mapsto (\texttt{cond}, (\texttt{eq}, \mathbf x, 0), 1, (\texttt{prod}, \mathbf x, (\texttt{fact}, (\texttt{sub}, \mathbf x, 1)))) 
    &(3)
\end{align*}
Offensichtlich ist, dass die Normalisierung nicht konvergiert, wenn entweder die Ersetzung von Regel $(3)$ an einer beliebigen Stelle Vorzug gegenüber Anwendung der anderen Regeln hat oder wenn immer der innerste transformierbare Teil eines Literals transformiert wird.
\end{beispiel}

Die Problematik ist eng verwandt mit der Auswertungsstrategie für funktionale Programme, wo zwischen \textit{strenger Auswertung} (engl. \textit{eager evaluation}) und \textit{fauler Auswertung} (engl. \textit{lazy evaluation}) unterschieden wird  \cite{EvalStrategien}. Die strenge Auswertung wählt immer die innerste Ersetzung zuerst aus. Vorteil ist die sehr einfache Umsetzung der Strategie: Ist ein Teilterm normalisiert, wird er es auch nach Anwendung von Ersetzungsregeln auf seine Ahnen bleiben. Nachteil ist, dass weniger Terme normalisiert werden können, siehe Beispiel \ref{bspFakutaetTerminiert}. Die falue Auswertung transformiert im Kontrast immer den äußeren Teil des Literals zuerst. 

Für die Umsetzung einer faulen Ersetzungsstrategie für ein Termersetzungssystem, ist es im allgemeinen Fall allerdings schwierig, schnell den den äußersten transformierbaren Teil zu finden, nachdem eine Ersetzung stattgefunden hat. In dieser Arbeit fällt das weniger ins Gewicht, da schon der einzelne Matchversuch je nach Muster exponentielle Laufzeitkosten aufweisen kann, Hoffman und O'Donnell haben deshalb allerdings einen Hybrid aus strenger und fauler Auswertung implementiert \cite{hoffmann1982programming}.











\chapter{Umsetzung} \label{secKernUmsetzungInCpp}

Der gesamte Quelltext der Umsetzung in C\texttt{++} ist auf GitHub einsehbar \cite{brunizzl2021Jul}. Hier gezeigte Ausschnitte sind teilweise gekürzt.
Dieses Kapitel hat keinen Anspruch auf Vollständigkeit. Der Fokus liegt viel mehr darauf, die theoretisch ausführlich beschriebenen Konzepte in der Praxis zu zeigen und die wichtigsten bisher in dieser Arbeit unbehandelten Ideen vorzustellen. Aufgrund der gegenseitigen Abhängigkeit der verschiedenen Ideen sollten die Abschnitte \ref{subsecKonzeptionelleUnterschiede} und \ref{subsecSyntax} nicht direkt mit dem Anspruch eines lückenlosen Verständnisses gelesen werden, da die darauf folgenden Abschnitte noch weiter ins Detail gehen.

%.........................................................................
%.........................................................................
%.........................................................................
\section{Konzeptionelle Unterschiede} \label{subsecKonzeptionelleUnterschiede}
Die Umsetzung implementiert nicht exakt die bisher beschriebenen Strukturen. Der erste Unterschied ist, dass die Konzepte \emph{Funktionssymbol} und \emph{Konstantensymbol} hier nicht unterschieden werden. Da der Zweck der Implementierung zudem alleine in der Vereinfachung von Ausdrücken über den Komplexen Zahlen $\mathbb C$ liegt, ist die Menge an Symbolen zudem im Code nicht generisch gehalten. Die beiden wichtigsten Arten von Symbolen für Literale sind Komplexe Zahlen $z \in \mathbb C$, sowie Zeichenketten $c_1 c_2\dots c_n$ beliebiger Länge $n$. Die einzelnen Zeichen $c_i$ stammen dabei aus dem Alphabet $\Sigma$, welches aus Klein-und Großbuchstaben des lateinischen Alphabetes, Ziffern von $0$ bis $9$, den Apostroph \verb|'| und dem Unterstrich \verb|_| besteht. Die Ausnahme bildet das erste Zeichen $c_1$, welches ein Klein-oder Großbuchstabe des Lateinischen Alphabetes sein muss. Zu den Zeichenketten gehören insbesondere auch die Funktionssymbole. Mit der dadurch entstehenden Möglichkeit Funktionssymbole als Werte zu behandeln, ergibt es Sinn auch Funktionsanwendungen von dynamisch bestimmten Funktionssymbolen zuzulassen. Das erste Element $f$ des Funktionsanwendungstupels $(f, \elems t 0 {n-1})$ ist in der Umsetzung also kein Symbol, sondern ein Term. Mit den bis hier diskutierten Änderungen und $\Sigma^+$ als Bezeichnung für die Menge von Zeichenketten über dem beschriebenen Alphabet $\Sigma$ sähe die idealisierte Menge aller Literale in der Umsetzung im Kontrast zu Definition \ref{defTerm} so aus:

$$T \coloneqq \Sigma^+ \cup \mathbb C \cup \curl*{(\elems t 0 n)~|~ \elems t 0 n \in T)}.$$

Die wichtigste Erweiterung der Implementierung ist allerdings die der anonymen Funktion bekannt als \emph{Lambda}. Das Lambdakalkül von Church \cite{ChurchLambda36} definiert Terme verwandt mit den in dieser Arbeit diskutierten. Der wichtigste Unterschied zum bisherigen Funktionssymbol ist, dass keine externe Interpretation für einen gegebenen Lambdaausdruck notwendig ist. Anstatt Funktionssymbole über Namen zu identifizieren und die Abbildungsvorschrift getrennt anzugeben, ist eine Lambdafunktion $f \in \Lambda$ ausschließlich durch ihre Abbildungsvorschrift identifiziert. Church erlaubt neben der Funktionsanwendung als Term lediglich Variablensymbole  $v \in V$ und Lambdafunktionen. Soll $T$ also eine Obermenge aller Ausdrücke im Lambdakalkül werden, müsste prinzipiell nur die Lambdafunktion selbst hinzugefügt werden. Die Menge der Variablen $V$, hier als \emph{Lambdaparameter} bezeichnet wird allerdings von den bisher erlaubten Zeichenketten in $\Sigma^+$ getrennt.

Muster sind in der Umsetzung auch entsprechend flexibler, schließlich sind sie eine Obermenge der Literale. Dazu kommt die Fähigkeit, Bedingungen an Mustervariablen zu stellen, um mögliche Matches weiter einzuschränken. Eine spezielle Form dieser Einschränkung ist dabei die der \emph{Wert-Mustervariable} $w \in W$, welche versucht Matches zu finden, die in Unterkapitel \ref{subsecNormalKombinieren} erlaubt werden, also Komplexe Zahlen wieder in Rechenausdrücke zu dekonstruieren. Näher behandelt wird die Wert-Mustervariable noch im folgenden Abschnitt \ref{subsecMustervariablen}. 
Zuletzt kann die Multi-Mustervariable aus Abschnitt \ref{subsecMulti} von hier an nicht mehr nur als abstrakte Idee gehandelt werden. Die Menge der Multi-Mustervariablen wird $X^*$ genannt. 

\begin{definition} \label{defKnotentypenMathe}
Die Menge der Terme $T$ ist in diesem Kapitel definiert als
$$T \coloneqq \Sigma^+ \cup \mathbb C \cup X \cup X' \cup X^* \cup W \cup V \cup \Lambda \cup \curl*{(\elems t 0 n)~|~ \elems t 0 n \in T}$$
mit
\begin{align*}
    \Sigma^+  &\coloneqq \text{Zeichenketten}\\
    \mathbb C &\coloneqq \text{Komplexe Zahlen}\\
    X         &\coloneqq \text{Mustervariablen}\\
    X^*       &\coloneqq \text{Multi-Mustervariablen}\\
    W         &\coloneqq \text{Wert-Mustervariablen}\\
    V         &\coloneqq \text{Lambdaparameter}\\
    \Lambda   &\coloneqq \text{Lambdafunktionen}.
\end{align*}
\end{definition}






%.........................................................................
%.........................................................................
%.........................................................................
\section{Syntax} \label{subsecSyntax}
In der Umsetzung wird der Termbaum immer aus einer ASCII Zeichenkette gebaut.
Solche Zeichenketten sind von hier an in \texttt{monospace} gesetzt. Auf eine formale Definition der genutzten Grammatik wird verzichtet.

\subsection{Funktionsanwendungen}
\begin{figure}
    \label{tabZucker}
    \centering
    \begin{tabular}{l l}
        \hline
        Normale Schreibweise & Alternative Syntax\\
        \hline \hline
        \verb|prod(-1, x)|         & \verb|-x|\\
        \verb|not(x)|              & \verb|!x|\\
        \verb|pow(x, y)|           & \verb|x^y|\\
        \verb|prod(x, y)|          & \verb|x * y|\\
        \verb|prod(x, y)|          & \verb|x y|\\
        \verb|prod(x, pow(y, -1))| & \verb|x / y|\\
        \verb|sum(x, y)|           & \verb|x + y|\\
        \verb|sum(x, prod(-1, y))| & \verb|x - y|\\
        \verb|cons(x, y)|          & \verb|x :: y|\\
        \verb|eq(x, y)|            & \verb|x == y|\\
        \verb|neq(x, y)|           & \verb|x != y|\\
        \verb|greater(x, y)|       & \verb|x > y|\\
        \verb|smaller(x, y)|       & \verb|x < y|\\
        \verb|greater_eq(x, y)|    & \verb|x >= y|\\
        \verb|smaller_eq(x, y)|    & \verb|x <= y|\\
        \verb|and(x, y)|           & \verb|x && y|\\
        \verb|or(x, y)|            & \verb!x || y!\\
        \verb|of_type__(x, y)|     & \verb|x :y|\\
        \hline
    \end{tabular}
    \caption{alternative Syntax für bestimmte Funktionssymbole}
\end{figure}

Die bisher genutzte Schreibweise $(f, x, y, z)$ für die Funktionsanwendung des Funktionssymbols $f$ auf die Parameter $x, y, z$ wurde in Abschnitt \ref{subsecTerm} eingeführt, um den Term syntaktisch von anderen Ideen zu differenzieren. Dies ist für die Umsetzung in C\texttt{++} nicht notwendig, da von vorne herein klar ist, ob eine Zeichenkette in einen Term übersetzt wird. Aus dem Grund werden Funktionsanwendungen in der Syntax \verb|f(x, y, z)| geparst. 
Weiter können Funktionsanwendungen bestimmter Funktionen auch mit fest definierten Infixoperatoren geschrieben werden, siehe Tabelle \ref{tabZucker}. Die Bindekraft der Operatoren nimmt in der Tabelle von oben nach unten ab.
Aus der Tabelle hervorzuheben sind zwei Dinge: Zum einen wird ein Leerzeichen zwischen zwei Termen als Multiplikation interpretiert, Funktionsanwendungen dürfen also kein Leerzeichen zwischen Funktion und Parametertupel setzen. \verb|f (x)| ist dementsprechend gleichbedeutend zu \verb|prod(f, x)|, während \verb|f (x, y)| einen Syntaxfehler darstellt. Die zweite Besonderheit ist das Fehlen von Funktionssymbolen für die Darstellung des additiven Inversen und des multiplikativen Inversen. Dies reduziert die Anzahl der Ersetzungsregeln, die benötigt werden um eine Gesetzmäßigkeit, die Summen oder Produkte involviert, abzubilden. 

\subsection{Lambdafunktionen} \label{subsubsecLambdaSyntax}
Da der ASCII Zeichensatz keine griechischen Buchstaben enthält, wird anstelle des kleinen Lambdas $\lambda$ der umgekehrte Schrägstrich \verb~\ ~genutzt. Die Identitätsfunktion $\lambda x.x$ wird dementsprechend \verb~\x .x~ geschrieben. Mehrere Parameter werden durch Leerzeichen getrennt: \verb~\x y .pow(x, y)~ ist eine Lambdafunktion mit identischem Verhalten zum Funktionssymbol \verb|pow|.

\subsection{Symbole}
Komplexe Zahlen auf der reellen Achse sind in der Syntax vergleichbar mit Darstellungen für Integer und Fließkommazahlen erlaubt in C. Möglich sind etwa \verb|3.1415|, \verb|42|, \verb|1.337e3| oder \verb|1.602e-19|. Komplexe Zahlen auf der imaginären Achse sind in der Struktur identisch, allerdings immer direkt gefolgt vom Zeichen \verb|i|. 

Die verschiedenen Mustervariablen werden auf unterschiedliche Weise identifiziert. Eine normale Mustervariable muss mit einem Unterstrich beginnen (\verb|_x|), eine Wert-Mustervariable mit einem Dollar (\verb|$x|) und eine Multi-Mustervariable endet in drei Punkten (\verb|xs...|). Besitzt ein Name keiner dieser besonderen Merkmale, wird daraus kontextabhängig ein Lambdaparameter oder ein Symbol aus $\Sigma^+$. Ist der umgebende Name bereits in einem umschließenden Lambda gebunden, wird die innerste solche Bindung gewählt. Als Beispiel ist im Ausdruck \verb|\x .x + y| das Symbol \verb|x| als Lambdaparameter interpretiert, während \verb|y| als Zeichenkette erhalten bleibt.
Im Ausdruck \verb|\x .\x .2 - x| kann der Parameter \verb|x| des äußeren Lambdas in der Definition des inneren Lambdas nicht mehr referenziert werden, der Gesamtausdruck ist damit identisch zu \verb|(\x .(\y .2 - y))|.

\subsection{Literale}
Ein Literal kann Lambdafunktionen, Funktionsanwendungen, Komplexe Zahlen und Symbole aus $\Sigma^+$ enthalten. Abbildung \ref{figBspLit} enthält Beispiele.

\begin{figure}
\begin{verbatim}
1 + a

sin(3 pi / 2 t) + 5 cos(pi / 8 t)

map(tup, \x .x^2, tup(1, 2, 3))
\end{verbatim}
\label{figBspLit}
\caption{Beispiele für Literale}
\end{figure}

\subsection{Ersetzungsregeln}
Es gibt zwei Varianten Ersetzungsregeln zu schreiben.
\begin{align*}
	~&~&~&~&\textit{<linke Seite>}&~ ~                       &\verb~=    ~ \textit{<rechte Seite>}&~&~&~&~\\
	~&~&~&~&\textit{<linke Seite>}&~\verb~|~~\textit{<Bedingungen>} &\verb~=    ~ \textit{<rechte Seite>}&~&~&~&~
\end{align*}
In der zweiten Variante ist \textit{<Bedingungen>} eine kommaseparierte Liste von Bedingungen an die vorkommenden Mustervariablen, wie in Abschnitt \ref{subsubsecBedingungen} diskutiert. 
Beispiele sind in Abbildung \ref{figBspRegeln} aufgeführt.

\begin{figure}
\begin{verbatim}
id = \x .x

0 xs... = 0

cos(($k + 1/2) pi) | $k :int = 0

change(_f, _g, _f(xs...)) = _g(xs...)

$a^2 + 2 $a _b + _b^2 = ($a + _b)^2

_aPow2 + _2a _b + _b^2 | 4 _aPow2 == _2a^2 = (1/2 _2a + _b)^2
\end{verbatim}
\label{figBspRegeln}
\caption{Beispiele für Ersetzungsregeln}
\end{figure}
Die beiden letzten Regeln haben den exakt gleichen Effekt, wie noch in Abschnitten \ref{subsubsecWertMustervariable} und \ref{subsubsecBedingungen} erörtert. Während die vorletzte dafür aber die Wert-Mustervariable \verb|$a| einsetzt, bringt die letzte Regel die Mustervariablen \verb|_aPow2| und \verb|_2a| über die angestellte Bedingung in Beziehung zueinander.



%.........................................................................
%.........................................................................
%.........................................................................
\section{Lambdafunktionen} \label{subsecLambdafunktionen}
Die umgesetzten Funktionen erweitern die Definition von Church, indem die selbe Lambdaabstraktion auch mehrere Parameter erlaubt. Während der Ausdruck $\lambda x y . x(y)$ für Church also nur eine vereinfachte Schreibweise\footnote{Die Klammern um $y$ werden in der Literatur oft weggelassen, für diese Arbeit sind sie allerdings notwendig.} für den Ausdruck $\lambda x .\lambda y .x(y)$ darstellt, handelt es sich für die hier beschriebene Umsetzung um zwei verschiedene Funktionen. 
Ein Lambdaparameter ist in der Syntax nicht von anderen Symbolen unterscheidbar\footnote{siehe \ref{subsubsecLambdaSyntax}}, wird intern allerdings durch einen Index dargestellt. Die Darstellung unterscheidet sich von De Bruijn Indizierung \cite{deBruijn} darin, dass jede Instanz der selben Variable immer den selben Index hat\footnote{Sollte eine Ersetzungsregel auch innerhalb eines Lambdas anwendbar sein, ist eine solche Herangehensweise vorteilhaft, da der Matchalgorithmus biser keinen Unterschied zwischen syntaktischer Gleichheit und semantischer Gleichheit macht.}.
 Jeder neu gebundene Parameter bekommt als Index die Anzahl der weiter außen bzw. in der selben Abstraktion vor ihm gebundenen Parameter. Zur Veranschaulichung wird hier der Index im Tiefsatz mitgeschrieben. Wichtig ist hervorzuheben, dass die hier nach wie vor dargestellten Namen ausschließlich der Übersichtlichkeit dienen und in der Umsetzung nicht gespeichert sind.
 Als Beispiel dient $\lambda x_0 y_1 .\lambda z_2 .x_0 + y_1 + z_2$. Das Zeichen $x$ wird in der äußersten Abstraktion zuerst gebunden, hat also keine Vorgänger und dementsprechend Index $0$. In der selben Abstraktion wird als zweiter Parameter weiter $y$ gebunden. Als Nachfolger von $x$ wird $y$ Index $1$ zugewiesen. Die Bindung von $z$ liegt innerhalb der Abbildungsvorschrift des äußeren Lambdas, $x$ und $y$ können also referenziert werden. Dementsprechend hat $z$ zwei Vorgänger und Index $2$. 
 
 Die Auswertung der Funktionsanwendung einer Lambdafunktion $f$ entspricht der $\beta$-Reduktion (von Church \cite{ChurchLambda36} als \emph{operation II} bezeichnet). Parameter von $f$ werden dafür durch die übergebenen Argumente ersetzt. Enthält die Definition von $f$ selbst weitere Lambdaabstraktionen, so werden die Indices derer Parameter um die Stelligkeit von $f$ erniedrigt.
 Der Ausdruck $(\lambda x_0 y_1 .\lambda z_2 .x_0 + y_1 + z_2)(3, 6)$ wird damit zum Ausdruck $\lambda z_0 .3 + 6 + z_0$ ausgewertet.
 
 Die gewählte Indizierung hat ein Problem, welches die De Bruijn Indizierung nicht besitzt. Der Index eines Lambdaparameters ist abhängig vom Kontext der bindenden Abstraktion. Ändert sich dieser Kontext, stimmt die bisherige Indizierung möglicherweise nicht mehr:
 Enthält ein Argument $a$ der Funktionsanwendung einer Lambdafunktion $f$ selbst eine Lambdafunktion $g$, so reicht es für die Auswertung von $f$ mit bisher diskutierten Konzepten nicht aus, den entsprechenden Lambdaparameter von $f$ einfach durch $a$ zu ersetzen. Sollte diese Ersetzung innerhalb einer Lambdaabstraktion $f'$ stattfinden, müssen alle Lambdaparameter von $g$ im Index um die Anzahl der weiter außen gebundenen Parameter erhöht werden. Als Beispiel ist $a = g = \lambda x_0 .x_0$ die Identität, welche der Funktion $f = \lambda x_0 .\lambda y_1 .x_0(y_1)$ übergeben wird. Teil der Definition von $f$ ist $f' = \lambda y_1 .x_0(y_1)$.
 $$(\lambda x_0 .\lambda y_1 .x_0(y_1))(\lambda x_0 .x_0)$$
 Alternativ kann der selbe Audruck dargestellt werden, ohne Information zu benutzen
 An dieser Stelle wird zudem eine alternative Notation eingeführt, die nur Information benutzt, die der Implementierung tatsächlich bekannt ist. Ehrlicher ist die Darstellung einer Lambdaabstraktionen mit $n$ Parametern als $\lambda [n] .\textit{<Definition>}$. Mit den Parametern geschrieben als Prozentzeichen gefolgt von ihrem Index, ist der problematische Ausdruck geschrieben als
 
 $$(\lambda [1] .\lambda [1] .\%0(\%1))(\lambda [1] .\%0)$$.
 Die Naive Ersetzung von $x_0$ in $f$ durch $g$ würde im folgenden Ausdruck resultieren.
 $$\lambda y_0 .(\lambda x_0 .x_0)(y_0)$$
 Ohne Namen ist jetzt klar, dass in der Abbildungsvorschrift von $g$ nun fälschlicherweise auf den Parameter von $f'$ abgebildet wird:
 $$\lambda [1] .(\lambda [1] .\%0)(\%0)$$
 Es gibt verschiedene Möglichkeiten den Fehler zu beheben. Die wohl einfachste Lösung ist Indices von eingesetzten Argumenten in der Auswertung einer Lambdafunktion entsprechend anzupassen. Das würde allerdings bedeuten, dass die Komplexität der Auswertung von der Größe der Argumente abhängig ist. Ein weiterer Nachteil ist, dass veränderte Argumente kopiert werden müssen. Das macht \emph{lazy evaluation} \cite{EvalStrategien}, also die Idee jede Funktionsanwendung höchstens ein Mal auswerten zu müssen, unmöglich.
 
 Eine Alternative ist die Unterscheidung zwischen sogenannten \emph{transparenten} Lambdas und \emph{nicht-transparenten} Lambdas, zweitere dargestellt durch umschließende geschweifte Klammern: $\{\lambda \textit{<Parameter>}.\textit{<Definition>}\}$, bzw. $\{\lambda [n].\textit{<Definition>}\}$. Kommt ein transparentes Lambda $f'$ in der Definition eines Lambdas $f$ vor, wird bei der Auswertung von $f$ auch in $f'$ nach Lambdaparametern gesucht und diese ersetzt, bzw. deren Index erniedrigt. Transparente Lambdas verhalten sich damit nicht anders als die Lambdas bisher. Ist $f'$ allerdings nicht-transparent, lässt die Auswertung von $f$ die Definition von $f'$ unberührt. 
 Mit zwei weiteren Bedingungen wird die Überprüfung von Argumenten bei der Auswertung Lambdas dann obszolet: 
 Zum einen darf ein \emph{außenliegendes} Lambda $f$ nicht transparent sein. Ein Lambda ist außenliegend, wenn es nicht Teil der Definition einem weiteren Lambdas ist, also zu seinen Ahnen keine Lambdaabstraktionen gehören. Zum anderen darf die Funktionsanwendung eines Lambdas nur ausgewertet werden, wenn das Lambda außenliegend ist. Damit wird garantiert, dass während der Auswertung des Lambdas $f$ kein Argument $a$ direkt transparente Lambdas enthält, höchtens als Teil $g'$ (im Beispiel nicht enthalten) eines nicht-transparenten Lambdas $g$.

Im bereits behandelten Problemfall sind $f$ und $g$ damit nicht-transparent. Nur $f'$ ist zu Beginn transparent, wird allerdings nach Auswertung von $f$ ebendfalls zur äußersten Lambdaabstraktion, verliert also seine Transparenz. 
\begin{align*}
    ~           &~\{\lambda x_0 .\lambda y_1 .x_0(y_1)\}(\{\lambda x_0 .x_0\}) \\
    \rightarrow &~\{\lambda y_0 .\{\lambda x_0 .x_0\}(y_0)\}\\
\end{align*}
Alternativ ohne Namen:
\begin{align*}
    ~           &~\{\lambda [1] .\lambda [1] .\%0(\%1)\}(\{\lambda [1] .\%0\}) \\
    \rightarrow &~\{\lambda [1] .(\{\lambda [1] .\%0\})(\%0)\}\\
\end{align*}
Bemerkenswert ist, dass mit den soeben formulierten Bedingungen der Ausdruck nicht weiter transformiert werden darf. Das liegt daran, dass die Funktionsanwendung von $g$ selbst noch vom Lambda $f'$ umschlossen ist. Es ist also nicht ausgeschlossen, dass Argumente der Funktionsanwendung von $g$ (hier nur $\%0$) direkt transparente Lambdas enthalten\footnote{Eng verwandt ist der Grund, warum auch bei der Verwendung von De Bruijn Indices ohne weiteres nur die äußerste Funktionsanwendung abgebildet werden darf. In dem Fall könnte sich sonst die Anzahl der Abstraktionen zwischen einer Variable in einem Argument und ihrer eigenen Abstraktion ändern. Der alte De Bruijn Index würde die Variable dann der falschen Abstraktion zuordnen.}.




%.........................................................................
%.........................................................................
%.........................................................................
\section{Mustervariablen} \label{subsecMustervariablen}
In der Umsetzung wird zwischen drei Arten von Mustervariablen unterschieden. Die Sonderform der Multi-Mustervariable ist bereits aus Kapitel \ref{subsecMulti} bekannt, die Wert-Mustervariable ist allerdings neu. Als drittes ist die normale Mustervariable nach wie vor vorhanden. Wert-Mustervariable und normale Mustervariable können zudem durch Bedingungen eingeschränkt werden.

\subsection{Wert-Mustervariable} \label{subsubsecWertMustervariable}
Nach Definition \ref{defMatch} ist das Match $v_p$ für ein Paar $(p, t)$ aus Muster $p$ und Literal $t$ dann gültig, wenn $\Const{lit}(p, v_p) = t$ gilt. In Kapitel \ref{secPattermatching} nicht diskutiert wurde die Möglichkeit Zahlen mit Rechenausdrücken zu matchen. Ein Muster $p = (\texttt{prod}, 2, \mathbf k)$ würde für das Literal $t = 12$ etwa ein Match $v_p$ mit $v_p~\mathbf k = 6$ besitzen. Um die Matchsuche nicht zu kompliziert zu gestalten, besitzt die Umsetzung zwei Einschränkungen. Zum einen ist eine solche Dekonstruktion des Musters zu einem Wert nur dann möglich, wenn das entsprechende Teilmuster exakt eine Instanz $w$ einer Wert-Mustervariable enthält. Zum anderen werden nur bijektive Funktionen\footnote{bzw. sonst Mehrdeutigkeiten ignoriert} dekonstruiert, wie im Beispiel die Multiplikation mit $2$. 


Vor einem Matchversuch muss ein Muster mit Wert-Mustervariablen zuerst entsprechend transformiert werden. Ziel der Transformation ist, das für den Matchalgorithmus direkt klar ist, ab wo das Muster einen Wert repräsentieren soll. Der Teilbaum des Musters, der eine Wert-Mustervariable $w$ enthält und als ganzes einen Wert matchen soll, ist idealerweise also direkt gekennzeichnet. In der Umsetzung ist das realisiert, indem eine Wert-Mustervariable im zu matchenden Muster kein Blatt darstellt, sondern einen Teilbaum markiert. Dieser entspricht allerdings nicht der ursprünglichen Umgebung um die Wert-Mustervariable, sondern der Inversen. 

\begin{figure}
\begin{verbatim}
value_match__(_i, _dom, _match) + _a + cs... | _a :complex 
    = value_match__(_i, _dom, _match - _a) + cs...
        
value_match__(_i, _dom, _match) * _a * cs... | _a :complex 
    = value_match__(_i, _dom, _match / _a) * cs...
        
value_match__(_i, _dom, _match) ^ 2                        
    = value_match__(_i, _dom, sqrt(_match))
        
sqrt(value_match__(_i, _dom, _match))                           
    = value_match__(_i, _dom, _match ^ 2)
\end{verbatim}
\label{figWertMusterBau}
\caption{Transformationsregeln von Mustern mit Wert-Mustervariable}
\end{figure}

Die Transformation in diese Form erfolgt über Anwendung von Ersetzungsregeln. Die Wert-Mustervariable wird dazu temporär als Funktionsanwendung des Symbols \verb|value_match__| gespeichert. Die Regeln sind - jeweils zweizeilig -  in Abbildung \ref{figWertMusterBau} aufgeführt. Das Funktionssymbol erwartet drei Argumente, hier die Mustervariablen \verb|_i| für den Identifikationsindex, \verb|_dom| für den erlauben Raum (kurz für \emph{domain}) und \verb|_match| für den Teilterm, der im Matchalgorithmus den Wert bestimmt. 


Die finale Repräsentation einer Wert-Mustervariablen entspricht in den gespeicherten Informationen fast exakt der Zwischenform als Funktionsanwendung von \verb|value_match__|. Ergänzt wird einzig, ob die jeweilige Instanz bindend oder gebunden ist.

Im Match der Wert-Mustervariablen $w$ mit einer komplexen Zahl $z$ als Literal wird der in Abbildung \ref{figWertMusterBau} \verb|_match| genannte Teilbaum genutzt, um den Wert, den die Wert-Mustervariable an ihrem ursprünglichen Ort im Muster für ein Match besitzen müsste zu bestimmen. Dafür enthält die Startkonfiguration der Wert-Mustervariable als Funktionsanwendung von \verb|value_match__| als drittes Argument das besondere Symbol \verb|value_proxy__|. Dieses wird auch nach Transformation des Musters exakt ein Mal in \verb|_match| vorkommen und kann als Platzhalter für das zu matchende Literal $z$ genutzt werden. Die Auswertung von \verb|_match| mit $z$ an Stelle von \verb|value_proxy__| gibt damit den für $w$ gesuchten Wert zurück. Je nachdem, ob $w$ eine bindende oder eine gebundene Instanz ist, wird der berechnete Wert im Matchzustand\footnote{siehe Abschnitt \ref{subsubsecMatchalgoCpp}} gespeichert oder mit dem bereits gespeicherten Wert verglichen.


\subsection{Bedingungen} \label{subsubsecBedingungen}

Muster können durch Bedingungen an normale Mustervariablen in möglichen Matches weiter eingeschränkt werden.
Diese Bedingungen übernehmen teilweise eine ähnliche Rolle wie die Wert-Mustervariablen\footnote{siehe Abschnitt \ref{subsubsecCos}}, sind dabei allerdings etwas flexibler. 


\subsubsection{Mögliche Ausdrücke}
Bedingungen, die nicht mit Wert-Mustervariablen dargestellt werden könnten, nutzen primär die Funktionssymbole \verb|contains|, \verb|neq| und \verb|of_type__|. Ersteres ist binär und prüft, ob das erste Argument Teilbaum des zweiten Argumentes ist\footnote{sie Abschnitt \ref{subsubsecDiff}}. Zweiteres (normal in Infixform \verb|!=| geschrieben) ist ebenfalls binär und prüft, ob zwei Teilbäume ungleich sind. Beide dürfen dabei allerdings ausschließlich direkt Mustervariablen als Argumente übergeben bekommen. 

Das Symbol \verb|of_type__| ist auch binär und wird normal nicht direkt hingeschrieben, sondern tritt als Infixoperator \verb|:| auf. Die Bedeutung variiert dann nach dem rechten Argument. Ist dies ein Knotentyp, etwa \verb|complex| oder \verb|f_app|, prüft \verb|of_type__|, ob das linke Argument von diesem Typ ist. Eine spezifischere Variante der Bedingung \verb|_x :f_app| ist {\verb|_x :|$f$}, mit einem Symbol $f$. In dem Fall wird geprüft, ob die linke Seite eine Funktionsanwendung von $f$ ist. Als Beispiel muss mit der Bedingung \verb|_x :sum| die Mustervariable \verb|_x| mit einer Summe, also einer Funktionsanwendung des Symbols \verb|sum|, gematcht sein, damit das Match gültig ist.
Der dritte Spezialfall tritt auf, wenn die rechte Seite eine fest definierte Teilmenge der Komplexen Zahlen ist, etwa \verb|_x :int|. Die linke Seite muss dann nicht direkt eine Mustervariable sein, sondern lediglich zu einer Komplexen Zahl ausgewertet werden können\footnote{siehe Abschnitt \ref{subsubsecCos}}.

\subsubsection{Umsetzung}
Eine Bedingung die mehrere Mustervariablen enthält, wird der Mustervariablen hinzugefügt, deren bindende Instanz auf der linken Regelseite als letztes vorkommt. Ist eine solche Mustervariable dann von mehreren Bedingungen abhängig, werden diese als Argumente des variadischen Funktionssymbols \verb|and| zusammengefasst. Die Überprüfung einer Bedingung passiert, wenn Algorithmus $\Const{findMatch}$ die entsprechende Mustervariableninstanz als Argument erhält. 





%.........................................................................
%.........................................................................
%.........................................................................
\section{Datenstruktur}
Die Knoten des Termbaumes liegen in einem Array, dem sogenannten \verb|Store|. Während eine Komplexe Zahl oder ein Lambdaparameter immer ein Arrayelement besetzen, kann die Funktionsanwendung auch mehrere direkt hintereinanderliegende Arrayelemente nutzen, je nachdem, wie viele Argumente referenziert werden müssen. Referenziert wird ein Knoten aus einem Paar aus Arrayindex und Knotentyp, ebendfalls dargestellt als natürliche Zahl. Dieses Paar wird \verb|NodeIndex| genannt. Der Knotentyp wird als \verb|NodeType| bezeichnet und kann 12 verschiedene Werte annehmen, eng verwandt mit Definition \ref{defKnotentypenMathe}:
\begin{enumerate}
	\setcounter{enumi}{-1} %beginne aufzählung bei 0
	\item {$\Sigma^+$.}
	\item {Komplexe Fließkommazahl\footnote{Die Nutzung von Fließkommazahlen ist für Computeralgebra nur bedingt geeignet, erlaubt aber die Verwendung der im C\texttt{++} Standard geforderten Rechenoperationen für Komplexe Zahlen. Über Auslesen der entsprechenden CPU-Flags kann dennoch garantiert werden, dass ausschließlich exakte Vereinfachungen durchgeführt werden.}, zusammengesetzt aus zwei 64-bit Fließkommazahlen für Realteil und Imaginärteil}
	\item {$\Lambda$}
	\item {$V$}
	\item {Funktionsanwendung in einem Literal}
	\item {Funktionsanwendung in einem Muster}
	\item {Hilfsyp genutzt in der Transformation eines Musters mit Wert-Mustervariablen}
	\item {bindende Instanz einer Mustervariable $\mathbf x \in X$ an die im Muster Bedingungen geknüpft sind}
	\item {bindene Instanz einer Mustervariable $\mathbf x \in X$ ohne weitere Bedingungen}
	\item {gebundene Instanz einer Mustervariable $\mathbf x \in X$}
	\item {$X^*$}
	\item {$W$}
\end{enumerate}
Manche der hier aufgezählten Knoten müssen neben ihrem Typen nur eine einzige natürliche Zahl kennen. Bereits diekutiert wurde das für Knoten von Typ $3$, den Lambdaparametern. Für ein Knoten dieser Art ist es nicht notwendig, die Information als eigenens Arrayelement des \verb|Store| zu speichern, der Index des Paares aus Index und Typ von \verb|NodeType| enthält direkt den erforderlichen Wert. Damit werden die entsprechenden Argumente ohne Indirektion direkt in einer Funktionsanwendung gespeichert. Knoten dieser Art sind Typ 0, 3, 6, 8 und 9. Für Symbole aus $\Sigma^+$ wird nicht der Name selbst gespeichert, sondern der Index in einem zentralen Namensspeicher. Nicht optimal ist damit, dass die Ordnung zweier Symbole aus $\Sigma^+$ zueinander von der Reihenfolge der Auflistung im Namensspeicher und damit von der Eingabereihenfolge abhängen, was aber nur ästhetische Nachteile birgt. Der Zugriff auf den Namensspeicher wird so auf die Eingabe und Ausgabe eines Terms beschränkt.


Alle im \verb|Store| mögliche Belegungen eines Feldes sind der der \verb|union| \verb|TermNode| aus Quelltext \ref{codeTermNode} zusammengefasst. Der Typ \verb|Complex| ist dabei nur ein Alias für \verb|std::complex<double>|. Interessant ist der Typ \verb|FApp|, welcher die Funktionsanwendung repräsentiert. Intern ist dieser Typ erneut polymorph, da das erste Arrayfeld einer Funktionsanwendung nicht nur die ersten \verb|NodeIndex| Referenzen auf Funktion und Argumente enthält, sondern auch die Information, wie viele Arrayfelder von der Funktionsanwendung besetzt sind und wie viele Argumente insgesamt gehalten werden. Folgende Felder im \verb|Store| halten ausschließlich weitere Argumente.


\begin{listing}
\footnotesize
\begin{minted}[xleftmargin=2em, linenos, breaklines]{cpp}
union TermNode 
{
    //nodes valid everywhere:
    Complex complex;
    FApp f_app;
    Lambda lambda;
    
    //nodes only expected in a pattern:
    RestrictedSingleMatch single_match;
    MultiMatch multi_match; //only expected in right hand side
    ValueMatch value_match;
    FAppInfo f_app_info; //metadata for PatternFApp
};
\end{minted}
\label{codeTermNode}
\caption{mögliche Einträge eines Feldes im Speicher}
\end{listing}



\begin{beispiel}~\\ \label{bspREPLliteral}
Das Literal \verb|tup(f(a, b)(c), 1, 2, 3)| wird in der folgenden Debugausgabe genutzt, um den Speicheraufbau zu veranschaulichen.
\begin{verbatim}
  head at index: 5
   0 | application:     { 121,  92,  93 }    f(a, b)
   1 | application:     {   0,  94 }         f(a, b)(c)
   2 | value      :                          1
   3 | value      :                          2
   4 | value      :                          3
   5 | application:     {  78,   1,   2,     tup(f(a, b)(c), 1, 2, 3)
   6 |   ...           3,   4 }
\end{verbatim}
Die linke Spalte gibt den Arrayindex an. Nach der Art des gespeicherten Knotens folgt dann für Funktionsanwendungen die Auflistung der Funktion und Argumente. Ganz rechts ist der Teilbaum beginnend an dem entsprechenden Arrayindex angegeben. Bemerkenswert ist hierbei, dass die Symbole \verb|f|, \verb|tup|, \verb|a|, \verb|b| und \verb|c| als solche nicht im Array gespeichert sind. Die Funktionsanwendung von \verb|f| gespeichert an Index 0 verrät allerdings, dass das Programm vor dem Parsen von Symbol \verb|f| bereits 121 andere unbekannte Symbole eingelesen hat. Analog ist \verb|a| als Index 92, \verb|b| als Index 93 und \verb|c| als Index 94 gespeichert. 
Interessant ist weiter, dass die Funktionsanwendung des Symbols \verb|tup| beginnend an Arrayindex 5 auch das darauffolgende Arrayelement belegt. Bis auf den Index von \verb|tup|, beziehen sich die restlichen Indices dieser Funktionsanwendung tatsächlich auf echte Elemente im Array. 
\end{beispiel}


\begin{beispiel}~\\ \label{bspREPLlambda}
Die in Abschnitt \ref{subsecLambdafunktionen} genutzte Notation für Lambdas mit namenlosen Lambdaparametern wird auch in der Umsetzung zur Darstellung von Lambdafunktionen genutzt. Als Bespiel wird die Speichernutzung des Literals \verb|(\f x .f(x, x, x))(sum, 3)| gezeigt.
\begin{verbatim}
  head at index: 4
   0 | application:     {   0,   1,   1,     %0(%1, %1, %1)
   1 |   ...           1 }
   2 | lambda     :                          {\[2].%0(%1, %1, %1)}
   3 | value      :                          3
   4 | application:     {   2,  81,   3 }    {\[2].%0(%1, %1, %1)}(sum, 3)
\end{verbatim}
Nach Auswertung von $\Const{normalize}$ ändert sich die Speicherbelegung.
\begin{verbatim}
  head at index: 0
   0 | value      :                          9
   1 | -----free slot-----
...
   6 | -----free slot-----
\end{verbatim}
Zuerst wurden die Argumente \verb|sum| und \verb|3| in eine Kopie der Lambdadefinition eingesetzt, dann wurde die Funktionsanwendung von \verb|sum| zur Komplexen Zahl \verb|9| ausgewertet. Diese kann in einem einzelnen Arrayelement gespeichert werden, der restliche Platz ist jetzt ungenutzt.
\end{beispiel}





%.........................................................................
%.........................................................................
%.........................................................................
\section{Algorithmen} \label{subsecCppAlgos}

Die beschriebene Datenstruktur wird in den meisten Fällen rekursiv abgelaufen. Der einzelne \verb|NodeIndex|, bietet aber nicht genug Information, um zu wissen, wo der referenzierte Knoten gespeichert ist. Das Array des \verb|Store| muss adressierbar sein. Für diesen Zweck gibt es Hilfstypen, die eine Referenz auf den \verb|Store|, bzw. das unterliegende Array mit dem \verb|NodeIndex| bündeln. Die mutierbare Variante wird als \verb|MutRef| bezeichnet. Soll nur lesen des Terms erlaubt sein, gibt es die Hilfstypen \verb|Ref| und \verb|UnsaveRef|. 

\begin{listing}
\footnotesize
\begin{minted}[xleftmargin=2em, linenos, breaklines]{cpp}
void free_tree(const MutRef ref)
{
    if (!is_stored_node(ref.type) || ref.store->decr_at(ref.index) != 0) { 
        return; 
    }
    switch (ref.type) {
    case NodeType(Literal::complex):
    case NodeType(SpecialMatch::multi):
        break;
    case NodeType(Literal::lambda):
        free_tree(ref.at(ref->lambda.definition));
        break;
    case NodeType(Literal::f_app):
        for (const NodeIndex subtree : ref) {
            free_tree(ref.at(subtree));
        }
        FApp::free(*ref.store, ref.index);
        return;
    case NodeType(SingleMatch::restricted_):
        free_tree(ref.at(ref->single_match.condition));
        break;
    case NodeType(SpecialMatch::value):
        free_tree(ref.at(ref->value_match.inverse));
        break;
    }
    ref.store->free_one(ref.index);
} //free_tree
\end{minted}
\label{codeFreeTree}
\caption{Speicherfreigabe eines Teilterms}
\end{listing}

Illustriert wird die Vorgehensweise des Ablaufens des Terms als Baum mit polymorphen Knoten anhand der Funktion \verb|free_tree|, dargestellt als Quelltext \ref{codeFreeTree}, welche im \verb|Store| die von dem übergebenden Teilbaum \verb|ref| belegten Arrayfelder freigibt, sofern in Zeile 3 nicht herausgefunden wird, dass der Teilbaum noch von anderer Stelle referenziert wird. Die Felder \verb|ref.type| und \verb|ref.index| von \verb|MutRef| sind als Daten von \verb|NodeIndex| zu erkennen. Die Methode \verb|ref.at(|$x$\verb|)| konstruiert eine Referenz zu einem neuen Teilterm $x$, übergeben als \verb|NodeIndex|.

\subsection{Eingebaute Auswertung und Normalisierung} \label{subsubsecAuswertungNormalCpp}

Die \glqq natürliche\grqq{} Auswertung der Funktionsanwendungen eingebauter Funktionssymbole, etwa \verb|sum|, \verb|prod|, \verb|sin|, etc. zu Komplexen Zahlen ist in der Umsetzung fest eingebaut und ausschließlich als Teil des ebenfalls fest eingebauten $\Const{normalize}$ Algorithmus vorhanden. Dieser unterscheidet sich insofern von Kapitel \ref{secErsteNormalform}, als dass alle beschriebenen Normalisierungsschritte in der nicht rekursiven Funktion \verb|normalize::outermost| implementiert sind, welche ausschließlich eine außenliegende Funktionsanwendung ändert. Das erlaubt in der Ersetzung eines Teilterms durch ein Muster ausschließlich die neuen Teile des Literals auf mögliche Normalisierungen zu prüfen.

\begin{listing}
\footnotesize
\begin{minted}[xleftmargin=2em, linenos, breaklines]{cpp}
NodeIndex normalize::recursive(const MutRef ref, const Options options)
{
    assert(ref.type != PatternFApp{});
    if (ref.type == Literal::f_app) {
        auto iter = begin(ref);
        *iter = normalize::recursive(ref.at(*iter), options);
        if (nv::is_lazy(*iter)) [[unlikely]] {
            const NodeIndex result = normalize::outermost(ref, options).res;
            return normalize::recursive(ref.at(result), options);
        }
        else {
            for (++iter; !iter.at_end(); ++iter) {
                *iter = normalize::recursive(ref.at(*iter), options);
            }
        }
    }
    return normalize::outermost(ref, options).res;
} //recursive
\end{minted}
\label{codeNormalizeRecursive}
\caption{rekursive Normalisierung eines Literals}
\end{listing}

Erst die Funktion \verb|normalize::recursive|, abgebildet als Quelltext \ref{codeNormalizeRecursive}, implementiert dann die Auswertungsstrategie für den Gesamtterm: Zuerst wird in Zeile 6 das Funktionssymbol berechnet, was in der aktuellen Funktionsanwendung angewendet wird. Möglich ist, dass ein Funktionssymbol faul ist\footnote{Das ist aktuell ausschließlich für \texttt{true} und \texttt{false} als binäre Funktionen der Fall: \texttt{true} bildet auf das erste Argument ab, \texttt{false} auf das zweite.}, also vor seinen Argumenten normalisiert wird. 
Sonst erfolgt in Zeile 13 der Rekursionsaufruf für jedes Argument, bevor zum Schluss in Zeile 17 der gesamte Term normalisiert wird.



\subsection{Matchalgorithmen} \label{subsubsecMatchalgoCpp}

Die große Lücke bei der Diskussion des Matchalgorithmus in Kapitel \ref{secPattermatching} ist die Datenstruktur, die nicht nur das finale Match enthalten muss, sondern auch, welches Musterargument gerade mit welchem Literal gematcht ist und das für jede Funktionsanwendung des Musters. Diese Datenstruktur wird als C\texttt{++} Referenz\footnote{äquivalent zu einem Zeiger, nur bereits dereferenziert} mit dem Name \verb|match_state| übergeben, wie in Quelltext \ref{codeRematchDecl} gezeigt. 

\begin{listing}
\footnotesize
\begin{minted}[xleftmargin=2em, linenos, breaklines]{cpp}
bool matches(const UnsaveRef pn_ref, const UnsaveRef ref, match::State& match_state);

bool rematch(const UnsaveRef pn_ref, const UnsaveRef ref, match::State& match_state);
\end{minted}
\label{codeRematchDecl}
\caption{Funktionsdelarationen der Algorithmen \ref{findMatch} und \ref{rematch}}
\end{listing}

Der Typ \verb|match::State| enthält primär drei Arrays. Im Array \verb|.single_vars| sind die Referenzen der Teilterme des Literals, mit dem die bindenden Instanzen der normalen Mustervariablen aktuell assoziiert sind als \verb|NodeIndex| gespeichert.
Im Array \verb|.value_vars| ist für jede Wert-Mustervariable der aktuell assoziierte Wert gespeichert. Diese beiden Arrays implementieren damit fast exakt das Match $v_p$. Die Indices der Mustervariablen werden in den Arrays auf die entsprechend assoziierten Literale abgebildet.
Nicht ausschließlich als Teil $v_p$ zu verstehen, ist der Teil von \verb|match::State|, der die Matchkonfiguration jeder Funktionsanwendung des Musters mit kommutativem Funktionssymbol oder mit Multi-Mustervariablen unter den Argumenten festhält. Das Array \verb|.f_app_entries| hält diese Konfigurationen mit einem Element für jede darauf zurückgreifende Funktionsanwendung. Ein solches Element besteht selbst aus einem Array mit einem Eintrag für jedes Argument der Funktionsanwendung, der speichert, mit dem Literal an welchem Index $k$ aktuell das Match besteht. Eine sekundäre Funktion besitzt \verb|.f_app_entries| für Multi-Mustervariablen. Die Argumente des Literals, die einer entsprechenden Mustervariablen zugeordnet werden, können hieraus rekonstruiert werden, da die Multi-Mustervariablen genau die Stellen besetzen, die mit keinem normalen Argument gematcht sind. Das erlaubt der Implementierung die Multi-Mustervariablen auf der linken Seite eines Musters exakt so zu nutzen, wie im Pseudocode angedeutet, also als einfache Logikwerte in einem Array bei jeder Funktionsanwendung. 




%.........................................................................
%.........................................................................
%.........................................................................
\section{Anwendung} \label{subsecCppAnwendung}

\subsection{Ableiten} \label{subsubsecDiff}

Wenn auch Ableitungsregeln nicht die Möglichkeiten des assoziativen und kommutativen Matches ausnutzen, bieten sie eine gute erste Übersicht der Nutzung eines Termersetzungssystems.

\begin{verbatim}
diff(_x, _x)                     = 1
diff(_a, _x) | !contains(_a, _x) = 0
diff(_g^_h, _x) = (diff(_h, _x) ln(_g) + _h diff(_g, _x) / _g) _g^_h
diff(_a, _x) | _a :sum = map(sum, \f .diff(f, _x), _a)
diff(_u vs..., _x) = diff(_u, _x) vs... + _u diff(prod(vs...), _x)
diff(_f(_y), _x) = diff(_y, _x) fdiff(_f)(_y)

fdiff(\x ._y) = \x .diff(_y, x)
fdiff(sin)     = cos
fdiff(cos)     = \x .-sin(x)
fdiff(exp)     = exp
fdiff(ln)      = \x .x^(-1)
fdiff(tan)     = \x .cos(x)^(-2)
...
\end{verbatim}

Die linken Seiten sind entweder Funktionsanwendungen des binären Funktionssymbols \verb|diff| oder des unären Funktionssymbols \verb|fdiff|. Während \verb|diff| einen beliebigen Term als erstes Argument erwartet, gefolgt von dem Symbol, nach dem abgeleitet werden soll, Leitet \verb|fdiff| unäre Funktionen ab. Das vereinfacht die Syntax zur Darstellung der Kettenregel, welche als letzte Regel für \verb|diff|  vor der Leerzeile umgesetzt ist. Interessant ist bei dem entsprechenden Muster \verb|diff(_f(_y), _x)|, dass die Funktion, deren Anwendung abgeleitet wird, repräsentiert durch die Mustervariable \verb|_f|, selbst eine Unbekannte ist. Die rechte Seite \verb|diff(_y, _x) fdiff(_f)(_y)| implementiert dann das \glqq innere mal äußere Ableitung \glqq{} Muster. Da \verb|fdiff(_f)| eine Funktion zurückgibt, kann auf diese Funktion dann das Argument \verb|_y| angewendet werden.

Im folgenden wird die rechte Seite der vierten Regel diskutiert.

\verb~    diff(_a, _x) | _a :sum = map(sum, \f .diff(f, _x), _a)~

Das Funktionssymbol \verb|map| hat eine fest in $\Const{normalize}$ eingebaute Auswertung: Sollte das dritte Argument eine Funktionsanwendung des ersten Argumentes $g$ sein, wird wieder eine Funktionsanwendung von $g$ zurückgegeben, allerdings mit dem zweiten Argument von \verb|map| angewendet auf jedes Argument von $g$. In Mustern geschrieben, sähe die Abbildungsvorschrift von \verb|map| wie folgt aus:

\begin{verbatim}
map(_g, _f, _g(xs...)) 
    = map_helper(_g(), _g(xs...), _f)

map_helper(_g(xs...), _g(_y, ys...), _f) 
    = map_helper(_g(xs..., _f(_y)), _g(ys...), _f)

map_helper(_g_app, _g(), _f) 
    = _g_app
\end{verbatim}

In der konkreten Anwendung im Muster wird also die Ableitung einer Summe umgeschrieben als die Summe der Ableitungen der Argumente. Das Muster

\verb|diff(_u + vs..., _x) = diff(_u, _x) + diff(sum(vs...), _x)|

hätte in wiederholter Anwendung den selben Effekt, die genutze Version benötigt allerdings nur eine einzige Ersetzung\footnote{sofern \texttt{map} in $\Const{normalize}$ ausgewertet wird} für eine Summe mit beliebig vielen Argumenten.

\begin{beispiel}~\\
Angewendet auf das Literal \verb|diff(sin(pi t) + 3, t)|, erzeugen die Ableitungsregeln folgende Zwischenergebnisse, bzw. folgende Normalform (Leerzeichen sind zur verbesserten Lesbarkeit ergänzt).
\begin{verbatim}

diff(3, t) + diff(sin(pi t), t)
0          + diff(sin(pi t), t)
             diff(pi t, t)                   fdiff(sin)(pi t)
             (pi diff(t, t) + t diff(pi, t)) fdiff(sin)(pi t)
             (pi diff(t, t) + t diff(pi, t))        cos(pi t)
             (pi 1          + t diff(pi, t))        cos(pi t)
             (pi            + t 0          )        cos(pi t)
             
pi cos(pi t)
\end{verbatim}
\end{beispiel}

\subsection{Exakte Auswertung trigonometrischer Funktionen} \label{subsubsecCos}
Die Regeln zur Erkennung von Nullstellen und Extrempunkte der Kosinus-Funktion lassen sich sehr kompakt mit Wert-Mustervariablen beschreiben: 

\begin{verbatim}
cos(             pi)           = -1
cos(($k + 0.5)   pi) | $k :int =  0
cos((2 $k)       pi) | $k :int =  1
cos((2 $k + 1)   pi) | $k :int = -1
\end{verbatim}

Eine alternative Regelmenge mit selbem Effekt nutzt Bedingungen.

\begin{verbatim}
cos(   pi)                     = -1
cos(_a pi) | _a + 1/2     :int =  0
cos(_a pi) | _a / 2       :int =  1
cos(_a pi) | (_a - 1) / 2 :int = -1
\end{verbatim}

Bemerkenswert ist, dass in beiden Regelmengen dem System kein numerischer Wert für das Symbol \verb|pi| genannt wird.

\subsection{Faktorisieren}
Schwieriger in der Frage, ob jede Regelanwendung zu einer Vereinfachung führt, ist das Faktorisieren einer Summe aus Produkten. Wer dieser Frage unkritisch gegenübersteht, kann die folgenden Ersetzungsregeln nutzen.

\begin{verbatim}
_a^2 + 2 _a _b + _b^2 = (_a + _b)^2
_a^2 - 2 _a _b + _b^2 = (_a - _b)^2
$a^2 + 2 $a _b + _b^2 = ($a + _b)^2
$a^2 - 2 $a _b + _b^2 = ($a - _b)^2

_a bs... + _a cs... = _a (prod(bs...) + prod(cs...))
_a bs... + _a       = _a (prod(bs...) + 1)
_a       + _a       = 2 _a
\end{verbatim}

Leider nicht direkt durch eine endliche Menge von Mustern abbildbar ist der binomische Lehrsatz. Die ersten beiden binomischen Formeln sind aber durch die vier Regeln im ersten Block vollständig beschrieben. Die unteren beiden Regeln des Blocks erkennen die Formeln auch dann, wenn \verb|$a| nicht direkt vorliegt, sondern als Wert bereits mit seiner Umgebung kombiniert wurde. Anwendbar ist die dritte Regel etwa auf das Literal \verb|81 + 18 sin(x) + sin(x)^2| mit $v_p$ \verb|$a| $= 9$.

Die dargestellten Regeln müssen deshalb kritisch betrachtet werden, weil sie im höchsten Grade nicht konfluent sind, wie bereits in Abschnitt \ref{subsubsecKonfluenz} diskutiert. Eine einfache Ersetzung des ersten gefundenen Musters kann deshalb möglicherweise nicht immer zum bestmöglichen Ergebnis führen.




















\chapter{Fazit} \label{secZusammenfassung}
Terme können als Baumstrukturen verstanden werden. Die Transformation eines Terms wird dementsprechend als Transformation eines Baumes umgesetzt. In dieser Arbeit erfolgt die Transformation über Ersetzungsregeln, die selbst als Paare von Bäumen dargestellt werden, die Bäume werden dabei als Muster bezeichnet. Mustervariablen fungieren als Platzhalter in einer Ersetzungsregel.

Es ist schwierig, effiziente Algorithmen zu finden, die eine große Teilmenge aller Muster mit assoziativen und kommutativen Funktionssymbolen erkennen. Die allgemeine Lösung bietet sich nur für wenige Probleme an, da kein deterministischer Algorithmus mit polynomieller Laufzeit bekannt ist. 
Die beschriebenen Algorithmen nutzen Backtracking, um unterschiedliche Matchmöglichkeiten der Teilmuster zu testen. 

Als teilweise Alternative zur Berücksichtigung von Assoziativität bei der Matchsuche bieten sich Konstrukte an, die für mehrere Argumente einer Funktionsanwendung gleichzeitig stehen können, hier als Multi-Mustervariablen bezeichnet. Vollständigen Ersatz bieten sie jedoch nicht.
Bestimmte kommutative Muster können in linearer Zeit in einem Literal erkannt werden, auch wenn gleiche Mustervariablen mehrfach auftreten. 

Die Zielsetzung Ausdrücke über den Komplexen Zahlen zu vereinfachen hat sich als sehr ambitioniert herausgestellt. Das implementierte Termersetzungssystem erlaubt einfache und schnelle Anpassung einer Normalisierungsstrategie. Vor allem aber die häufig fehlende Konfluenz der gewählten Regelmengen machen es schwierig die Vereinfachung exakt zu kontrollieren. 
Die Implementierung von Wert-Mustervariablen erlaubt in einigen Fällen Muster zu definieren, die sehr nah an gewohnten mathematischen Schreibweisen liegen. In ihren Möglichkeiten sind sie gegenüber dem implementierten Bedingungssystem jedoch stärker eingeschränkt. In beiden Fällen können aber fast ausschließlich Bedingungen an die Wertebereiche Komplexer Zahlen gestellt werden.

\section{Ausblick}
Drei Gebiete sind für den Autor besonders interessant. Zum einen sind die Möglichkeiten bestimmte Muster schneller zu finden noch nicht ausgeschöpft. Weiter wird die fehlende Konfluenz der genutzten Regelmengen in der Umsetzung noch wenig berücksichtigt. Denkbar wäre für die Vereinfachung recht kleiner Terme etwa die gleichzeitige Anwendung aller möglichen Ersetzungsregeln mit anschließender Auswahl der besten Normalform. Um in diesem Fall die Größe des Literals / der Literale und damit auch den Ersetzungsaufwand kontrollieren zu können, müssten möglichst viele gleiche Abschnitte von mehreren Versionen des Literals geteilt werden. 

In dieser Arbeit nicht besprochen ist die Kompilierung von Mustern zu entsprechenden Matchfunktionen. Die Schwierigkeit steigt hierbei allerdings mit der Ausdruckskraft der Muster.








\printglossaries

\printbibliography

\end{document}








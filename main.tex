

\documentclass{scrartcl}
\usepackage[utf8]{inputenc}

\title{Entwicklung eines Termersetzungssystems für assoziative und kommutative Ausdrücke\\ \textit{Version 0.0.4}}
\author{Bruno Borchardt}
\date{\today}

\usepackage{csquotes}
\usepackage[ngerman]{babel}

\usepackage{graphicx}
\usepackage{mathtools}
\usepackage{amssymb}
\usepackage{amsmath} %  \begin{cases}  Wert 1 & Bedingung 1 \\ Wert 2 & Bedingung 2 \\ \end{cases} 
\numberwithin{figure}{section} %label nr. beinhaltet section

\usepackage{qtree}
\usepackage[linesnumbered, german, ruled, vlined]{algorithm2e}

\usepackage[backend=biber, style=alphabetic]{biblatex}
\addbibresource{quellen.bib}
%\bibliographystyle{plain}
%\bibliography{references}

\usepackage{xcolor} %\textcolor{blue}{This is a sample text in blue.}

\usepackage{minted}
\renewcommand{\listingscaption}{Quelltext}
\usemintedstyle{friendly}

\setlength{\parindent}{0pt} %keine Einrückung nach absatz

\usepackage{amsthm}
\theoremstyle{definition} %keine kursiven theoreme

%.........................................................................
%................................ Macros .................................
%.........................................................................

% baut teile eines tupels: "t_1, ..., t_n"
\newcommand{\elems}[3]{{#1}_{#2}, \dots, {#1}_{#3}}
\newcommand{\tOneN}{\elems t 1 n}

% stapelt zweiten parameter auf ersten
\newcommand{\stapel}[2]{\mathrel{\overset{\makebox[0pt]{\mbox{\normalfont\tiny\sffamily {#2}}}}{#1}}}

%gibt text in fett und rot aus
\newcommand{\BFred}[1]{\textbf{\textcolor{red}{#1}}}

%in algorithm folgt nach "if" kein "then", nach "for" kein "do" ...
\SetKwIF{If}{ElseIf}{Else}{if}{ }{else if}{else}{}
\SetKwFor{For}{for}{}{}
\SetKwFor{While}{while}{}{}

%\paren*{a + b} skaliert automatisch klammern um a + b
\DeclarePairedDelimiter\paren{(}{)} 
\DeclarePairedDelimiter\curl{\{}{\}}

%.........................................................................
%................................ Body ...................................
%.........................................................................
\begin{document}

\maketitle

\tableofcontents

\clearpage
%\cleardoublepage <- für andere Dokumenttypen



\section{Einleitung} \label{secEinleitung}
\textcolor{red} {
\begin{itshape}
Anmerkung: Hier kommt hin, was halt in eine Einleitung soll:
\begin{itemize}
    \item was ist ein Termersetzungssystem?
    \item Einsatz von Termersetzungssystemen
    \begin{itemize}
        \item Beweisprüfer /Beweisassistenten
        \item Arithmetikausdrücke vereinfachen
        \item Optimierender Compiler
        \item Interpreter funktionaler Sprachen
        \item bestimmt noch mehr
    \end{itemize}
\end{itemize}
\end{itshape}
}

Die Idee Computer zu nutzen um symbolische Ausdrücke zu manipulieren ist fast so alt wie der Computer selbst.  LISP ist als eine der ersten höheren Programmiersprachen bereits für diesen Zweck geschaffen worden \cite{lisp}. 

\subsection{Zielsetzung}
Ziel der Arbeit ist Design und Umsetzung eines Termersetzungssystems zur Vereinfachung algebraischer Ausdrücke. Der Kern des Termersetzungssystems ist ein Algorithmus zur Erkennung eines bestimmten Musters in einem Term. 
Der Algorithmus soll ein Muster dabei möglichst nicht nur erkennen, wenn der Term die exakt identische Struktur aufweist. Bestimmte Äquivalenzklassen, wie etwa alle Permutationen der Parameter in einer kommutativen Funktion sollen bereits auf der Mustererkennungsebende berücksichtig werden. Die Formulierung einer Menge von Ersetzungsregeln für das Termersetzungssystem muss also möglichst kompakt möglich sein. 
Damit ist das Ziel die Mustererkennung möglichst schnell durchführen zu können beim Treffen von Designentscheidungen in der Musterstruktur nicht ausschlaggebend, eine polinomielle Laufzeitkomplexität ist allerdings angestrebt.\\
Das Leistungsvermögen des entwickelten Termersetzungssystems wird in einer Anwendung zur Vereinfachung algebraischer Ausdrücke über den Komplexen Zahlen getestet. 

\begin{beispiel}~\\
Es werden vier Vereinfachungsregeln definiert:
\begin{alignat*}{4}
    ~& a \cdot b + a \cdot c & &= a \cdot (b + c) &~~~& (1) \\
    ~& \sqrt{a}              & &= a^{\frac 1 2}   &~~~& (2) \\
    ~& \paren{a^b}^c         & &= a^{b \cdot c}   &~~~& (3) \\
    ~& \sin(a)^2 + \cos(a)^2 & &= 1               &~~~& (4)
\end{alignat*}
Der folgende Ausdruck kann durch Ersetzung der Struktur der linken Seite einer Verienfachungsregel durch die Struktur der rechten Seite vereinfacht werden. Desweiteren werden Ausdrücke ohne Unbekannte ausgewertet.
\begin{equation*}
    \begin{split}
	3 \cdot \sin(x + y)^2 + 3 \cdot \sqrt{\cos(x + y)^4}
	&\stapel = {(1)} 3 \cdot \paren*{\sin(x + y)^2 + \sqrt{\cos(x + y)^4}} \\
	&\stapel = {(2)} 3 \cdot \paren*{\sin(x + y)^2 + \paren*{{\cos(x + y)^4}}^{\frac 1 2}}\\
	&\stapel = {(3)} 3 \cdot \paren*{\sin(x + y)^2 + {\cos(x + y)^{4 \cdot \frac 1 2}}}\\
	& =              3 \cdot \paren*{\sin(x + y)^2 + {\cos(x + y)^2}}\\
	&\stapel = {(4)} 3 \cdot 1\\
    & = 3
    \end{split}
\end{equation*}
Hervorzuheben ist dabei, dass Variablennamen wie $a$ oder $b$ in den Vereinfachungsregeln eine andere Bedeutung haben, als die Variablen $x$ und $y$ im zu vereinfachenden Ausdruck. Erstere sind Platzhalter in der Ersetzungsregel, stehen damit also representativ für einen Teil des Ausdrucks, der transformiert wird, während zweitere ihre Bedeutung außerhalb des Ersetzungssystems haben. Formalisiert wird dieser Unterschied in Abschnitt \ref{subsecMuster}. Als Beispiel gilt für die erste Umformung $b = \sin(x + y)^2$ mit $b$ aus Regel $(1)$.
\end{beispiel}

\subsection{Aufbau der Arbeit}
In Abschnitt \ref{secGrundlegendeDefinitionen} werden die Begriffe eingeführt, die zur Beschreibung eines zu transformierenden Ausdrucks, aber auch zur Beschreibung der Transformation selbst notwendig sind. Mit den dann etablierten Begriffen werden die Algorithmen zur Normalisierung ohne Mustererkennung aus Kapitel \ref{secErsteNormalform} und die Algorithmen zur Mustererkennung in Abschnitt \ref{secPattermatching} erläutert. Aufbauend auf Patternmatching tut dann Kapitel \ref{secTermersetzungssystem}, wo die Transformation eines Terms durch vordefinierte Muster beschrieben wird.
Die tatsächliche Umsetzung und ihre Abweichungen von vorhergehenden Ideen ist in den Abschnitten \BFred{TODO} und \ref{secKernUmsetzungInCpp} behandelt. 
Zum Schluss wird das entwickelte Termersetzungssystem genutzt um Terme über den komplexen Zahlen zu vereinfachen. Weiter werden alternative Sprachen und Systeme mit ähnlichem Fokus diskutiert und diese Arbeit damit verglichen.







\section{Grundlegende Definitionen} \label{secGrundlegendeDefinitionen}

\subsection{Term}
Eine Menge von Termen $T$ ist in dieser Arbeit immer  in Abhängigkeit der Mengen $F$ und $C$, sowie der \emph{Stelligkeitsfunktion} $\mathrm{arity} \colon F \rightarrow \mathbb{N} \cup \{\omega\}$ definiert, ähnlich der Notation von Benanav et. al. in \cite{NPHardMatching}. $F$ enthält die sogenannten \emph{Funktionssymbole}. Beispiele für mögliche Elemente in $F$ sind \texttt{sin} und \texttt{sqrt}, zudem auch Operatoren wie die Division, etwa geschrieben als \texttt{divide}. Die Stelligkeitsfunktion $\mathrm{arity}$ gibt für jedes Funktionssymbol an, wie viele Parameter erwartet werden. Eine mögliche Stelligkeit der genannten Beispielsymbole ist die folgende.

$$\mathrm{arity} f = \begin{cases}
2 & f  = \texttt{divide}\\
1 & f \in \{\texttt{sin}, \texttt{sqrt}\}\\
\end{cases}$$

Kann eine Funktionssymbol $f$ beliebig viele Parameter entgegennehmen, wird gesagt, dass $f$ \emph{variadische} Stelligkeit hat oder \emph{variadisch} ist. Die Stelligkeitsfunktion bildet $f$ dann auf $\omega$ ab. 

Die Menge $C$ enthält die \emph{Konstantensymbole}. Mit den genannten Beispielen für Funktionssymbole, ergibt etwa $C = \mathbb R$ Sinn. Wichtig ist allerdings, dass im folgenden nicht vorausgesetzt wird, dass zwangsweise jedem Konstantensymbol ein eindeutiger numerischer Wert zugeordnet werden kann\footnote{Die Symbole unbekannten Wertes werden häufig von den Konstantensymbolen getrennt und Variablensymbole genannt. Diese Unterscheidung wird hier nicht getroffen, primär um die Definitionen einfach zu halten.}.



Ein Term $t \in T(F, C)$ ist dann  {
\begin{itemize}
	\item{ein Konstantensymbol, also $t \in C$}
	\item{oder eine \emph{Funktionsanwendung} des Funktionssymbols $f \in F$ mit $\mathrm{arity} f \in \{n, \omega\}$ 
		auf die Terme ${\tOneN \in T(F, C)}$, geschrieben ${t = (f, \tOneN)}$}
\end{itemize}}
In Mengenschreibweise:
$$T(F, C) \coloneqq C \cup \curl*{
(f, \tOneN)~|
~f\in F,~\mathrm{arity}(f) \in \{n, \omega\},~ \tOneN \in T(F, C)
}$$ 
Eine Funktionsanwendung wird in der Literatur oft mit dem Funktionssymbol außerhalb des Tupels geschrieben (\cite{buch1977}, \cite{NPHardMatching}), also $f(\tOneN)$ statt $(f, \tOneN)$. Zum deutlicheren Abheben von Funktionen die Terme transformieren zu Termen selbst, wird diese Schreibweise hier keine Verwendung finden. 


\begin{figure}
\Tree [.\texttt{divide} 3 [.\texttt{sin} 1 ] ]
\label{ersterBeispielBaum}
\caption{Baumdarstellung des Terms $(\texttt{divide}, 3, (\texttt{sin}, 1))$ }
\end{figure}

\newtheorem{bBaum}[bsp]{Beispiel}
\begin{bBaum}~\\
Als Beispiel lässt sich der Ausdruck $\frac 3 {\sin 1}$ in der formalen Schreibweise als Term mittels der Funktionssymbole $\texttt{sin}$ und $\texttt{divide}$, sowie den Konstantensymbolen $3$ und $1$ darstellen als $(\texttt{divide}, 3, (\texttt{sin}, 1))$. Ein Term kann dabei auch immer als Baum\footnote{In der theoretischen Informatik auch Syntaxbaum oder AST (englisch für \textit{Abstract Syntax Tree})} aufgefasst werden, etwa das aktuelle Beispiel in in Abb. \ref{ersterBeispielBaum} .
\end{bBaum}

Mit dem Kontext der Baumdarstellung lassen sich nun die folgenden Begriffe auf Terme übertragen. In der Funktionsanwendung $t = (f, \tOneN)$ sind $\tOneN$ die \emph{Kinder} ihres \emph{Vaters} $t$. Kinder sind allgemeiner \emph{Nachkommen}. Nachkommen verhalten sich transitiv, also ein Nachkomme $z$ des Nachkommen $y$ von $x$ ist auch ein Nachkomme von $x$. Umgekehrt ist $x$ \emph{Ahne} von $y$ und $z$. \\


\subsection{Funktionsauswertung}
Die Erweiterung des Funktionssymbols zur Funktion, die von einem Raum $Y^n$ nach $Y$ abbildet, folgt mittels der $\mathrm{eval}$ Funktion frei nach \cite{buch1977}.

\begin{equation*}
    \begin{split}
	\mathrm{eval} &\colon \paren*{F \rightarrow \bigcup_{n \in \mathbb{N}} Y^n \rightarrow Y} \times (C \rightarrow Y) \rightarrow T \rightarrow Y\\
	\mathrm{eval} &(u, v)~t = \begin{cases}
		u~f~(\elems {\mathrm{eval}(u, v)~t} 1 n) & t = (f, \tOneN)\\
		v~t                                      & t \in C\\
		\end{cases}
    \end{split}
\end{equation*}
Gilt $\mathrm{arity} f = n \in N$ für ein $f \in F$, ist zudem die Funktion $u~f \colon Y^n \rightarrow Y$ in ihrer Definitionsmenge auf Dimension $n$ eingeschränkt. 
Die Funktion $u$ wird als \emph{Interpretation} der Funktionssymbole $F$, die Funktion $v$ als Interpretation der Konstantensymbole $C$ und das Paar $(u, v)$ als Interpretation der Terme $T(F, C)$ bezeichnet. Die Funktion $\mathrm{eval}(u, v) \colon T \rightarrow Y$ ist eine \emph{Auswertung} nach $Y$. 
\\~\\

\newtheorem{bEval}[bsp]{Beispiel}
\begin{bEval} \label{bEval}
Sei $F = \{\texttt{sum}, \texttt{prod}, \texttt{neg} \}$ und $C = \mathbb{N}$ mit $\mathrm{arity}~ \texttt{sum} = \mathrm{arity}~ \texttt{prod} = \omega$ und $\mathrm{arity}~ \texttt{neg} = 1$.
Die Interpretation $(u, v)$ kann so gewählt werden, dass jeder Term in $T$ zu einer ganzen Zahl $n \in \mathbb{Z}$ auswertbar ist.

\begin{equation*}
    \begin{split}
    u~\texttt{sum}  ~(\elems y 1 n) &= \Sigma_{k = 1}^n y_k\\
    u~\texttt{prod} ~(\elems y 1 n) &=    \Pi_{k = 1}^n y_k\\
    u~\texttt{neg}~y &= -y\\
    &\\
    v~y &= y
    \end{split}
\end{equation*}

Hervorzuheben ist dabei, dass $u~\texttt{neg} \colon \mathbb Z \rightarrow \mathbb Z$ nur eine ganze Zahl als Parameter erwartet, während $u~\texttt{sum}$ und $u~\texttt{prod}$ Tupel ganzer Zahlen beliebiger Länge abbilden können.
Der Term $t = (\texttt{sum}, 3, (\texttt{prod}, 2, 4), (\texttt{neg}, 1))$ kann dann ausgewertet werden zu 
\begin{equation*}
    \begin{split}
    \mathrm{eval}(u, v)~t &= \mathrm{eval}(u, v) (\texttt{sum}, 3, (\texttt{prod}, 2, 4), (\texttt{neg}, 1)) \\
    &= u~\texttt{sum}~(\mathrm{eval}(u, v)~3, \mathrm{eval}(u, v)(\texttt{prod}, 2, 4),  \mathrm{eval}(u, v) (\texttt{neg}, 1)) \\
    &= u~\texttt{sum}~(v~3, u~\texttt{prod}~(\mathrm{eval}(u, v)~2, \mathrm{eval}(u, v)~4), u~\texttt{neg}~ (v~1)) \\
    &= u~\texttt{sum}~(3, u~\texttt{prod}~(v~2, v~4), u~\texttt{neg}~ 1) \\
    &= u~\texttt{sum}~(3, u~\texttt{prod}~(2, 4), -1) \\
    &= u~\texttt{sum}~( 3, 8, -1) \\
    &= 10 \\
    \end{split}
\end{equation*}
\end{bEval}


\subsubsection{Konstruktor}
Eine direkt aus der Struktur des Terms folgende Interpretation $u_c$ für Funktionssymbole ist die des \emph{Konstruktors}. Als Konstruktor eines Typen $A$ wird allgemein eine Funktion bezeichnet, die nach $A$ abbildet. Für typenlose Terme ist diese Definition deshalb schwierig. Insbesondere in funktionalen Sprachen wird das Konzept allerdings noch verfeinert. In Haskell transformiert ein Konstruktor Daten nicht im eigentlichen Sinne, sondern bündelt lediglich ein Tag, das festhält welcher Konstruktor genutzt wurde, mit den $n$ übergebenen Paramtern zu einem Tupel der Größe $n+1$ \cite{haskellConstructor}. Das entspricht exakt der Funktionsanwendung in einem Term, mit dem Funktionssymbol als Tag. Wenn $u_c$ angewendet auf ein Funktionssymbol $f$ das $n$ Tupel von Parametern auf eine Funktionsanwendung von $f$ mit den selben $n$ Parametern abbildet, ist $u_c~f$ damit ein Konstruktor. 

\newtheorem{defKonstruktor}[bsp]{Definition}
\begin{defKonstruktor}~\\
 Mit $f \in F$ und $\mathrm{arity} f = n \in \mathbb N$ 
gilt 
$$u_c~f \colon T^n \rightarrow T, ~(\tOneN) \mapsto (f, \tOneN)$$
Mit einem beliebigen $v \colon C \rightarrow C'$ ändert die Auswertung $\mathrm{eval}(u_c, v) \colon T(F, C) \rightarrow T(F, C')$ damit nur die Konstantensymbole eines Terms, lässt aber die sonstige Struktur unverändert. Insbesondere ist $\mathrm{eval}(u_c, v) \colon T \rightarrow T$ mit $v~y = y$ die Identität.

Die Interpretation $u_c$ reicht für bestimmte Funktionssymbole aus, etwa kann so das Funktionssymbol $\texttt{pair}$ ein Paar als Term darstellen.
$$u_c~\texttt{pair} \colon T^2 \rightarrow T, ~(a, b) \mapsto (\texttt{pair}, a, b)$$
Äquivalent ist die Darstellung endlicher Mengen und Tupel mit variadischen Funktionssymbolen \texttt{set} und \texttt{tup} möglich\footnote{Da eine Menge ihren Elementen keine Reihenfolge gibt, muss $u_c~\texttt{set}$ im Unterschied zu $u_c~\texttt{tup}$ prinzipiell nicht die ursprüngliche Parameterreihenfolge erhalten. In Kapitel \ref{subsecNormalSortieren} wird eine Größenrelation zur möglichen Umordung diskutiert.}.
\end{defKonstruktor}


\newtheorem{bspKonstruktor}[bsp]{Beispiel}
\begin{bspKonstruktor}~\\
Ein Graph $G = (V, E)$ wird als Paar der Menge von Knoten $V$ und Menge der Kanten $E$ definiert. Eine Kante ist dabei eine zweielementige Menge von Knoten. Der vollständige Graph auf drei Knoten ist $K_3 = \paren*{\{1, 2, 3\}, \curl*{\{1, 2\}, \{2, 3\}, \{1, 3\}}}$. Mit der Interpretation der Funktionssymbole $\texttt{pair}$ und $\texttt{set}$ als Konstruktoren lässt sich $K_3$ auch als Term darstellen:
$$(\texttt{pair}, (\texttt{set}, 1, 2, 3), (\texttt{set}, (\texttt{set}, 1, 2), (\texttt{set}, 2, 3), (\texttt{set}, 1, 3)))$$
\end{bspKonstruktor}

\subsection{Muster} \label{subsecMuster}

Bisher wurden die Objekte beschrieben, die in dieser Arbeit transformiert werden sollen. Die Transformationsregeln selbst lassen sich allerdings auch als Paare von bestimmten Termen darstellen. Zur Abgrenzung beider Konzepte werden die zu transformierenden Terme $t\in T(F, C)$ von hier an \emph{Literal} genannt, Terme die  Teil einer Regeldefinition sind werden \emph{Muster} genannt. Die Menge der Muster $M(F, C)$ ist dabei eine Obermenge der Literale, da sie deren Konstantensymbole um die Menge der \emph{Mustervariablen} $X$ erweitert\footnote{Die Ergänzung der Funktionssymbole um Mustervariablen ist genau so möglich, wird aber vor allem um die Notation verdaubar zu halten in den folgenden Kapiteln außen vor gelassen.}. Konkrete Elemente $\mathbf x \in X$ werden im folgenden \textbf{fett} geschrieben.
$$M(F, C) \coloneqq T(F, C \cup X)$$

Eine \emph{Ersetzungsregel} für Literale $t \in T(F, C)$ hat die Form $(l, r) \in M(F, C) \times M(F, C)$. Die linke Seite $l$ steht für das Muster, dass im Literal durch einen Ausdruck der Form der rechten Seite $r$ ersetzt werden soll. Für die bessere Lesbarkeit wird statt $(l, r)$ auch $l \mapsto r$ geschrieben.

\newtheorem{bMuster}[bsp]{Beispiel} 
\begin{bMuster} \label{bMuster}
Die Regel, die die Summe zweier identischer Terme $a$ als Produkt von $2$ und $a$ transformiert wird geschrieben als
$$(\texttt{sum}, \mathbf a, \mathbf a) \mapsto (\texttt{prod}, 2, \mathbf a)$$
Wird die Regel jetzt auf das Literal 
$t = (\texttt{sum}, (\texttt{sin}, 3), (\texttt{sin}, 3))$ angewandt, kann man $t$ zu $t' = (\texttt{prod}, 2, (\texttt{sin}, 3))$ transformieren. 
Hervorzuheben ist dabei, dass die Mustervariable $\mathbf a$ selbst nicht mehr im Ergebnisterm vorkommt. Sie wurde stattdessen durch den Teilterm ersetzt, der im Ursprungsliteral an der Stelle von $\mathbf a$ stand, nämlich $(\texttt{sin}, 3)$.
\end{bMuster}

\newtheorem{defMatch}[bsp]{Definition}
\begin{defMatch}
Für ein Paar $(p, t) \in M(F, C) \times T(F, C)$ ist eine Funktion $v_p \colon X \rightarrow T(F, C)$ ein \emph{Match}, wenn folgendes gilt:
$$\mathrm{eval}(u_c, \tilde v_p)~ p = t$$
$$\tilde v_p~ c = \begin{cases}
	v_p~ c & c \in X\\
	c      & c \in C \setminus X
\end{cases}$$
$v_p$ muss die Mustervariablen in $p$ so durch Literale ersetzen, dass ein Term identisch zu $t$ entsteht. 
Im vorangegangenen Beispiel \ref{bMuster} gilt damit $v_p~ \mathbf a = (\texttt{sin}, 3)$.

Im Folgenden wird der Begriff des Matches noch etwas weiter gefasst. Es werden nach wie vor die Mustervariablen durch Literale ersetzt, allerdings muss nicht direkt das Ergebnis der Ersetzung, sondern nur eine normalisierte Form des Ergebnisses mit dem Literal $t$ übereinstimmen:
$$\mathrm{normalize}~(\mathrm{eval}(u_c, \tilde v_p)~ p) = t$$
Aus der Definition ist bereits klar, dass $\mathrm{normalize} \colon T(F, C) \rightarrow T(F, C)$ Terme auf Terme abbildet und ein Match $v_p$ nur dann gefunden werden kann, wenn $t$ im Bild von $\mathrm{normalize}$ liegt. 
Gedacht ist $\mathrm{normalize}$ als Mittel, um Unterschiede zwischen Termen zu reduzieren, die die Auswertung für eine gebebende Interpretation $(u, v)$ nicht ändern. Da das Ergebnis von $\mathrm{normalize}$ aber nur von dem übergebenen Term abhängig ist, können nie direkt Unterschiede zwischen mehreren Termen verglichen und beseitigt werden. Ist die normalisierte Form $t'$ eines Terms $t$ ein Fixpunkt von $\mathrm{normalize}$, ist in dem Kontext klar, dass $t$ und $t'$ die gleiche Auswertung mit der Interpretation $(u, v)$ besitzen, da auch $\mathrm{normalize}~t' = \mathrm{normalize}~t$ gilt. Wäre $\mathrm{normalize}~t' \neq t'$, würde die Funktion nicht die ihr zugedachte Aufgabe erfüllen. Deshalb muss $\mathrm{normalize}$ eine Projektion sein, also eine Funktion, für die jeder Bildpunkt gleichzeitig ein Fixpunkt ist. 

Welche Unterschiede $\mathrm{normalize}$ beseitigt soll hier nicht festgelegt werden. Klar ist aber, dass je nach Wahl von $\mathrm{normalize}$ zwar ein einzelnes Muster sehr mächtig werden kann, also ein Match mit sehr vielen Termen möglich ist, das Finden des Matches dann im allgemeinen Fall allerdings immer aufwändiger wird. 

\end{defMatch}

Ein \emph{Matchalgorithmus} ist eine Vorgehensweise für ein gegebenes Paar $(p, t) \in M(F, C) \times T(F, C)$ ein gültiges Match zu finden. Perfekt wird ein solcher Algorithmus dann genannt, wenn jedes mögliche Match gefunden werden kann.







\section {Normalform} \label{secErsteNormalform}

Das Kernthema dieser Arbeit ist die Vereinfachung von Termen. Eine Vereinfachung ist allerdings nur gültig, sofern sich die Bedeutung des vereinfachten Terms gegenüber der des ursprünglichen Terms nicht geändert hat. Da ein Term in sich keine Bedeutung trägt, muss eine Vereinfachung immer in Bezug auf eine Interpretation $(u, v)$ gesehen werden. Etwa kann der Ausdruck $X A X^{-1}$ zu $A$ vereinfacht werden, wenn $X, A \in \mathbb{C} \setminus \{0\}$, allerdings ist die Vereinfachung allgemein nicht möglich, sollten die Symbole $X$ und $A$ für Matritzen stehen. \\
Im Folgenden wird von der Assoziativität oder Kommutativität bestimmter Funktionssymbole gesprochen. Diese ist immer im Kontext der Interpretation $(u, v)$ zu sehen. Gleichzeitig ist aber auch klar, dass unabhängig von der Interpretation verschiedene Funktionssymbole die Rolle der Multiplikation übernehmen müssen, sollte sowohl skalare Multiplikation als auch Matrixmultiplikation im selben Term möglich sein. $X A X^{-1}$ als Matrixmultiplikation könnte der Term $(\texttt{prod'}, X, A, (\texttt{pow}, X, -1))$ darstellen. Sind $A$ und $X$ Skalare, wäre der Ausdruck als $(\texttt{prod}, X, A, (\texttt{pow}, X, -1))$ schreibbar. Das Funktionssymbol $\texttt{prod'}$ steht dann für ein nicht-kommutatives Produkt, während die Reihenfolge der Parameter in einer Funktionsanwendung von $\texttt{prod}$ keine Rolle spielt.\\

In diesem Abschnitt werden einfache Termumformungen beschrieben, die isolierte Eingenschaften einzelner Funktionen ausnutzen. Ziel ist es Äquivalenzklassen für die Erkennung von Mustern zu schaffen, die über die Austauschbarkeit jeder Mustervariable mit einem beliebigen Literal hinausgehen. Als Beispiel dient die Regel der Faktorisierung, normal geschrieben $a \cdot b + a \cdot c = a \cdot (b + c)$. In der in Unterkapitel \ref{subsecMuster} etablierten Musterschreibweise, mit Mustervariablen \textbf{fett} geschrieben, wird daraus:
$$(\texttt{sum}, (\texttt{prod}, \mathbf a, \mathbf b), (\texttt{prod}, \mathbf a, \mathbf c)) \mapsto (\texttt{prod}, \mathbf a, (\texttt{sum}, \mathbf b, \mathbf c))$$
Ziel ist es, die Regel auf das Literal $(\texttt{sum}, (\texttt{prod}, x, y), (\texttt{prod}, w, x, z))$ anwendbar zu machen, bzw eine Regel schreiben zu können, die eine ähnliche Struktur hat und anwendbar ist. 
Würde das Literal geschrieben sein als $(\texttt{sum}, (\texttt{prod}, x, y), (\texttt{prod}, x, (\texttt{prod}, w, z)))$, gäbe es ein Match $v_p$ der linken Regelseite mit dem Literal, mit $v_p~\mathbf a = x$, $v_p~\mathbf b = y$ und $v_p~\mathbf c = (\texttt{prod}, w, z)$. Ergebnis dieses Kapitels wird eine in Abschnitt \ref{subsecMuster} genutzte Projektion ${\mathrm{normalize} \colon T \rightarrow T}$ sein, welche die Beispielregel in leicht abgewandelter auf das Beispielliteral in seiner ursprünglichen Form anwendbar macht.\\

Weiteres Ziel dieses Kapitels ist, dass möglichst viele Literale mit identischer Auswertung identische normalisierter Terme besitzen. So soll etwa die Normalisierung von $t_1 = (\texttt{sum}, a, b, c)$ identisch zur Normalisierung von $t_2 = (\texttt{sum}, b, a, c)$ identisch zur Normalisierung von $t_3 = (\texttt{sum}, (\texttt{sum}, b, a), c)$ sein. Je mehr Literale identischen Wertes auch zu identischen Termen normalisiert werden, desto besser können Muster erkannt werden, in denen dieselbe Mustervariable mehrfach vorkommt. Interessant ist dabei, dass derselbe Effekt auch erreicht werden würde, wenn man eine Menge von Ersetzungsregeln um entsprechende normalisierende Ersetzungsregeln ergänzt. Wo die Grenze in der Arbeitsteilung von einer fest implementierten $\mathrm{normalize}$ Funktion zu den Regeln in einem Termersetzungssystem liegt, ist prinzipiell fast beliebig und in erster Linie eine Frage des Aufwandes, sowohl in Programmierung als auch Laufzeit. Etwa würde eine Darstellung natürlicher Zahlen ähnlich der Church-Numerale, wie sie in der Fachliteratur, etwa bei Baader und Nipkow \cite{baader_nipkow_1998}, üblich ist, erlauben, Rechenoperationen auf den natürlichen Zahlen komplett mit einer endlichen Menge von Mustern auszuwerten. Nachteile dieser Vorgehensweise wären allerdings eine langsamere Auswertung, mehr Speicherbedarf und nur sehr schwierig zu lesende Ergebnisse. Andersherum wäre etwa die Anwendung der ersten binomischen Formel prinzipiell auch in der $\mathrm{normalize}$ Funktion möglich, allerdings steht der Aufwand manuell  auf das Muster zu testen nur möglicherweise minimalen Geschwindigkeitsvorteilen gegenüber. Die Transformationen, die in diesem Kapitel der $\mathrm{normalize}$ Funktion zugewiesen werden, sollen also idealerweise nicht einfacher mit Mustern implementierbar sein.\\

In diesem Kapitel werden häufig Abschnitte der Parameter einer Funktionsanwendung beliebiger Länge der Form $\elems t i k$ vorkommen. Kompakt wird $ts...$ für den (möglicher\-weise leeren) Abschnitt des Funktionsanwendungstupels geschrieben. Das $s$ in $ts...$ ist dann nicht als einzelnes Symbol zu lesen, sondern als Suffix um $t$ in den Plural zu setzen. \\$(f, \elems t 1 k, a, \elems t {k+2} n)$ kann also äquivalent $(f, ts..., a, rs...)$ geschrieben werden, mit $(\elems t 1 k) = (ts...)$ und $(\elems t {k+2} n) = (rs...)$.\\

\subsection {Assoziative Funktionsanwendungen}
Die geschachtelte Anwendung einer assoziativen Funktion führt je nach Klammersetzung zu verschiedenen mathematisch equivalenten Termen. Als Beispiel dient hier die Addition, dargestellt als Anwendung des Funktionssymbols $\texttt{sum}$. Die folgenden Ausdrücke sind paarweise verschiedene Terme, jedoch in ihrer Interpretation als Summe von $a$, $b$, $c$ und $d$ alle mathematisch äquivalent.
\begin{equation*}
	\begin{split}
	   (\texttt{sum}, (\texttt{sum}, (\texttt{sum}, a, b), c), d) 
    &= (\texttt{sum}, (\texttt{sum}, a, (\texttt{sum}, b, c)), d)\\
	&= (\texttt{sum}, (\texttt{sum}, a, b), (\texttt{sum}, c, d))\\
	&= (\texttt{sum}, a, (\texttt{sum}, b, (\texttt{sum}, c, d)))\\
	&= \dots \\
	\end{split}
\end{equation*}
Es gibt mehrere Optionen eine solche Schachtelung in einem Term zu normalisieren, also in eine eindeutige Form zu bringen. Die erste ist, festzulegen, dass in der normalisierten Form höchstens einer der beiden Parameter einer binären assoziativen Funktion wieder Anwendung des selben Funktionssymbols sein darf. Wählt man den zweiten Parameter dafür aus, wird die Summe in der Normalform dargestellt als $(\texttt{sum}, a, (\texttt{sum}, b, (\texttt{sum}, c, d)))$. Ein Problem der Methode ist, dass nicht immer alle Parameter eines assoziativen Funktionssymbols direkt vorliegen.\\
Alternativ kann man die Summe von zwei Parametern auch als Spezialfall einer Summe von $n \in \mathbb{N}$ Parametern auffassen, gewohnt geschrieben als $\Sigma_{x \in \{a, b, c, d\}} x$. Dieser Weg wird im Folgenden gewählt, wobei die Darstellung als Term dann $(\texttt{sum}, a, b, c, d)$ ist. Assoziative Funktionen sind in der gewählten Darstellung damit variadisch. Eker (\cite{BipartiteGraphMatching}), Benanav (\cite{NPHardMatching}) und Kounalis (\cite{ACPatternCompiler}) wählen unter anderen ebenfalls diese Darstellung. 
Die Normalisierung von Funktionsanwendungen des assoziativen Funktionssymbols $f$ bedeutet dann geschachtelte Funktionsanwendungen in eine einzelne Funktionsanwendung zu übersetzen. 
$$(f, as..., f(bs...), cs...) \mapsto (f, as..., bs..., cs...)$$
Der Spezialfall ist eine assoziative Funktionsanwendung mit nur einem Parameter. Diese kann immer zu dem Parameter selbst normalisiert werden. 

Als Algorithmus sind die Überlegungen dargestellt in Algorithmus \ref{flatten}.
Die Funktion $u_n \colon T_C \rightharpoonup C$ kann hier und in den weiteren Algorithmen dieses Kapitels als \glqq{natürliche}\grqq{} Interpretation der Menge von Funktionssymbolen gesehen werden, ähnlich $u$ in Beispiel \ref{bEval}. Prinzipiell ist für die Gültigkeit des Kapitel aber egal, in welchem Kontext die Abbildungsvorschrift von $u_n$ Sinn ergibt oder ob ein solcher Kontext überhaupt existiert. wichtig ist nur, das $u_n$ über das gesamte Kapitel hinweg eine einheitliche Abbildungsvorschrift hat.

\begin{algorithm}
\DontPrintSemicolon
\caption{$\mathrm{flatten} \colon T \rightarrow T$}\label{flatten}
\KwIn{$t \in T(F, C)$}

\If{$t = (f, t_1)$ mit $u_n~f$ assoziativ}{
    \Return {$t_1$}
}
\ElseIf{$t = (f, t_1, \dots, t_n)$ mit $u_n~f$ assoziativ}{
    \While{$t = (f, as..., (f, bs...), cs...)$}{
        $t \leftarrow (f, as..., bs..., cs...)$\;
    }
}
\Return {$t$}
\end{algorithm}

\subsection{Kommutative Funktionsanwendungen} \label{subsecNormalSortieren}
Eine Normalform für kommutative Funktionsanwendungen erfordert eine totale Ordnung auf der Menge aller Terme $T(F, C)$. Aufbauend auf einer totalen Ordnung von $F$ sowie $C$, kann eine lexikographische Ordnung $<$ von $T$ frei nach \cite{LexikografischeOrdnung} wie folgt definiert werden. 
\begin{enumerate}
	\item{sind $c, \tilde{c} \in T$ Konstantensymbole, so ist die Ordnung identisch zu der Ordnung in $C$}
	\item{sind $c, a, \in T$ sowie $c$ ein Konstantensymbol und $a$ eine Funktionsanwendung gilt $c < a$ }
	\item{sind $a = (f, ts...), b = (g, rs...) \in T$ Funktionsanwendungen und ist $f \neq g$ gilt $a < b \iff f < g $}
	\item{sind $a = (f, t_1, \dots, t_n), b = (f, r_1, \dots, r_m) \in T$ Funktionsanwendungen ist die Ordnung wie folgt}
	\begin{itemize}
		\item{wenn $\exists k \leq \min{(n, m)} \colon \forall i < k ~ t_i = r_i ,~ t_k \neq r_k $ gilt ${a < b \iff t_k < r_k}$}
		\item{ist $n < m$ und $\forall i < n\colon t_i = r_i$ gilt $a < b$}
		\item{ist $n = m$ und $\forall i \leq n\colon t_i = r_i$ gilt $a = b$}
	\end{itemize}
\end{enumerate}
Zur Normalisierung einer kommutativen Funktionsanwendung werden zuerst alle Parameter normalisiert, dann können die Parameter nach der lexikographischen Ordnung $<$ von $T$ sortiert werden. 

\subsection{Teilweise Auswertung} \label{subsecNormalKombinieren}

\begin{algorithm}
\DontPrintSemicolon
\caption{$\mathrm{combine} \colon T \rightarrow T$}\label{combine}
\KwIn{$t \in T(F, C)$}

\If{$\mathrm{eval}(u_n, \mathrm{id})~t = c \in C$}{
    \Return {$c$}
}
\ElseIf{$t = (f, t_1, \dots, t_n)$ und $u_n~f$ assoziativ}{
    \If{$f~u_n$ kommutativ}{
        \While{$t = (f, xs..., a, ys..., b, zs...) $ und $ u_n~f~(a, b) = c \in C$}{
            $t \leftarrow (f, xs..., c, ys..., zs...)$\;
        }
    }
    \Else{
        \While{$t = (f, xs..., a, b, ys...) $ und $ u_n~f~(a, b) = c \in C$}{
            $t \leftarrow (f, xs..., c, ys...)$\;
        }
    }
}
\end{algorithm}

Mit der Darstellung einer assoziativen Funktion mit einem variadischen Funktionssymbol $f \in F$, kann eine Funktionsanwendung von $f$ in bestimmen Fällen teilweise ausgewertet werden. Als Beispiel kann die Summe der Symbole $1$, $3$ und $\texttt{x}$ geschrieben als $(\texttt{sum}, 1, 3, \texttt{x})$ zur Summe $(\texttt{sum}, 4, \texttt{x})$ transformiert werden. 
Gilt allgemein für ein Funktionssymbol $f \in F$, dass $u_n~f$ assoziativ ist, reicht es aus zwei aufeinander folgende Parameter $a$ und $b$ in einer Funktionsanwendung von $f$ zu finden, mit denen die Funktionsanwendung $(f, a, b)$ auswertbar wäre. $a$ und $b$ können dann entsprechend ersetzt werden.
$$\mathrm{eval}(u_n, \mathrm{id})~(f, a, b) = c \in C \implies (f, xs..., a, b, ys...) \mapsto (f, xs..., c, ys...)$$

Ist $u_n~f$ zudem kommutativ, müssen $a$ und $b$ nicht notwendigerweise direkt aufeinander folgen:
$$\mathrm{eval}(u_n, \mathrm{id})~(f, a, b) = c \in C \implies (f, xs..., a, ys..., b, zs...) \mapsto (f, xs..., c, ys..., zs...)$$

Eine normalisierte Funktionsanwendung enthält keine zwei auf diese Art ersetzbare Parameter $a$ und $b$ mehr. Weiter ist jede Funktionsanwendung, die als ganzes zu einer Konstante $c \in C$ auswertbar ist, ausgewertet.\\
Als Algorithmus dargestellt sind die Überlegungen in Algorithmus \ref{combine}. Die Verfahren zur Normalisierung assoziativer und kommutativer Funktionssymbole werden auch von Eker \cite{BipartiteGraphMatching} beschrieben.

\subsection{Kombination der einzelnen Vereinfachungen} \label{subsecKomboNormal}

\begin{algorithm}
\DontPrintSemicolon
\caption{$\mathrm{normalize} \colon T \rightarrow T$}\label{normalize}
\KwIn{$t \in T(F, C)$}

\If {$t = (f, t_1, \dots, t_n)$}{
	\For {$i \in \{1, \dots, n\}$}{
		$t_i \leftarrow \mathrm{normalize}~t_i$\;
	}
}
$t \leftarrow \mathrm{flatten}~t$\;
$t \leftarrow \mathrm{combine}~t$\;
\If {$t = (f, t_1, \dots, t_n)$ mit $u_n~f$ kommutativ}{
	sortiere $t_1, \dots, t_n$ lexikographisch nach Ordnung $<$\;
}
\Return $t$ 
\end{algorithm}
Algorithmus \ref{normalize} kombiniert die einzelnen Überlegungen dieses Kapitels: Zuerst werden alle Parameter einer Funktionsanwendung normalisiert, dann die Funktionsanwendung selbst.

\BFred{TODO: normalize ist eine Projektion}







\chapter{Mustererkennung} \label{secPattermatching}

In Kapitel \ref{subsecMuster} wurde die Konzepte des Musters und des Matches eingeführt, zweiteres insbesondere in einer weiterfassenden Form, was auch erlaubt Muster mit strukturell nicht exakt identischen Literalen zu assoziieren, sofern die Unterschiede mit der Projektion $\mathrm{normalize} \colon T \rightarrow T$ beseitigt werden können. Die in Kapitel \ref{secErsteNormalform} beschriebene Funktion $\mathrm{normalize}$ ist als solche Projektion nutzbar.  

In diesem Kapitel wird ein Algorithmus entwickelt, der die Äquivalenzklassen der verschieden geschachtelten Funktionsanwendungen eines assoziativen Funktionssymbols mit den selben Parametern, sowie die Äquivalenzklassen der verschieden permutierten Parameter in der Funktionsanwendung eines kommutativen Funktionssymbols beim Finden eines Matches berücksichtigt. Die teilweise Auswertung von $\mathrm{normalize}$ aus Kapitel \ref{subsecNormalKombinieren} wird allerdings in diesem Kapitel nicht verfolgt.


%.........................................................................
%.........................................................................
%.........................................................................
\section{Grundstruktur} \label{subsecPatternmatchingGrundstruktur}

Dem Ergebnis eines Matchalgorithmus müssen zwei Dinge entnehmbar sein. Zum ersten muss klar sein, ob ein Match $v_p \colon X \rightarrow T$ gefunden wurde. Wurde ein Match gefunden, muss zudem dessen Abbildungsvorschrift zurückgegeben werden. Der Rückgabetyp von Algorithmus \ref{simpleMatchAlgorithmShell} ist deswegen nicht nur das finale Match, sondern auch ein Wahrheitswert $b \in \mathit{Bool} \coloneqq \{\mathrm{false}, \mathrm{true}\}$. Alternativ kann die Menge aller möglichen Matches zurückgegeben werden. Diese Idee wird im Folgenden allerdings nicht weiter verfolgt, da sie mit den Anforderungen an hier behandelte Muster auch im besten Fall schnell exponentielle Laufzeiten produziert. Sind aber Mehrfachnennungen einer Mustervariable in einem Muster nicht erlaubt, haben Hoffman und O'Donnell in \cite{patternMatchingInTrees} gezeigt, dass sehr effiziente Algorithmen zum gleichzeitigen finden von Matches einer ganzen Menge von Mustern in allen Teiltermen eines Literals mit dieser Grundidee möglich sind.\\

Da eine Mustervariable in dieser Arbeit mehrfach in einem Muster vorkommen darf, muss ein Algorithmus beim Suchen nach einem Match $v_p \colon X \rightarrow T$ zu jedem Zeitpunkt wissen, für welche $x \in X$ das Match $v_p~x$ bereits feststeht. $v_p$ ist also nicht nur Rückgabewert eines Matchalgorithmus, sondern muss mit den Funktionswerten für bereits besuchte Mustervariablen auch Eingabe in den Algorithmus sein. In Algorithmus \ref{simpleMatchAlgorithmShell} wird $v_p$ deswegen als partielle Funktion definiert, welche zu Beginn keine einzige Mustervariable nach $T$ abbilden kann. \\

\begin{algorithm}
\DontPrintSemicolon
\caption{$\mathrm{simpleMatchAlgorithmShell} \colon M \times T \rightarrow (\mathit{Bool}, X \rightharpoonup T)$}\label{simpleMatchAlgorithmShell}
\KwIn{$p \in M$, $t \in T$}

\textbf{let} $v_p \colon X \rightharpoonup T,~ x \mapsto \bot$\;
\Return {$\mathrm{simpleMatchAlgorithm}(p, t, v_p)$}
\end{algorithm}

\begin{algorithm}
\DontPrintSemicolon
\caption{$\mathrm{simpleMatchAlgorithm} \colon M \times T \times (X \rightharpoonup T) \rightarrow (\mathit{Bool}, X \rightharpoonup T)$}\label{simpleMatchAlgorithm}
\KwIn {$p \in M$, $t \in T$, $v_p \colon X \rightharpoonup T$}

\If {$p \in X$ \KwAnd $v_p~p = \bot$} {
	$(v_p~p) \leftarrow t$\;
	\Return {$(\mathrm{true}, v_p)$}
}
\ElseIf {$p \in X$ \KwAnd $v_p~p \neq \bot$}{
	\Return {$(v_p~p = t, v_p)$}
}
\ElseIf {$p \in C \setminus X$} {
	\Return {$(p = t, v_p)$}
}
\ElseIf {$p = (f, \elems p 0 {m-1})$ \KwAnd $t = (f, \elems t 0 {n-1})$}{
	\For {$k \in \{0, \dots, {n-1}\}$}{
		$(\mathit{success}_k, v_p) \leftarrow \mathrm{simpleMatchAlgorithm}(m_k, t_k, v_p)$\;
		\If {$\mathrm{not}~\mathit{success}_k$}{
			\Return {$(\mathrm{false}, v_p)$}
		}
	}
	\Return {$(\mathrm{true}, v_p)$}  
}
\Else {
	\Return {$(\mathrm{false}, v_p)$}  
}
\end{algorithm}


Wenn das Match streng definiert ist, also der Unterschied zwischen einem Muster $p$ und einem Literal $t$ für die Existenz eines Matches $v_p$ aussschließlich darin bestehen darf, dass Teilterme von $t$ in $p$ durch eine Mustervariable repräsentiert werden, ist ein einfacher Matchalgorithmus fast trivial. Auf der Idee von $\mathrm{simpleMatchAlgorithm}$ basieren allerdings auch die späteren Algorithmen dieses Kapitels. Diese ist, dass mit einer Tiefensuche, die parrallel durch Muster und Literal läuft, nach einem Unterschied zwischen beiden gesucht wird. Mustervariablen funktionieren dabei als Wildcard, wenn eine identische Mustervariable in der Tiefensuche vorher noch nicht gefunden wurde. Andernfalls vergleichen sie identisch zu dem Teilbaum, der mit dem ersten Vorkommen der Mustervariable verglichen wurde. Die Aufgabe diese vorher begegneten Teilbäume zu speichern übernimmt $v_p$, was erklärt, warum $v_p$ auch als Parameter für Algorithmus \ref{simpleMatchAlgorithm} notwendig ist. Ist das gesamte Muster durchlaufen worden ohne einen strukturellen Unterschied zum Literal zu finden, ist $v_p$ das resultierende Match.\\

\begin{lemma}~\\
Die Laufzeit von Algorithmus \ref{simpleMatchAlgorithmShell} ist linear abhängig von der Anzahl der Funktionssymbole und Konstantensymbole des Literals.
\end{lemma}

\textbf{Beweis}.\\
Gibt es ein Match, wird jedes Funktionssymbol und Konstantensymbol des Literals höchstens ein mal in simpleMatchAlgorithm abgelaufen. Wird eine Funktionsanwendung $t$ im Literal parrallel zu einer Mustervariable $\mathbf x$ im Muster abgelaufen, bleiben die Nachkommen von $t$ unbesucht wenn $\mathbf x$ noch nicht gematcht wurde. Andernfalls wird jeder Nachkomme von $t$ höchstens ein mal abgelaufen, um Gleichheit zu $v_p~\mathbf x$ zu testen.
Gibt es kein Match wird das Literal so lange indentisch zum anderen Fall abgelaufen, bis ein struktureller Unterschied festgestellt wurde. Dann bricht der Algorithmus ab.
\hfill $\square$\\


\begin{definition}~\\
Die Instanz einer Mustervariable $\mathbf x$ wird als \emph{bindend}  bezeichnet, wenn sie in einer Tiefensuche durch das gesamte Muster als erste Instanz abgelaufen wird. Da Algorithmus \ref{simpleMatchAlgorithm} das Muster in einer Tiefensuche abläuft, ist die Bedingung $p \in X$ \KwAnd $v_p~p = \bot$ in der ersten Zeile damit genau dann wahr, wenn $p$ bindend ist\footnote{Der Begriff \emph{bindend} ist so zu verstehen, dass $\mathbf x$ nach Ablauf der ersten Instanz in Algorithmus \ref{simpleMatchAlgorithm} einen festen Wert $v_p~\mathbf x$ hat, also für den spätere Teile des Musters an diesen Wert gebunden ist.}. Weitere Instanzen von $\mathbf x$ im selben Muster werden als \emph{gebunden} bezeichnet.
\end{definition}


%.........................................................................
%.........................................................................
%.........................................................................
\section{Multi-Mustervariablen} \label{subsecMulti}

Von Anfang an werden Funktionssymbole in dieser Arbeit als möglicherweise variadisch definiert. Das ist in sofern ein Problem, als Muster bisher immer nur eine feste Anzahl an Parametern für jede Funktionsanwendung angeben können. Ist ein variadisches Funtionssymbol zudem assoziativ, ließe sich dieses Problem prinzipiell beheben, wenn Assoziativität im Matchalgorithmus berücksichtigt würde. Das Muster $\tilde p = (f, \mathbf x, \mathbf y)$ würde für ein assoziatives Funktionssymbol $f$ dann auch Literale wie $\tilde t = (f, a, b, c, d)$ matchen, mit verschiedenen Optionen für $v_p$, etwa $v_p~\mathbf x = (f, a, b)$ und $v_p~\mathbf y = (f, c, d)$. Ist auch die leere Funktionsanwendung $(f)$ von $f$ erlaubt \footnote{Das ergibt dann Sinn, wenn $f$ ein neutrales Element $e \in T$ besitzt, da $(f, as..., (f), bs...)$ mit $\mathrm{normalize}$ zu $(f, as..., bs...) = (f, as..., e, bs...)$ umgeformt wird. Im Folgenden wird von der Existenz eines Neutralen Elementes ausgegangen.}, gäbe es fünf verschiedene Matches $v_p$ für $\paren*{\tilde p, \tilde t}$ mit nicht-kommutativem $f$.

\begin{lemma}~\\
Das Muster $p = (f, \elems {\mathbf x} 1 m)$ hat mit dem Literal $t = (f, \elems a 1 n)$ genau ${m + n - 1}\choose n$ mögliche Matches, wenn $f$ assoziativ aber nicht kommutativ ist $(1)$. Ist $f$ assoziativ und kommutativ gibt es $m^n$ mögliche Matches $(2)$.\\
\end{lemma}

\textbf{Beweis}.\\
$(1)$: Es gibt $m$ möglicherweise leere Abschnitte in den $n$ Parametern von $t$, welche jeweils eine Mustervariable $\mathbf x_i$ matchen. Stellt man eine Abschnittsgrenze mit einem Strich $~|~$ und ein Parameter von $t$ mit einem Stern $~*~$ dar, kann die Aufteilung der Parameter von $t$ über ein String aus $m - 1$ Strichen und $n$ Sternen dargestellt werden. 
Als Beispiel ist $~**|**~$ der String zur Aufteilung von $\tilde t$ aus dem Anfang des Abschnittes zum beschriebenen Match $v_p$.
Es gibt ${m + n - 1}\choose n$ Möglichkeiten die $n$ Sterne auf die ${m + n - 1}$ möglichen Plätze zu verteilen.\\

$(2)$: Jeder der $n$ Parameter von $t$ kann unabhängig der restlichen Parametern zu einer der $m$ Mustervariablen gematcht werden. Insgesamt ergeben sich so $m^n$ Kombinationen.
\hfill $\square$\\

Schon für nicht-kommutative aber assoziative Funktionssymbole $f$ gibt es somit Muster $p$ mit einer Anzahl möglicher Matches, die exponentiell mit der Größe des Literals steigt. Ist ein solches Muster $p$ Teil eines größeren Musters $p'$ und kommen Mustervariablen von $p$ auch in anderen Teilen von $p'$ vor, so ist nicht direkt ersichtlich, wie ein Algorithmus aussehen würde, der in $P$ liegt und bestimmen kann, dass es kein Match für $p'$ mit einem entsprechenden Literal gibt, bzw. das Match findet. Die Existenz eines solchen Algorithmus ist unwarscheinlich: Benanav hat 1987 gezeigt, dass das Problem NP-vollständig ist \cite{NPHardMatching}.
Von dem perfekten Matchalgorithmus wird aus diesem Grund abgesehen. Für viele Spezialfälle sind allerdings bessere Algorithmen möglich. Eine wichtige Klasse solcher Spezialfälle ist die, wo von vorne herein klar ist, welche Mustervariable möglicherweise mehrere Parameter des Literals matchen soll. Würde man etwa bei der Ersetzung der ersten Binomischen Formel eine weitere Mustervariable $\mathbf c$ hinzufügen, um die Binomische Formel auch in einer Summe mit mehr als drei Summanden zu erkennen, kann die Ersetzungregel geschrieben werden als
$$(\texttt{sum}, (\texttt{pow}, \mathbf a, 2), (\texttt{prod}, 2, \mathbf a, \mathbf b), (\texttt{pow}, \mathbf b, 2), \mathbf c) \mapsto (\texttt{sum}, (\texttt{pow}, (\texttt{sum}, \mathbf a, \mathbf b), 2), \mathbf c).$$
$\mathbf c$ ist damit vom Autor des Musters ausschließlich dazu gedacht überbleibende Summanden \glqq aufzusaugen\grqq{}. Dieser Gedanke bleibt aber bisher dem Algorithmus verborgen.
Die in dieser Arbeit gewählte Lösung zur Beschreibung von beliebig vielen Parametern in einem Muster ist im Prinzip schon in Kapitel \ref{secErsteNormalform} eingeführt worden. Die Schreibweise $(f, ts...)$ als kompakte Alternative zu $(f, t_1, \dots, t_n)$ hat viele der zur Beschreibung von Assoziativität gewünschten Eigenschaften. Ferner können so auch Muster mit nicht assoziativen variadischen Funktionssymbolen dargestellt werden. Eine \textit{Multi-Mustervariable} der Form $\mathbf{xs...}$ kann also nicht nur genau einen Parameter in einer Funktionsanwendung matchen, sondern beliebig viele, auch keinen. Um den Matchalgorithmus nicht zu kompliziert zu gestalten, darf jede Multi-Mustervariable auf der linken Seite einer Ersetzungsregel nur höchstens ein Mal vorkommen\footnote{Das macht zudem mehrere Multi-Mustervariablen in der selben Funktionsanwendung eines kommutativen Funktionssymbols auf der linken Seite einer Ersetzungsregel überflüssig. Diese Konstellation ist dementsprechend im Folgenden nicht berücksichtigt.}. Die rigorose Beschreibung des Konzeptes gestaltet sich allerdings mit der bisher eingeführten Ideen schwierig, da eine Multi-Mustervariable nur Teil einer Funktionsanwendung ist und damit auch alleine keinen vollständigen Term repräsentiert. Konnte eine Matchfunktion $v_p \colon X \rightarrow T$ vorher einfach auf die Menge aller Terme abbilden, wäre dies nach Hinzufügen der Multi-Mustervariablen nicht mehr möglich. Entsprechend umständlicher würde auch die Beschreibung der Auswertung eines Musters werden. \\

Formal wird die Multi-Mustervariable damit nicht als echtes neues Symbol in die Menge der Muster aufgenommen, sondern ist lediglich eine vereinfachende Schreibweise, die wie auch vorher immer für eine beliebige Anzahl an Teiltermen steht, in diesem Fall Mustervariablen. Ein Muster mit einer Multi-Mustervariable $\mathbf{xs...}$ repräsentiert also formal unendlich viele konkrete Muster mit konkreten Mustervariablen $\mathbf{x_i}$:
\begin{equation*}
	\begin{split}
			(f, \mathbf{ts...}) = \{&(f), \\
			&(f, \mathbf{x_1}),\\
			&(f, \mathbf{x_1}, \mathbf{x_2}), \\
			&(f, \mathbf{x_1}, \mathbf{x_2}, \mathbf{x_3}), \\
			&\dots \}    		
	\end{split}
\end{equation*}
Für die folgenden Algorithmen dieses Kapitels, sowie der echten Umsetzung, ist es allerdings nicht praktikabel diese Definition anzuwenden. Mit der Restriktion, dass jede Multi-Matchvariable auf der linken Seite einer Ersetzungsregel höchstens ein Mal vorkommen darf, ist eine sehr einfache Verwaltung möglich. Für Funktionsanwendungen kommutativer Funktionssymbole in einem Muster muss lediglich zwischen \emph{enthält eine Multi-Matchvariable} und \emph{enthält keine Multi-Matchvariable} unterschieden werden. Multi-Matchvariablen in Funktionsanwendungen nicht-kommutativer Funktionssymbole haben allerdings eine eindeutige Position. Im Folgenden werden aber auch hier keine weiteren Parameter für Multi-Matchvariablen hinzugefügt. Alternativ wird für jeden tatsächlichen Term in den Parametern einer solchen Funktionsanwendung festgehalten, ob er Nachfolger einer Multi-Matchvariable ist und weiter, ob an dem letzten Parameter noch eine Multi-Matchvariable anschließt. 
Die Ersetzungsregel für die erste Binomische Formel anwendbar auf Summen beliebiger Länge wird demgemäß  geschrieben als:
$$(\texttt{sum}, (\texttt{pow}, \mathbf a, 2), (\texttt{prod}, 2, \mathbf a, \mathbf b), (\texttt{pow}, \mathbf b, 2), \mathbf{cs...}) \mapsto (\texttt{sum}, (\texttt{pow}, (\texttt{sum}, \mathbf a, \mathbf b), 2), \mathbf{cs...}).$$
Die Summe der linken Seite setzt sich für die folgenden Algorithmen dieses Kapitels dennoch nur aus drei Summanden zusammen. Die syntaktisch als Parameter geschriebenen $\mathbf{cs...}$ kommen hier in der linken Seite der Regel nur als Wahrheitswert vor, denn es gilt \glqq Die Summe \emph{enthält eine Multi-Matchvariable}\grqq{}. Lediglich auf der rechten Seite ist relevant, um welche Multi-Matchvariable es sich handelt, sollte es mehrere geben. Dieses Kapitel befasst sich allerdings ausschließlich mit der linken Seite einer Regel.



%.........................................................................
%.........................................................................
%.........................................................................
\section{Kommutative Muster} \label{subsecACMuster}

Die Algorithmen \ref{findMatch}, \ref{rematch}, \ref{findPermutation}, \ref{findDilation} und \ref{findIdentic} bilden zusammen die Grundlage des finalen Matchalgorithmus dieser Arbeit. Der Startpunkt einer Matchsuche ist der Aufruf von $\mathrm{findMatch}$. Hier wird allerdings im Kontrast zu $\mathrm{simpleMatchAlgorithm}$ nicht direkt ein Rekursionsaufruf durchgeführt, sondern abhändig von der Form der vorgefundenen Funktionsanwendung eine entsprechende Strategie für die Suche eines Matches gewählt. Die Algorithmen, die die entsprechenden Strategien implementieren sind $\mathrm{findPermutation}$, $\mathrm{findDilation}$ und $\mathrm{findIdentic}$. Alle hier vorgestellten Strategien nutzen dabei Backtracking um die verschiedenen Möglichkeiten zu testen. Sollte dabei die Notwendigkeit auftreten, für einen bereits gematchten Teil des Musters ein neues Match mit dem selben Literal zu finden, wird in allen drei Strategien $\mathrm{rematch}$ aufgerufen. Dieser Algorithmus ist ähnlich zu $\mathrm{findMatch}$, erwartet dementsprechend allerdings, dass das übergebende Muster $p$ bereits mit dem übergebenen Literal $t$ gematcht ist. Die eigentliche Arbeit wird bei $\mathrm{rematch}$ allerdings erneut an die Suchstrategien abgegeben. Da der Startpunkt dort allerdings davon abhängig ist, ob bereits eine bestimmte Zuordnung der Parameter als erfolgreich matchend festgehalten ist oder nicht, wird diese Information als letztes Argument jeder Strategie mit übergeben.


Im Grundaufbau funktionieren alle Strategien gleich. Die Parameter $\elems p 0 {m-1}$ des Musters $p$ werden in der vorliegenden Reihenfolge mit den Parametern $\elems t 0 {n-1}$ des Literals $t$ gematcht. kann für den aktellen Parameter $p_i$ kein Match mehr gefunden werden, wird probiert die vorhergehenden Parameter $\elems p 0 {i-1}$ neu zu matchen, beginnend mit $p_{i-1}$. Mit welchen Parametern $t_k$ ein Match dabei erlaubt ist, ist nach Strategie unterschiedlich. Am stärksten eingeschränkt ist $\mathrm{findIdentic}$. $p_i$ kann dort nur mit $t_k$ gematcht werden, wenn $k = i$ gilt\footnote{Algorithmus \ref{simpleMatchAlgorithm} hat ausschließlich auf diese Weise nach einem Match gesucht}. Das andere Extrem stellt $\mathrm{findPermutation}$ da. Hier kann jedes $p_i$ mit jedem $t_k$ gematcht werden, vorrausgesetzt $t_k$ ist noch nicht mit einem Parameter aus $\elems p 0 {i-1}$ gematcht. In der Freiheit dazwischen steht $\mathrm{findDilation}$, welche die Reihenfolge der $p_i$ untereinander gleich halten muss, jedoch eine Lücke beliebiger Länge zwischen $p_{i-1}$ und $p_i$ erlaubt, sofern im Muster an dieser Stelle eine Multi-Matchvariable steht\footnote{Wie in Abschnitt \ref{subsecMulti} erörtert, treten diese hier nicht als echte Parameter auf.}.

\begin{algorithm}
\DontPrintSemicolon
\caption{$\mathrm{findMatch} \colon M \times T \rightarrow \mathit{Bool}$}\label{findMatch}
\KwIn {$p \in M$, $t \in T$}

\If {$p \in X$ \KwAnd $p$ bindend}{
	merke: $v_p~p = t$\;
	\Return {$\mathrm{true}$}
}
\ElseIf {$p \in X$ \KwAnd $p$ gebunden}{
	\Return {$v_p~p = t$}
}
\ElseIf {$p \in C \setminus X$} {
	\Return {$p = t$}
}
\ElseIf {$p = (f, \elems p 0 {m-1})$ \KwAnd $t = (f, \elems t 0 {n-1})$}{
	\If {$u~f$ kommutativ} {
		\Return {$\mathrm{findPermutation}(p, t, \mathrm{false})$}
	}
	\ElseIf {$\elems t 1 n$ enthalten Multi-Matchvariablen} {
		\Return {$\mathrm{findDilation}(p, t, \mathrm{false})$}
	}
	\ElseIf {$m = n$} {
		\Return {$\mathrm{findIdentic}(p, t, \mathrm{false})$}
	}
}
\Return{$\mathrm{false}$}  
\end{algorithm}

\begin{algorithm}
\DontPrintSemicolon
\caption{$\mathrm{rematch} \colon M \times T \rightarrow \mathit{Bool}$}\label{rematch}
\KwIn {$p \in M$, $t \in T$}
\If {$p = (f, \elems p 0 {m-1})$ \KwAnd $t = (f, \elems t 0 {n-1})$} {
	\If {$u~f$ kommutativ} {
		\Return {$\mathrm{findPermutation}(p, t, \mathrm{true})$}
	}
	\ElseIf {$\elems t 1 n$ enthalten Multi-Matchvariablen} {
		\Return {$\mathrm{findDilation}(p, t, \mathrm{true})$}
	}
	\ElseIf {$m = n$} {
		\Return {$\mathrm{findIdentic}(p, t, \mathrm{true})$}        
	}
}        
\Return {$\mathrm{false}$}  
\end{algorithm}

Abweichend von bisherigen Algorithmen wird von hier an im Pseudocode nicht mehr jede tatsächlich notwendige Information explizit übergeben. Als Beispiel muss $v_p$ nach wie vor von jedem Funktionsaufruf aktualisiert werden, ist aber im Pseudocode der Algorithmen \ref{findMatch}, \ref{findPermutation}, \ref{findDilation}, etc. nicht länger explizit in Parameterliste oder als Rückgabewert erwähnt.
Anstelle der konkreten Zuweisung eines Wertes zu einem Namen, dargestellt duch den Pfeil nach links \glqq $\leftarrow$\grqq{}, wird die Veränderung einer solchen nicht explizit erwähnten Datenstruktur nur mit dem Wort \glqq merke\grqq{} dargestellt.

\subsection{findMatch und rematch}
Algorithmus \ref{findMatch} ist in der Struktur ähnlich zu $\mathrm{simpleMatchAlgorithm}$. Neben dem Auslagern der Rekursionsaufrufe in die verschiedenen Matchstrategien, besteht ein Unterschied im Umgang mit Mustervariablen. Für $\mathrm{simpleMatchAlgorithm}$ wird getestet, ob der Funktionswert $v_p~\mathbf x$ für eine Mustervariable $\mathbf x$ bereits definiert ist, wenn $\mathbf x$ angetroffen wird und herausgefunden werden muss ob die Instanz bindend oder gebunden ist. Das reicht für $\mathrm{findMatch}$ nicht, da dieser Algorithmus auch funktionieren muss, wenn die verschiedenen Matchstrategien ein Backtracking beinhalten, d.h., $\mathrm{rematch}$ aufrufen.
Der Algorithmus $\mathrm{rematch}$ ist fast identisch zur unteren Hälfte von $\mathrm{findMatch}$, ruft die verschiedenen Matchstrategien allerdings mit $\mathrm{true}$ als letzen Parameter auf, was bedeutet, dass direkt zum Backtracking gesprungen wird.



\subsubsection {findIdentic}
\begin{algorithm}
\DontPrintSemicolon
\caption{$\mathrm{findIdentic} \colon M \times T \times \mathit{Bool} \rightarrow \mathit{Bool}$}\label{findIdentic}
\KwIn {$p = (f, \elems p 0 {n-1}) \in M$, $t = (f, \elems t 0 {n-1}) \in T$, $\mathit{starteGematcht} \in \mathit{Bool}$}
\Let {$i \leftarrow 0$}\;
\If {$\mathit{starteGematcht}$} {
	$i \leftarrow n$\;
	\Goto \texttt{\ref{backtrackRematchMuster}}\;
}
\Loop {} {
	\nlset{matche $p_i$} \label{backtrackMatchMuster}
	\While {$\mathrm{findMatch}(p_i, t_i)$} {
		$i \leftarrow i + 1$\;
		\lIf {i = n} {\Return {$\mathrm{true}$}}    
	}
	\nlset{zurück} \label{backtrackRematchMuster}
	\DoWhile {$\mathrm{not}$ $\mathrm{rematch}(p_i, t_i)$} { 
		\lIf {i = 0} {\Return {$\mathrm{false}$}}
		$i \leftarrow i - 1$\;
	}    
	$i \leftarrow i + 1$\;
}
\end{algorithm}


Als Strategie mit den wenigsten Freiheiten ist die Umsetzung von $\mathrm{findIdentic}$ die kürzeste. Die Laufvariable $i$ steht gleichzeitig als Index für die Parameter von Muster und Literal. Wenn der Aufruf aus $\mathrm{findMatch}$ erfolgt, wird mit $i = 0$ gestartet und in Abschnitt \texttt{\ref{backtrackMatchMuster}} versucht für alle $i$ bis $n-1$ $p_i$ mit $t_i$ zu matchen. Sollte das  für ein $i$ fehlschlagen, besteht die Hoffnung, dass einer der Parameter $p_j \in \{\elems p 0 {i-1}\}$ anders als bisher mit $t_j$ gematcht werden kann, wodurch dann das Match von $p_i$ mit $t_i$ ermöglicht wird. Gefunden wird $p_j$ in Abschnitt \texttt{\ref{backtrackRematchMuster}}. 
Soll das gesamte Muster neu gematcht werden, startet $\mathrm{findIdentic}$ bei einem Aufruf durch $\mathrm{rematch}$ deswegen bei \texttt{\ref{backtrackRematchMuster}} und $i = n$.

\begin{lemma}\label{lemKomplexitaetFindPermutation}~\\
Die Laufzeitkomplexität von Algorithmus \ref{findIdentic} bei der Suche eines Matches für ein Muster $p = (f, \elems {\mathbf x} 0 {n-1})$ mit einem Literal $t = (f, \elems t 0 {n-1})$ ist in $\mathcal O(n)$, wenn jeder Parameter $t_i$ von $t$ nur $\mathcal O(1)$ Konstantensymbole und Funktionssymbole besitzt.
\end{lemma}

\textbf{Beweis}.\\
Der Ausdruck $\mathrm{not}$ $\mathrm{rematch}(pi , ti)$ ist für kein $i$ wahr, da $\mathrm{rematch}$ für Mustervariablen immer $\mathrm{false}$ zurückgibt. Die äußere Schleife wird also nur exakt ein Mal durchlaufen. Sowohl ein Durchlauf der \textbf{while}-Schleife, als auch ein Durchlauf der \textbf{do-while}-Schleife ist in $\mathcal O(1)$, da jedes $t_i$ nur $\mathcal O(1)$ Teilterme hat, bzw. $\mathrm{rematch}(pi , ti)$ für Mustervariblen $p_i$ direkt $\mathrm{false}$ zurückgibt. Entweder wird der Abschnitt \texttt{\ref{backtrackMatchMuster}} exakt $n$ Mal abgelaufen und $\mathrm{true}$ zurückgegeben oder $n' < n$ Mal abgelaufen, woraufhin auch Abschnitt \texttt{\ref{backtrackRematchMuster}} $n'$ Mal abgelaufen wird, bis $\mathrm{false}$ zurückgegeben wird. 
\hfill $\square$\\



\subsubsection {findPermutation}
\begin{algorithm}
\DontPrintSemicolon
\caption{$\mathrm{findPermutation} \colon M \times T \times \mathit{Bool} \rightarrow \mathit{Bool}$}\label{findPermutation}
\KwIn {$p = (f, \elems p 0 {m-1}) \in M$, $t = (f, \elems t 0 {n-1}) \in T$, $\mathit{starteGematcht} \in \mathit{Bool}$}
\Let {$i \leftarrow 0$, $k \leftarrow 0$}\;
\If {$\mathit{starteGematcht}$} {
	$i \leftarrow m$\;
	\Goto \texttt{\ref{permutationRematchMuster}}\;
}
 \If {$m > n$} {
	\Return {$\mathrm{false}$}
 }
 \nlset{matche $p_i$}\label{PermutationHauptschleifenbeginn}
 \While {$i < m$} {
	\While {$k < n$} {
		\If {$t_k$ ist mit keinem Parameter von $p$ gematcht} {
			\If {$\mathrm{findMatch}(p_i, t_k)$} {
				\Goto \texttt{\ref{permutationNaechstesMuster}}\;
			}
		}
		$k \leftarrow k + 1$\;
	}
	\nlset{zurück}\label{permutationRematchMuster}
	\If {$i = 0$} {
		\Return {$\mathrm{false}$}
	}
	$i \leftarrow i - 1$\;
	{$k \leftarrow k'$ aus \glqq $p_{i}$ ist mit $t_{k'}$ gematcht\grqq{}}\;
	\If {not $\mathrm{rematch}(p_{i}, t_{k})$} {
		merke: $p_{i}$ ist nicht mehr mit $t_{k}$ gematcht\;
		$k \leftarrow k + 1$\;
		\Goto \texttt{\ref{PermutationHauptschleifenbeginn}}\;
	} 
	\nlset{weiter}\label{permutationNaechstesMuster}
	merke: $p_i$ ist mit $t_k$ gematcht\;
	$i \leftarrow i + 1$\;
	$k \leftarrow 0$\;    
 }
 \Return {$p$ enthält eine Multi-Matchvariable \KwOr alle $t_k$ wurden gematcht}
\end{algorithm}

Algorithmus \ref{findPermutation} probiert das Muster einer kommutativen Funktionsanwendung $p$ auf ein Literal $t$ der gleichen Form zu matchen. Die beiden Laufvariablen $i$ und $k$ sind Index für die Muster Parameter $\elems p 0 {m-1}$, bzw. der Parameter des Literals $\elems t 0 {n-1}$. Für die Suche eines Matches wird zuerst versucht $p_0$ mit $t_0$ zu matchen. Schlägt das fehl, wird $k$ hochgezählt, bis $p_0$ ein $t_k$ matchen kann oder jedes $t_k$ getestet wurde. Im erfolglosen Fall wird die Suche beendet, da für $p_0$ alle verfügbaren Freiheitsgerade getestet wurden. Wurde $p_0$ erfolgreich mit $t_k$ gematcht, wiederholt sich der Prozess für $p_1$, mit der Ausname, dass $t_k$ jetzt nicht mehr als Matchkandidat zur Verfügung steht. Wird ein Parameter gefunden, der $p_1$ matcht, wird $i = 3$ gesetzt und der Prozess wiederholt sich für $p_3$. Falls die Suche für $p_1$ in dem Durchlauf allerdings erfolglos war, heißt es allerdings nicht, dass kein Match von $p$ und $t$ möglich ist. Es besteht die Option, dass $p_0$ mit $t_k$ noch auf eine andere Weise als die bisherige gematcht werden kann. Beinhaltet $p_1$ Mustervariablen, die in $p_0$ bindend vorkommen, eröffnet ein ändern der Bindung möglicherweise neue Matchmöglichkeiten für $p_1$. Aus dem Grund wird im Abschnitt \texttt{\ref{permutationRematchMuster}} zuerst versucht $p_0$ mit $t_k$ zu rematchen. Sollte das fehlschlagen, ist es möglich, dass $p_0$ noch mit Parametern von $t$ gematcht werden kann, die nach $t_k$ aufgelistet sind, was wiederum $p_1$ erlauben würde, ein Match mit $t_k$ zu testen. Auch diese Option wird ausprobiert.
Wurde für alle $p_i$ ein Match gefunden, so ist ganz $p$ mit ganz $t$ gematcht, falls gleichzeitig alle $t_k$ gematcht sind oder $p$ eine Multi-Matchvariable enthält. Sollte keiner der beiden Fälle eintreten, ist an dieser Stelle kein Match von $p$ und $t$ möglich. 


\begin{lemma}\label{lemKomplexitaetFindPermutation}~\\
Die Laufzeitkomplexität von Algorithmus \ref{findPermutation} bei der Suche eines Matches für ein Muster $p = (f, \elems {\mathbf x} 0 {m-1})$ mit einem Literal $t = (f, \elems t 0 {n-1})$ ist in $\mathcal O(n^m)$, wenn jeder Parameter $t_k$ von $t$ nur $\mathcal O(1)$ Konstantensymbole und Funktionssymbole besitzt.
\end{lemma}

\textbf{Beweis}.\\
Der Beweis erfolgt als Induktion über $m$.
Für den Induktionsanfang mit $m = 1$ muss $\mathbf x_0$ höchstens mit allen $n$ Parametern des Literals verglichen werden. Unabhängig davon, ob die Instanz von $\mathbf x_0$ bindend oder gebunden ist, terminiert $\mathrm{findMatch}(\mathbf x_0, t_k)$ in $\mathcal O(1)$, da alle $t_k$ in ihrer Größe beschränkt sind. \\
Im allgemeinen Fall kann die Anwendung von $\mathrm{findPermutation}$ mit $p$ und $t$ auf höchstens $m$ Anwendungen des Algorithmus mit Mustergröße $m-1$ zurückgeführt werden. Erneut kann $\mathbf x_0$ potenziell mit jedem $t_k$ matchen. Das Match mit $t_k$ erfolgt wie im Indunktionsanfang beschrieben in $\mathcal O (1)$. Die anschließende Suche nach Matches für $x_i$ mit $i > 0$ ist in der Komplexität äquivalent zu einem neuen Aufruf von $\mathrm{findPermutation}$ mit dem Muster $p' = (f, \elems {\mathbf x} 1 {m-1})$ und dem Literal $t' = (f, \elems t 0 {k-1}, t_k', \elems t {k+1} {n-1})$, wobei $t_k'$ ein spezieller Wert ist, mit dem ein Match zu jedem Muster in $\mathcal O (1)$ abgelehnt wird \footnote{Der Wert von $t_k'$ ist sonst nicht notwendig, da in Bereich \texttt{\ref{PermutationHauptschleifenbeginn}} durch die if-Abfrage der problematische Matchversuch mit $t_k$ umgangen wird.}. Gibt dieser Aufruf $\mathrm{false}$ zurück, gibt auch $\mathrm{rematch}(\mathbf x_0, t_k)$ in $\mathcal O (1)$ $\mathrm{false}$ zurück. Der Übergang von $t_k$ zu $t_{k-1}$ erfolgt ebenfalls in $\mathcal O (1)$. Für jedes der bis zu $n$ Matches von $\mathbf x_0$ mit einem $t_k$ treten damit Laufzeitkosten von $\mathcal O (1)$ außerhalb der Rekursion auf. Jeder der $n$ Rekursionsaufrufe hat nach Induktionshypothese eine Laufzeit in $\mathcal O (n^{m-1})$. Insgesamt ergibt sich so also eine Laufzeit in $n \cdot \mathcal O (1) \cdot \mathcal O (n^{m-1}) = \mathcal O (n^m)$.
\hfill $\square$\\


\subsubsection {findDilation}
\begin{algorithm}
\DontPrintSemicolon
\caption{$\mathrm{findDilation} \colon M \times T \times \mathit{Bool} \rightarrow \mathit{Bool}$}\label{findDilation}
\KwIn {$p = (f, \elems p 0 {m-1}) \in M$, $t = (f, \elems t 0 {n-1}) \in T$}
\Let $i \leftarrow 0$, $k \leftarrow 0$\;
\If {$\mathit{starteGematcht}$} {
	$i \leftarrow m$\;
	\Goto \texttt{\ref{dilationRematchLastNeedle}}\;
}
\If {$m = 0$} {
	\Return {$\mathrm{true}$}
}
\If {$n = 0$} {
	\Return {$\mathrm{false}$}
}
\nlset{matche $p_i$} \label{dilationMatchCurrentNeedle} 
\If {$k < n$} {
	\DoWhile {$p_i$ ist Nachfolger einer Multi-Matchvariable \KwAnd $k < n$} { 
		\If {$\mathrm{findMatch}(p_i, t_k)$} {
			\Goto \texttt{\ref{dilationPrepareNextNeedle}}\;
		}
		$k \leftarrow k + 1$\;
	}
}
\nlset{zurück} \label{dilationRematchLastNeedle} 
\While {$i > 0$} {
	$i \leftarrow i - 1$\;
	{$k \leftarrow k'$ aus \glqq $p_{i}$ ist mit $t_{k'}$ gematcht\grqq{}}\;
	\If {$\mathrm{rematch}(p_i, t_k)$} {
		\Goto \texttt{\ref{dilationPrepareNextNeedle}}\;
	}
	\ElseIf {$p_i$ ist Nachfolger einer Multi-Matchvariable} {
		$k \leftarrow k + 1$\;
	\Goto \texttt{\ref{dilationMatchCurrentNeedle}}\;
	}
}
\Return{$\mathrm{false}$}\;
\nlset{weiter} \label{dilationPrepareNextNeedle} 
merke: $p_i$ ist mit $t_k$ gematcht\;
$i \leftarrow i + 1$\;
$k \leftarrow k + 1$\;
\If {$i < m$} {
	\Goto \texttt{\ref{dilationMatchCurrentNeedle}}\;
}
\ElseIf {$k < n$ \KwAnd nach $p_{m-1}$ folgt keine Multi-Matchvariable} {
	\Goto \texttt{\ref{dilationRematchLastNeedle}}\;
}
\Return {$\mathrm{true}$}
\end{algorithm}


Algorithmus \ref{findDilation} teilt die Musterparameter $\elems p 0 {m-1}$ in Blöcke der Form $\elems p i j$ ein. Ein solcher Block muss Elementweise einen Block $\elems t k {k + j - i}$ der Parameter des Literals matchen. Die Aufteilung der Musterparameter in Blöcke ist dabei fest. Wenn zwischen $p_{i-1}$ und $p_{i}$ eine Multi-Matchvariable liegt, ist dort eine Blockgrenze. Aus dem Grund wird in \texttt{\ref{dilationMatchCurrentNeedle}} nur mehr als ein Schleifendurchlauf erlaubt, wenn $p_i$ einen neuen Block beginnt, $t_k$ also nicht durch den bereits gematchen Beginn des Blockes festgelegt ist. 
Konnte $p_i$ nicht gematcht werden, wird  $i$ im Bereich \texttt{\ref{dilationRematchLastNeedle}} so lange heruntergezählt, bis entweder $\mathrm{rematch}$ erfolgreich ist oder $p_i$ das erste Element des aktuellen Blocks ist. Besonderes Verhalten tritt erneut für $i = 0$ auf. War jeder Matchversuch für $p_0$ erfolglos, gibt es keine Möglichkeit mehr die Teilterme von $p$ so zu \glqq strecken\grqq{}\footnote{daher auch der Name $\mathrm{findDilation}$}, dass ein Match mit $t$ gefunden werden kann, ähnlich der Situation für $p_0$ in $\mathrm{findPermutation}$. Anders als bei $\mathrm{findPermutation}$ wird $k$ im Bereich \texttt{\ref{dilationPrepareNextNeedle}} allerdings hochgezählt, da $p_i$ nie ein Literal $t_k$ matchen darf, wenn $p_{i-1}$ bereits mit $t_l$ gematcht ist und $l > k$.


\begin{lemma}\label{lemKomplexitaetDilation}~\\
Die Laufzeitkomplexität von Algorithmus \ref{findDilation} bei der Suche eines Matches für ein Muster $p$  mit einem Literal $t = (f, \elems t 0 {n-1})$ ist in $\mathcal O(n^m)$, wenn jeder Parameter $t_k$ von $t$ nur $\mathcal O(1)$ Konstantensymbole und Funktionssymbole besitzt und $p$ eine Funktionsanwendung von $f$ auf $m$ Mustervariablen $\elems {\mathbf x} 0 {m-1}$, 
wobei jede Mustervariable $\mathbf x_i$ Nachfolger einer Multi-Matchvariable ist und auf $x_{m-1}$ eine Multi-Matchvariable folgt. Für $m = 3$ gilt also $p = (f, \mathbf{as...}, \mathbf x_0, \mathbf{bs...}, \mathbf x_1, \mathbf{cs...}, \mathbf x_2, \mathbf{ds...})$.
\end{lemma}

\textbf{Beweis}.\\
Sei $D(m, n)$ die asymptotische Laufzeit von Algorithmus \ref{findDilation}. Erneut wird ein Induktionsbeweis über $m$ beschrieben.
Mit $m = 1$ ist der Fall identisch zu dem Indunktionsanfang des Beweises von Lemma \ref{lemKomplexitaetFindPermutation}, es gilt $D(1, n) = \mathcal O(n)$. 

Für $m > 1$ wird $D(m, n)$ auf $D(m-1, n)$ zurückgeführt. Für jedes $k$ muss nach erfolgreichem Match von $\mathbf x_0$ mit $t_k$ versucht werden, die restlichen Mustervariablen $\elems {\mathbf x} 1 {m-1}$ in den restlichen Argumenten $\elems {t} {k+1} {n-1}$ des Literals $t$ zu matchen. Das entspricht einem erneuten Aufruf von $\mathrm{findDilation}$ mit neuem Muster $p'$ ohne $\mathbf x_0$ und neuem Liteal $t' = (f, \elems t {k+1}, {n-1})$.
Im rechenaufwändigsten Fall passiert das für jedes $k$.
$$D(m, n) = \sum_{k = 0}^{n-1} \paren*{\mathcal O(1) + D(m-1, n - k - 1)}$$
Trotz der vorgegebenden Reihenfolge der Matches von Musterparametern im Literal, folgt die selbe Komplexitätsabschätzung wie für $\mathrm{findPermutation}$.
\begin{equation*}
	\begin{split} 
		D(m, n) 
		&= \sum_{k = 0}^{n-1} \paren*{\mathcal O(1) + D(m-1, n - k - 1)}\\ 
		&< \sum_{k = 0}^{n-1} \paren*{\mathcal O(1) + D(m-1, n)}\\
		&= \mathcal O(n) + \mathcal O(n) \cdot D(m-1, n)\\
		&= \mathcal O(n) \cdot D(m-1, n)\\
		&= \mathcal O(n^m)
	\end{split}
\end{equation*}
\hfill $\square$\\


%.........................................................................
%.........................................................................
%.........................................................................
\section{Bessere Laufzeit für kommutative Muster} \label{subsecCMuster}

Benanav zeigte 1987, dass auch das Matchproblem mit einem kommutativen Funktionssymbol NP-vollständig ist \cite{NPHardMatching}. Dennoch kann die Laufzeit von $\mathrm{findPermutation}$ für bestimmte Arten von Mustern verbessert werden. In diesem Abschnitt wird das versucht, indem a priori ausgeschlossen wird, dass bestimmte Reihenfolgen von Musterparametern erfolgreich matchen können. Diese Reihenfolgen müssen dann im Algorithmus nicht mehr geprüft werden.
Vorraussetzung dafür wird sein, dass sowohl Muster als auch Literal mit $\mathrm{normalize}$ aus Kapitel \ref{secErsteNormalform} normalisiert werden. Insbesondere die Sortierung nach der Relation $<$ aus Definition \ref{defOrdnungKleiner} ist relevant.
Bestimmte Muster mit kommutativen Funktionssymbolen können mit dieser einfachen Maßname bereits in linearer Zeit mit einem Literal abgeglichen werden. Besteht ein Muster etwa aus der Anwendung eines komutativen Funktionssymbols $f$ auf $m$ Parameter $\elems p 1 m$ und sind die Parameter $p_i$ ausschließlich Funktionsanwendungen paarweise verschiedener Funktionssymbole, ist garantiert, dass die Parameter eines normalisierten Literals für ein Match in der selben Reihenfolge liegen müssen. Wenn auch alle Teilmuster $p_i$ dieser Struktur folgen, bzw. nicht kommutativ sind, wird ein Match zuverlässig bereits mit $\mathrm{findIdentic}$ (Algorithmus \ref{findIdentic}) gefunden\footnote{ohne Berücksichtigung von Matches, welche durch Assoziativität ermöglicht würden}. Das liegt daran, dass für alle $i, j \in \{1, \dots, m\}, i < j$ gilt, dass $p_i$ ausschließlich Literale matchen kann, die vor jedes Literal sortiert werden, welches mit $p_j$ gematcht werden könnte. Diese Idee der Ordnung von Mustern wird im Folgenden ausgeführt.


\begin{definition}~\\
Man sagt dass Muster $p_1$ ist \emph{stark kleiner} als das Muster $p_2$ oder $p_1 \prec p_2$, wenn für alle Matchfunktionen $v_p$ gilt, dass $\mathrm{lit}~(p_1, v_p) < \mathrm{lit}~(p_2, v_p)$. Gilt immer $\mathrm{lit}~(p_1, v_p) \leq \mathrm{lit}~(p_2, v_p)$, sagt man $p_1$ ist \emph{stark kleiner-gleich} als $p_2$ oder $p_1 \preceq p_2$.
\end{definition}

\begin{beispiel}~\\
Wenn $\texttt{sin} < \texttt{cos}$, gilt $p_1 \prec p_2$ für $p_1 = (\texttt{pow}, (\texttt{sin}, \mathbf x), 2)$ und $p_2 = (\texttt{pow}, (\texttt{cos}, \mathbf y), 2)$. Im Kontrast sind die Muster $\hat p_1 = (\texttt{pow}, \mathbf x, 2)$ und $\hat p_2 = (\texttt{pow}, \mathbf y, 3)$ zueinander nicht stark geordnet. Gezeigt werden kann das mit Literalten $t_1$, $t_2$ und $t_3$, wenn ${t_1 < t_2 < t_3}$ gilt, aber $\hat p_1$ sowohl $t_1$ als auch $t_3$ matchen kann und $\hat p_2$ das Literal $t_2$ matchen kann. Mit $1 < 2 < 3$ erfüllen $t_1 = (\texttt{pow}, 1, 2)$, $t_2 = (\texttt{pow}, 2, 3)$ und $t_3 = (\texttt{pow}, 3, 2)$ die Bedingungen.
\end{beispiel}

\begin{lemma}~\\
Sei $p$ ein Muster, welches mindestens die Mustervariable $\mathbf x$ enthält und $q$ ein Muster bestehend nur aus der Mustervariable $\mathbf y$.
Wenn es zwei Konstantensymbole $c_1 \neq c_2 \in C$ ohne Auswertung gibt, gilt weder $p \preceq q$, noch $q \preceq p$.
\end{lemma}

\textbf{Beweis}.\\
Sei $v_p^{(i)}$ eine Matchfunktion mit $v_p^{(i)}~\mathrm x = c_i$ und sonst von $i$ unabhängigen Funktionswerten. Die Literale $t_1 = \mathrm{lit}~(p, v_p^{(1)})$ und $t_2 = \mathrm{lit}~(p, v_p^{(2)})$ sind von einander verschieden und können beide sowohl $p$ als auch $q$ matchen. 
\hfill $\square$\\

Mit ähnlichen Argumenten 


Einfach zu sehen ist, dass nur sehr wenige Muster zueinander stark geordnet sind, wenn die allgemeinste Form des Matches erlaubt ist. Problematisch ist dabei vor allem $\mathrm{combine}$ (Algorithmus \ref{combine}) als Teil von $\mathrm{normalize}$. Dadurch wird es möglich Muster mit Funktionsanwendungen und Literale bestehend nur aus einer Konstanten zu matchen. Da es aufwändig ist überhaupt erst zu bestimmen, welche Muster Matches dieser Art erlauben, wird $\mathrm{normalize}$ hier abweichend von Kapitel \ref{secErsteNormalform} ohne $\mathrm{combine}$ angenommen. Nur Assoziativität und Kommutativität werden berücksichtigt.

\begin{lemma}~\\
Für folgende Formen von normalisierten Mustern $p$ und $q$ gilt $p \prec q$:
\begin{enumerate}
	\item{$p$ und $q$ enthalten keine Mustervariablen und $p < q$} \label{itemStarkKleiner1}
	
	\item{$p$ ist ein Konstantensymbol aber keine Mustervariable und $q$ ist eine Funktionsanwendung}  \label{itemStarkKleiner2}
	
	\item{$p$ und $q$ sind Funktionsanwendungen verschiedener Funktionssymbole $f$ und $g$ mit $f < g$}  \label{itemStarkKleiner3}
		
	\item{$p = (f, \elems p 0 {m-1})$ und $q = (f, \elems q 0 {n-1})$ sind Funktionsanwendungen des selben nicht-kommutativen Funktionssymbols $f$ und einer der folgenden Punkte trifft zu.
	\begin{enumerate}
		\item{$m < n$, $\forall j \in \{0, \dots, m-1\} \colon p_j = q_j$, $p$ hat keine Multi-Mustervariablen in seinen Argumenten und wenn $q$, dann erst nach $q_m$}
		\item{$\exists i < min\{n, m\} \colon p_i \prec q_i$, $\forall j \in \{0, \dots, i  - 1\} \colon p_j = q_j$ und weder $p$ noch $q$ haben Multi-Mustervariablen in ihren Argumenten vor $p_i$ bzw. $q_i$}
	\end{enumerate}
	} \label{itemStarkKleiner4}
	
	\item{$p = (f, \elems p 0 {m-1})$ und $q = (f, \elems q 0 {n-1})$ sind Funktionsanwendungen des selben kommutativen Funktionssymbols $f$, keine mit Multi-Mustervariablen unter ihren Argumenten und einer der folgenden Punkte trifft zu.
	\begin{enumerate}
		\item{$m < n$ und $\forall j \in \{0, \dots, m-1\} \colon p_j = q_j$}
		\item{$\forall i \in \{0, \dots, m-1\}, j \in \{0, \dots, n-1\} \colon p_i \prec q_j \lor p_i = q_j$}
	\end{enumerate}
	} \label{itemStarkKleiner5}
	
\end{enumerate}
\end{lemma}

\textbf{Beweis}~\\
In allen Fällen wird eine Mustervariable nur an Punkten erlaubt, die nicht zur Bestimmung der Ordnung von $p$ und $q$ unter der Relation $<$ beitragen, bzw die für kein Literal anstelle der Mustervariable die Ordnung der normalisierten Terme beeinflussen würden. Trivial ist das für Fall \ref{itemStarkKleiner1}.

Für die restlichen Fälle muss klar sein, dass ein Literal der Form $(f, xs...)$ auch nach Normalisierung diese Form behält. Die Reihenfolge der Argumente und damit auch ob $f$ kommutativ ist, spielen keine Rolle. Das Normalisieren mehrerer geschachtelter Anwendungen des selben assoziativen Funktionssymbols $f$ entfernt nie die äußerste Funktionsanwendung, die Struktur bleibt also auch so erhalten.
Fall \ref{itemStarkKleiner2} und Fall \ref{itemStarkKleiner3} sind ist damit bewiesen.

Mit der selben Argumentation bleibt $f$ auch in den Fällen \ref{itemStarkKleiner4} und \ref{itemStarkKleiner5} immer das äußerste Funktionssymbol erhalten. Da die Reihenfolge der Argumente in Fall \ref{itemStarkKleiner4} auch nach Auswertung der Musterinterpretation gleich bleibt, 

\hfill $\square$\\


Für folgende Formen von Mustern $p$ und $q$ gilt $p \preceq q$:
\begin{enumerate}
	\item{$p \prec q$}
	
	\item{$p = q$}
	
	\item{kontextabhängig, wenn $p = \mathbf x$ und $q = \mathbf y$ Mustervariablen sind und das gesamte Muster $r$, in dem sowohl $p$ als auch $q$ sich befinden \emph{symmetrisch} zu $\mathbf x$ und $\mathbf y$ ist. Symmetrisch ist $r$ dann, wenn $\mathbf x$ und $\mathbf y$ in ganz $r$ getauscht werden können um $r'$ zu erhalten und die normalisierte Form von $r'$ identisch zu $r$ ist.}
\end{enumerate}

\BFred{TODO: Beweise $p \preceq q$ in beschriebenen Fällen}

Beispiele für symmetrische Muster sind beide Seiten der ersten Binomischen Formel, wobei $\mathbf a$ symmetrisch zu $\mathbf b$ ist:
$$(\texttt{sum}, (\texttt{pow}, \mathbf a, 2), (\texttt{prod}, 2, \mathbf a, \mathbf b), (\texttt{pow}, \mathbf b, 2), \mathbf {c...}) \mapsto (\texttt{sum}, (\texttt{pow}, (\texttt{sum}, \mathbf a, \mathbf b), 2), \mathbf {c...})$$


\begin{lemma}\label{lemTransitivStark}~\\
$\prec$ und $\preceq$ sind transitiv $(1)$ und für $p, q, r \in M$  gilt $p \prec q \preceq r \implies p \prec r$ sowie $p \preceq q \prec r \implies p \prec r$ $(2)$. 
\end{lemma}

\textbf{Beweis}~\\
\BFred{TODO: beweise Lemma \ref{lemTransitivStark}}

\section{Value-Matchvariablen}
\BFred{Je nachdem, wie schnell ich den Rest zu einer Runden Gesamtstruktur bekomme, werde ich auch noch ein Wort über Muster verlieren, die den combine Teil von normalize zumindest teilweise berücksichtigen.}




\section{Termersetzungssystem} \label{subsecTermersetzungssystem}

Das Ziel dieses Abschnittes ist erneut die Normalisierung eines Terms. Im Unterschied zu Kapitel \ref{secErsteNormalform} werden die Ersetzungsregeln hier nicht im Algorithmus festgelegt, sondern erst als Parameter mit übergeben. 

Ist eine Ersetzungsregel auf das übergebende Literal $l = l^{(0)}$ oder ein Teil dessen anwendbar, so wird das Ergebnis der Ersetzung $l^{(1)}$ genannt. Auf dem selben Weg kann aus $l^{(1)}$ der Term $l^{(2)}$ erzeugt werden oder allgemeiner aus $l^{(i)}$ der Term $l^{(i+1)}$. Ist auf keinen Teil von $l^{(n)}$ mehr eine Regel anwendbar, wird $l^{(n)}$ als \emph{Normalform} von $l$ zu den übergebenden Ersetzungsregeln bezeichnet. 

Es ist möglich, dass ein Literal $l$ mit einer bestimmten Regelmenge mehr als nur eine Normalform besitzt. Eine einfache Regelmenge mit dieser Eigenschaft besteht aus zwei Regeln mit identischer linker Seite aber unterschiedlicher rechter Seite. Bestimmte Regeln können allerdings auch in Isolation mehrere Normalformen produzieren. Ein Beispiel ist die folgende Regel, welche $\mathbf x$ ausklammert:
$$(\texttt{sum}, \mathbf x, (\texttt{prod}, \mathbf x, \mathbf{ys...}), \mathbf{zs...}) 
\mapsto (\texttt{sum}, (\texttt{prod}, \mathbf x, (\texttt{sum}, 1, (\texttt{prod}, \mathbf{ys...})) , \mathbf{zs...})$$

Für das Literal $l = (\texttt{sum}, a, b, (\texttt{prod}, a, b))$ existiert sowohl die Normalform 
$$l' = (\texttt{sum}, b, (\texttt{prod}, a, (\texttt{sum}, 1, b))),$$ 
als auch 
$$l'' = (\texttt{sum}, a, (\texttt{prod}, b, (\texttt{sum}, 1, a))).$$

Das ist an dieser Stelle allerdings nicht neu: Algorithmus \ref{rematch} ($\mathrm{rematch}$) ist eine direkte Antwort auf Muster dieser Art. Die Fragestellung welche Mengen von Ersetzungsregeln eindeutige Normalformen unabhängig von der Reihenfolge ihrer Anwendung produzieren (\emph{konfluent} sind) ist im allgemeinen Fall nicht entscheidbar \cite{KonfluenzUnentscheidbar}.



\subsection{Strategien}
Eine besondere Eigenschaft der Normalisierung aus Kapitel \ref{secErsteNormalform} ist, dass die dadurch definierte Normalform eines Terms eindeutig ist. Gemeint ist damit, dass nicht nur die Funktion $\mathrm{normalize}$ deterministisch ist, sondern dass sie immer das selbe Ergebnis produziert, wie eine beliebige andere Strategie, die so lange die erläuterten drei Regeln Sortieren kommutativer Funktionsanwendungen, Zusammenfassen assoziativer Funktionsanwendungen und Auswerten auswertbarer Teile auf jeden Teilterm anwendet, bis keine Anwendung mehr zu einer Veränderung führt. Dies ist allerdings nicht für beliebige Mengen von Ersetzungsregeln der Fall. Das einfachste Gegenbeispiel sind zwei Regeln mit gleicher linker Seite aber unterschiedlicher rechter Seite.




Es gibt verschiedene Strategien in einem Term nach Teiltermen zu suchen, die transformiert werden können. Die Funktion $\mathrm{normalize}$ aus Kapitel \ref{secErsteNormalform} geht von innen nach außen, normalisiert also zuerst die Argumente einer Funktionsanwendung, bevor die Funktionsanwendung selbst normalisiert wird. Das hat den Vorteil, dass $\mathrm{normalize}$ ein sehr einfachen Aufbau besitzen kann, schließlich bleiben die Argumente in ihrer Normalform, wenn die umschließende Funktionsanwendung selbst normalisiert wird. Jede Funktionsanwendung eines zu normalisierenden Terms muss also nur ein Mal besucht werden.


Hier wird zuerst probiert Ersetzungsregeln auf das Literal als ganzes anzuwenden, dann, sollte das Literal eine Funktionsanwendung sein, auf seine Argumente. Der Vorteil besteht darin, dass mehr Terme normalisiert werden können \cite{EvalStrategien}. Der Nachteil besteht darin, dass der zur Umsetzung notwendige Algorithmus aufwendiger ist, da die Normalisierung eines Argumentes möglicherweise neue Transformationen für umschließende Terme eröffnet.

Ein grundlegendes Problem ist, dass die Normalform eines Literals für entsprechend gewählte Regelsätze von der gewählten Normalisierungsstrategie abhängt. 
























\section{Sekundäre Konzepte mit Relevanz für die Umsetzung} \label{secHilfUmsetzungInCpp}

\subsection{Speicher} \label{subsecCppSpeicher}

\subsection{SumEnum} \label{subsecCppSumEnum}

\subsection{Lambdafunktionen} \label{subsecLambdafunktionen}
Muster sind dafür gebaut bestimmte Formen von Termen zu erkennen. Der Term, der aus einer Musteranwendung resultieren soll, ist allerdings nicht immer nur ein fester Ausdruck, dessen einzige Freiheitsgerade durch einfaches Ersetzen der im Match der linken Seite gebundenen Mustervariablen durch die entsprechenden Literale beschrieben werden kann. Ist eine umfassendere Transformation notwendig, wäre es mit der etablierten Musterersetzung möglich Helferfunktionen zu definieren. Von Vorteil ist die Nutzung der Musterersetzung allerdings nur dann, wenn auch in der Helferfunktion eine umfassende Fallunterscheidung gemacht werden muss. Die Vergrößerung der Menge der zu matchenden Muster hat allerdings nicht nur eine an die Geschwindigkeit der Mustererkennung gekoppelte Auswertungsgeschwindigkeit,  sondern gibt dem Funktionssymbol der Helferfunktion auch für die gesamte Regelmenge eine Bedeutung. 
Generische Namen wie \texttt{helper} für diese Art von Funktionssymbol sind fehleranfällig, da eine als Ersetzungsregel definierte Helferfunktion für die gesamte Regelmenge sichtbar ist. Die deutliche Abgrenzung der Namen von Helferfunktionen untereinander führt zu sehr verboser Namensgebung.
Das Konstrukt der separat definierten Helferfunktion auch für sehr einfache Abbildungsvorschriften wird deswegen als unelegant und fehleranfällig bewertet. 
Im Kontrast ist die rechte Seite einer Ersetzungsregel dann einfach zu lesen, wenn die genutzten Funktionssymbole eine bereits bekannte Bedeutung haben. Für Funktionssymbole wie \texttt{sum}, \texttt{prod} oder \texttt{pow} ist diese Bedeutung eingebaut, die Auswertungsregeln sind über die $\mathrm{eval}$ Funktion direkt implementiert. Die diskutierten Helferfunktionen ebendfalls auf diese Art einzubauen ist allerdings nicht praktikabel. Zum einen ist die Anforderung der begrenzten Sichtbarkeit dann noch weniger erfüllt: Ein Helfer wäre nicht nur für eine, sondern sogar alle Regelmengen sichtbar. Zum anderen ist die Implementierung eines Helfers über $\mathrm{eval}$ vergleichsweise sehr aufwendig und bietet neben logischen Fehlern im Helfer auch die Möglichkeit Fehler beim Umgang mit der unterliegenden Datenstruktur zu machen. Ziel ist also ein Konstrukt, was es erlaubt einfache Funktionen als Teil eines Musters zu definieren und über $\mathrm{eval}$ statt der Musterersetzung auswerten zu lassen. 
Ein Konzept welches diese Anforderungen erfüllt ist die Lambdafunktion, frei nach \cite{ChurchLambda36}. Im Kontext eines Terms ist eine Lambdafunktion primär ein Funktionsymbol, welches nur durch seine Stelligkeit und Abbildungsvorschrift identifiziert ist. Die Abbildungsvorschrift selbst ist ein Term, welcher neben sonst erlaubten Konstantensymbolen noch vom Lambda gebundene Variablen enthalten kann. Gleichzeitig ist ein Lambda allerdings auch ein gültiges Konstantensymbol, kann also selbst Parameter einer Funktionsanwendung sein.


\begin{definition}
Eine besondere Klasse von Funktionssymbolen und Konstantensymbolen ist die der Lambdas. Die Notation wird eingeleutet durch ein kleines Lambda, gefolgt von der Nennung der gebundenen Variablen. Der Term, der die Abbildungsvorschrift beschreibt folgt als letztes und ist durch ein Punkt von den Variablenbindungen getrennt. Auswertung der Funktionsanwendung eines Lambdas ist $\beta$-Reduktion. \BFred{Ich vermute sachen wie $\beta$-Reduktion oder DeBrujin sollten mit Quellen verknüpft sein?}
$$
\lambda \textit{Variablenname(n)}~.\textit{Abbildungsvorschrift}
$$
Als Beispiel bindet das Funktionssymbol $f = \lambda x y.(\texttt{pow}, x, y)$ die Variablen $x$ und $y$. Die Abbildungsvorschrift von $f$ ist $(x, y) \mapsto (\texttt{pow}, x, y)$, alternativ könnte man also schreiben $f = \texttt{pow}$. Wichtig ist, dass die hier definierten Lambdas in Abweichung vom Lambdakalkül auch mehrere Parameter direkt abbilden können. Während die Schreibweise eines einzelnen kleinen Lambdas gefolgt von mehreren Variablennamen in der Literatur also nur eine Kurzschreibweise für ein Lambda, welches den ersten Variablennamen bindet und auf ein Lambda, welches den zweiten Variablennamen bindet, abbildet, ist, werden hier tatsächlich mehrere Variablennamen von nur einem Lambda gebunden. Definiert man also $g = \lambda x.\lambda y.(\texttt{pow}, x, y)$ ist $f \neq g$, denn eine korrekte Funktionsanwendung von $f$ ist $(f, 1, 2)$, während $((g, 1), 2)$ eine korrekte Funktionsanwendung von $g$ ist\footnote{Sind Lambdas als Konstantensymbole erlaubt, muss dementsprechend die Definition eines Funktionssymbols entsprechend erweitert werden um die Funktionsanwendung einer Funktionsanwendung zu erlauben.}. \\
Intern werden für die Unterscheidung der in Lamdas gebundenen Variablen keine Zeichenketten, sondern DeBrujin Indizes verwendet. Mit DeBrujin Index als Index ergänzt ist $f = \lambda x_0 y_1.(\texttt{pow}, x_0, y_1)$ und $g = \lambda x_0.\lambda y_1.(\texttt{pow}, x_0, y_1)$ sowie $(g, 1) = \lambda y_0.(\texttt{pow}, 1, y_0)$.

Das Beispiel $g$ zeigt die definierende Eigenschaft der DeBrujin Indizes. Durch die Indexverschiebung der gebundenenen Variablen in einer geschachtelten Lambdadefinition um die Anzahl der bereits vorher gebundenen Variablen wird jede Variable eindeutig einer Lambdafunktion zugeordnet. Wird eine Funktionsanwendung eines Lambdas $f$ der Stelligkeit $n$ ausgewertet und befindet sich eine Variable mit DeBrujin Index $i > n$ in der Abbildungvorschrift von $f$, so wird die Variable nicht durch einen Parameter von der Funktionasanwendung von $f$ ersetzt, sondern nur $n$ von $i$ subtrahiert.
\end{definition}

Mit Lambdas als Konstantensymbolen ist es erlaubt, als Parameter der Funktionsanwendung eines Lambdas $a$ ein Lambda $p$ zu übergeben. Es gibt zwei Möglichkeiten die DeBrujin Indizes der von $p$ gebundenen Variablen nach Ersetzung anzupassen. 
Eine Möglichkeit ist, bei der Ersetzung der gebundenen Variablen während der Auswertung eines Lambdas immer die aktuelle Verschiebung mitzuschreiben und bei Ersetzung einer gebundenen Variable durch ein Lambda diese Verschiebung zu allen gebundenen Variablen im resultierenden Lambda dazuzuaddieren. Problem dieses Ansatzes ist zum einen, dass dann nicht nur getestet werden muss ob ein Parameter ein Lambda ist, sondern auch ob ein Parameter ein Lambda enthält. Diese Operation ist in ihrer Komplexität linear in der Anzahl der Funktionsanwendungen und Konstantensymbole des Parameters und deswegen unerwünscht. \\
Die implementierte Lösung des Problems unterscheidet stattdessen zwischen zwei Arten von Lambdas: Ein \emph{transparentes} Lambda ist ausschließlich in der Abbildungvorschrift eines weiteren Lambdas erlaubt. Es verhält sich wie erklärt, kann also auch Variablen enthalten, die vom umgebenden Lambda gebunden sind und hat dementsprechend auch die Indizes der selbst gebundenen Variablen um die Anzahl der weiter außen gebundenen Variablen verschoben. Ein Lambda, welches nicht Teil der Abbildungsvorschrit eines anderen Lamdas ist, ist nicht transparent. Hat es eine Stelligkeit von $n$, bindet also $n$ Variablen, haben diese die DeBrujin Indizes in $\{0, 1, \dots, n-1\}$. Die Funktionsanwendung eines Lambdas wird ausschließlich dann ausgewertet, wenn sie selbst nicht Teil einer Lambdadefinition ist, damit also auch nicht transparent ist. 
Enthält das Ergebnis einer solchen Funktionsauswertung transparente Lambdas, werden diese untransparent, wenn sie ihrerseits nach der Auswertung nicht mehr Teil einer Lambdadefinition sind. 
Ist ein Lambda $p$ Parameter der Funktionsanwendung eines Lambdas $a$, wird $p$ mit der Auswertung der Anwendung von $a$ an die entsprechenden Stellen der Abbildungsvorschrift von $a$ plaziert. Auch wenn diese entsprechenden Stellen innerhalb geschachtelter Lambdas liegen, wird keine Indexverschiebung der von $p$ gebundenen Variablen vorgenommen. $p$ hat sich vor der Auswertung von $a$ allerdings nicht innerhalb eines Lambdas befunden, sonst wäre auch $a$ innerhalb dieses selben Lambdas gewesen, hätte also nicht ausgewertet werden dürfen. Weder $p$, noch $a$ können damit transparent sein. Das nicht transparente Lambda $p$ darf innerhalb der Abbildungsvorschrift eines Lambdas $f$ liegen. Wird eine Funktionsanwendung von $f$ ausgewertet, Wird in $p$ aber nicht nach von $f$ gebundenen Variablen gesucht. In diesem Kontext ist auch die Benennung der Eigenschaft der Tranzparenz zu verstehen: Die Auswertung der Funktionsanwendung eines Lambdas probiert nur in transparenten Teilen der Abbildungvorschrift die gebundenen Variablen zu ersetzten.\\
Diese Verwaltungsstrategie hat den Vorteil, dass die Auswertung der Funktionsanwendung eines Lambdas die Parameter nicht verändern muss, in der Komplexität also auch nicht von der Größe der Parameter abhängt. Der Nachteil ist, dass Lambdas selber nicht normalisiert werden. Das ist allerdings für den geplanten Verwendungszweck ohnehin nicht erforderlich, da Lambdas nicht Teil des Ergebnisses einer Termtransformation sein sollen, sondern lediglich die Transformation selbst vereinfachen. \BFred{TODO: fasse Relation von a, f und p in Bilder}










\chapter{Umsetzung in C\texttt{++}} \label{secKernUmsetzungInCpp}

%.........................................................................
%.........................................................................
%.........................................................................
\section{Konzeptionelle Unterschiede}
Die in diesem Kapitel vergestellte Umsetzung implementiert nicht exakt die bisher beschriebenen Strukturen. Der erste Unterschied ist, dass die Konzepte \emph{Funktionssymbol} und \emph{Konstantensymbol} hier nicht unterschieden werden. Da der Zweck der Implementierung zudem alleine in der Vereinfachung von Ausdrücken über den Komplexen Zahlen $\mathbb C$ liegt, ist die Menge an Symbolen zudem im Code nicht generisch gehalten. Die beiden wichtigsten Arten von Symbolen für Literale sind Komplexe Zahlen $z \in \mathbb C$, sowie Zeichenketten $c_1 c_2\dots c_n$ beliebiger Länge $n$. Die einzelnen Zeichen $c_i$ stammen dabei aus dem Alphabet $\Sigma$, welches aus Klein-und Großbuchstaben des lateinischen Alphabetes, Ziffern von $0$ bis $9$, den Abostroph \verb|'| und dem Unterstrich \verb|_| besteht. Die Ausname bildet das erste Zeichen $c_1$, welches ein Klein-oder Großbuchstabe des Lateinischen Alphabetes sein muss. Zu den Zeichenketten gehören insbesondere auch die Funktionssymbole. Mit der dadurch entstehenden Möglichkeit Funktionssymbole als Werte zu behandeln, ergibt es Sinn auch Funktionsanwendungen von dynamisch bestimmten Funktionssymbolen zuzulassen. Das erste Element $f$ des Funktionsanwendungstupels $(f, \elems t 0 {n-1})$ ist in der Umsetzung also kein Symbol, sondern ein Term. Mit den bis hier diskutierten Änderungen und $\Sigma^+$ als Bezeichnung für die Menge von Zeichenketten über dem beschriebenen Alphabet $\Sigma$ sähe die idealisierte Menge aller Literale in der Umsetzung im Kontrast zu Definition \ref{defTerm} so aus:

$$T \coloneqq \Sigma^+ \cup \mathbb C \cup \curl*{(\elems t 0 n)~|~ \elems t 0 n \in T)}.$$

Die wichtigste Erweiterung der Implementierung ist allerdings die der anoymen Funktion bekannt als \emph{Lambda}. Das Lambdakalkül von Church \cite{ChurchLambda36} definiert Terme verwandt mit den in dieser Arbeit diskutierten. Der wichtigste Unterschied ist, dass keine externe Interpretation für einen gegebenden Lambdaausdruck notwendig ist. Anstatt Funktionssymbole über Namen zu identifizieren und die Abbildungsvorschrift getrennt anzugeben, ist eine Lambdafunktion $f \in \Lambda$ ausschließlich durch ihre Abbildungsvorschrift identifiziert. Church erlaubt neben der Funktionsanwendung als Term lediglich Variablensymbole  $v \in V$ und Lambdafunktionen. Soll $T$ also eine Obermenge aller Ausdrücke im Lambdakalkül werden, müsste prinzipiell nur die Lambdafunktion selbst hinzugefügt werden. Die Menge der Variablen $V$, hier als \emph{Lambdaparameter} bezeichnet wird allerdings von den bisher erlaubten Zeichenketten in $\Sigma^+$ getrennt.

Muster sind in der Umsetzung auch entsprechend flexibler, schließlich sind sie eine Obermenge der Literale. Dazu kommt die Fähigkeit, Bedingungen an Mustervariablen zu stellen, um mögliche Matches weiter einzuschränken. Eine spezielle Form dieser Einschränkung ist dabei die der \emph{Wert-Mustervariable} $w \in W$, welche versucht Matches zu finden, die in Unterkapitel \ref{subsecNormalKombinieren} erlaubt werden, also Komplexe Zahlen wieder in Rechenausdrücke zu dekonstruieren. Näher behandelt wird die Wert-Mustervariable noch im folgenden Abschnitt \ref{subsecMustervariablen}. 
Zuletzt kann die Multi-Mustervariable aus Abschnitt \ref{subsecMulti} von hier an nicht mehr nur als abstrakte Idee gehandelt werden. Die Menge der Multi-Mustervariablen wird $X^*$ genannt. 

\begin{definition}
Die Menge der Terme $T$ ist in diesem Kapitel definiert als
$$T \coloneqq \Sigma^+ \cup \mathbb C \cup X \cup X' \cup X^* \cup W \cup V \cup \Lambda \cup \curl*{(\elems t 0 n)~|~ \elems t 0 n \in T, n > 0)}$$
mit
\begin{align*}
    \Sigma^+  &\coloneqq \text{Zeichenketten}\\
    \mathbb C &\coloneqq \text{Komplexe Zahlen}\\
    X         &\coloneqq \text{Mustervariablen}\\
    X^*       &\coloneqq \text{Multi-Mustervariablen}\\
    W         &\coloneqq \text{Wert-Mustervariablen}\\
    V         &\coloneqq \text{Lambdaparameter}\\
    \Lambda   &\coloneqq \text{Lambdafunktionen}.
\end{align*}
\end{definition}




%.........................................................................
%.........................................................................
%.........................................................................
\section{Lambdafunktionen} \label{subsecLambdafunktionen}
Die umgesetzten Funktionen erweitern die Definition von Church, indem die selbe Lambdaabstraktion auch mehrere Parameter erlaubt. Während der Ausdruck $\lambda x y . x(y)$ für Church also nur eine vereinfachte Schreibweise\footnote{Die Klammern um $y$ werden in der Literatur oft weggelassen, für diese Arbeit sind sie allerdings notwendig.} für den Ausdruck $\lambda x .\lambda y .x(y)$ darstellt, handelt es sich für die hier berschriebende Umsetzung um zwei verschiedene Funktionen. 
Ein Lambdaparameter ist in der Syntax nicht von anderen Symbolen unterscheidbar\footnote{siehe \ref{subsubsecLambdaSyntax}}, wird intern allerdings durch einen Index dargestellt. Die Darstellung unterscheidet sich von De Bruijn Indizierung \cite{deBruijn} darin, dass jedes Vorkommen einer Variable immer den selben Index hat.
 Jeder neu gebundene Parameter bekommt als Index die Anzahl der weiter außen bzw. in der selben Abstraktion vor ihm gebundenen Parameter. Zur Veranschaulichung wird hier der Index im Tiefsatz mitgeschrieben. Wichtig ist hervorzuheben, dass die nach wie vor dargestellten Namen ausschließlich der Übersichtlichkeit dienen und in der Umsetzung nicht gespeichert sind.
 Als Beispiel dient $\lambda x_0 y_1 .\lambda z_2 .x_0 + y_1 + z_2$. Das Zeichen $x$ wird in der äußersten Abstraktion zuerst gebunden, hat also keine Vorgänger und dementsprechend Index $0$. In der selben Abstraktion wird als zweiter Parameter weiter $y$ gebunden. Als Nachfolger von $x$ wird $y$ Index $1$ zugewiesen. Die Bindung von $z$ liegt Innerhalb der Abbildungsvorschrift des äußeren Lambdas, $x$ und $y$ können also referenziert werden. Dementsprechend hat $z$ zwei Vorgänger und Index $2$. 
 
 Die Auswertung der Funktionsanwendung einer Lambdafunktion $f$ entspricht der $\beta$-Reduktion (von Church \cite{ChurchLambda36} als \emph{operation II} bezeichnet). Parameter von $f$ werden dafür durch die übergebenen Argumente ersetzt. Enthält die Definition von $f$ selbst weitere Lambdaabstraktionen, so werden die Indices derer Parameter um die Stelligkeit von $f$ erniedrigt.
 Der Ausdruck $(\lambda x_0 y_1 .\lambda z_2 .x_0 + y_1 + z_2)(3, 6)$ wird damit zum Ausdruck $\lambda z_0 .3 + 6 + z_0$ ausgewertet.
 
 Die gewählte Indizierung hat ein Problem, welches die De Bruijn Indizierung nicht besitzt. Der Index eines Lambdaparameters ist abhängig vom Kontext der bindenden Abstraktion. Ändert sich dieser Kontext, stimmt die bisherige Indizierung möglicherweise nicht mehr:
 Enthält ein Argument $a$ der Funktionsanwendung einer Lambdafunktion $f$ selbst eine Lambdafunktion $g$, so reicht es für die Auswertung von $f$ mit bisher diskutierten Konzepten nicht aus, den entsprechenden Lambdaparameter von $f$ einfach durch $a$ zu ersetzen. Sollte diese Ersetzung innerhalb einer Lambdaabstraktion $f'$ stattfinden, müssen alle Lambdaparameter von $g$ im Index um die Anzahl der weiter außen gebundenen Parameter erhöht werden. Als Beispiel ist $a = g = \lambda x_0 .x_0$ die Identität, welche der Funktion $f = \lambda x_0 .\lambda y_1 .x_0(y_1)$ übergeben wird. Teil der Definition von $f$ ist $f' = \lambda y_1 .x_0(y_1)$.
 $$(\lambda x_0 .\lambda y_1 .x_0(y_1))(\lambda x_0 .x_0)$$
 Alternativ kann der selbe Audruck dargestellt werden, ohne Information zu benutzen
 An dieser Stelle wird zudem eine alternative Notation eingeführt, die nur Information benutzt, die der Implementierung tatsächlich bekannt ist. Ehrlicher ist die Darstellung einer Lambdaabstraktionen mit $n$ Parametern als $\lambda [n] .\textit{<Definition>}$. Mit den Parametern geschrieben als Prozentzeichen gefolgt von ihrem Index, ist der problematische Ausdruck geschrieben als
 
 $$(\lambda [1] .\lambda [1] .\%0(\%1))(\lambda [1] .\%0)$$.
 Die Naive Ersetzung von $x_0$ in $f$ durch $g$ würde im folgenden Ausdruck resultieren.
 $$\lambda y_0 .(\lambda x_0 .x_0)(y_0)$$
 Ohne Namen ist jetzt klar, dass in der Abbildungsvorschrift von $g$ nun fälschlicherweise auf den Parameter von $f'$ abgebildet wird:
 $$\lambda [1] .(\lambda [1] .\%0)(\%0)$$
 Es gibt verschiedene Möglichkeiten den Fehler zu beheben. Die wohl einfachste Lösung ist Indices von eingesetzten Argumenten in der Auswertung einer Lambdafunktion entsprechend anzupassen. Das würde allerdings bedeuten, dass die Komplexität der Auswertung von der Größe der Argumente abhängig ist. Ein weiterer Nachteil ist, dass veränderte Argumente kopiert werden müssen. Das macht \emph{lazy evaluation} \cite{EvalStrategien}, also die Idee jede Funktionsanwendung höchstens ein Mal auswerten zu müssen, unmöglich.
 
 Eine Alternative ist die Unterscheidung zwischen sogenannten \emph{transparenten} Lambdas und \emph{nicht-transparenten} Lambdas, zweitere dargestellt durch umschließende geschweifte Klammern: $\{\lambda \textit{<Parameter>}.\textit{<Definition>}\}$, bzw. $\{\lambda [n].\textit{<Definition>}\}$. Kommt ein transparentes Lambda $f'$ in der Definition eines Lambdas $f$ vor, wird bei der Auswertung von $f$ auch in $f'$ nach Lambdaparametern gesucht und diese ersetzt, bzw. deren Index erniedrigt. Transparente Lambdas verhalten sich damit nicht anders als die Lambdas bisher. Ist $f'$ allerdings nicht-transparent, lässt die Auswertung von $f$ die Definition von $f'$ unberührt. 
 Mit zwei weiteren Bedingungen wird die Überprüfung von Argumenten bei der Auswertung Lambdas dann obszolet: 
 Zum einen darf ein \emph{außenliegendes} Lambda $f$ nicht transparent sein. Ein Lambda ist außenliegend, wenn es nicht Teil der Definition einem weiteren Lambdas ist. Zum anderen darf die Funktionsanwendung eines Lambdas nur ausgewertet werden, wenn das Lambda außenliegend ist. Damit wird garantiert, dass während der Auswertung des Lambdas $f$ kein Argument $a$ direkt transparente Lambdas enthält, höchtens als Teil $g'$ (im Beispiel nicht enthalten) eines nicht-transparenten Lambdas $g$.

Im bereits behandelten Problemfall sind $f$ und $g$ damit nicht-transparent. Nur $f'$ ist zu Beginn transparent, wird allerdings nach Auswertung von $f$ ebendfalls zur äußersten Lambdaabstraktion, verliert also seine Transparenz. 
\begin{align*}
    ~           &~\{\lambda x_0 .\lambda y_1 .x_0(y_1)\}(\{\lambda x_0 .x_0\}) \\
    \rightarrow &~\{\lambda y_0 .\{\lambda x_0 .x_0\}(y_0)\}\\
\end{align*}
Alternativ ohne Namen:
\begin{align*}
    ~           &~\{\lambda [1] .\lambda [1] .\%0(\%1)\}(\{\lambda [1] .\%0\}) \\
    \rightarrow &~\{\lambda [1] .(\{\lambda [1] .\%0\})(\%0)\}\\
\end{align*}
Bemerkenswert ist, dass mit den soeben formulierten Bedingungen der Ausdruck nicht weiter transformiert werden darf. Das liegt daran, dass die Funktionsanwendung von $g$ selbst noch vom Lambda $f'$ umschlossen ist. Es ist also nicht ausgeschlossen, dass Argumente von $g$ (hier nur $\%0$) direkt transparente Lambdas enthalten.


%.........................................................................
%.........................................................................
%.........................................................................
\section{Mustervariablen} \label{subsecMustervariablen}
In der Umsetzung wird zwischen drei Arten von Mustervariablen unterschieden. Die Sonderform der Multi-Mustervariable ist bereits aus Kapitel \ref{subsecMulti} bekannt, die Wert-Mustervariable ist allerdings neu. Als drittes ist die normale Mustervariable nach wie vor vorhanden. Wert-Mustervariable und normale Mustervariable können zudem durch Bedingungen eingeschränkt werden.

\subsection{Wert-Mustervariable} \label{subsubsecWertMustervariable}
Nach Definition \ref{defMatch} ist Match $v_p$ für ein Paar $(p, t)$ aus Muster $p$ und Literal $t$ dann gültig, wenn die Normalform von $p$ mit den Mustervariablen ersetzt duch die Funktionswerte von $v_p$ identisch zu $t$ ist. In Kapitel \ref{secPattermatching} nicht diskutiert wurde die Möglichkeit Zahlen mit Rechenausdrücken zu matchen. Ein Muster $p = (\texttt{prod}, 2, \mathbf k)$ würde für das Literal $t = 12$ etwa ein Match $v_p$ mit $v_p~\mathbf k = 6$ besitzen. Um die Matchsuche nicht zu kompliziert zu gestalten, besitzt die Umsetzung zwei Einschränkungen. Zum einen ist eine solche Dekonstruktion des Musters zu einem Wert nur dann möglich, wenn das entsprechende Teilmuster exakt eine Instanz $w$ einer Wert-Mustervariable enthält. Zum anderen werden nur bijektive Funktionen\footnote{bzw. sonst Mehrdeutigkeiten ignoriert} dekonstruiert, wie im Beispiel die Multiplikation mit $2$. 


Vor einem Matchversuch muss ein Muster mit Wert-Mustervariablen zuerst entsprechend transformiert werden. Ziel der Transformation ist, das für den Matchalgorithmus direkt klar ist, ab wo das Muster einen Wert repräsentieren soll. Der Teilbaum des Musters, der eine Wert-Mustervariable $w$ enthält und als ganzes einen Wert matchen soll, ist idealerweise also direkt gekennzeichnet. Um den Wert $v_p~w$ zu identifizieren, ist \BFred{TODO}

\subsection{Bedingungen} \label{subsubsecBedingungen}
\BFred{TODO}

%.........................................................................
%.........................................................................
%.........................................................................
\section{Syntax} \label{subsecSyntax}
In der Umsetzung wird der Termbaum immer aus einer ASCII Zeichenkette gebaut.
Solche Zeichenketten sind von hier an in \texttt{monospace} gesetzt. Auf eine formale Definition der genutzten Grammatik wird verzichtet.

\subsection{Funktionsanwendungen}
\begin{figure}
    \label{tabZucker}
    \centering
    \begin{tabular}{l l}
        \hline
        Normale Schreibweise & Alternative Syntax\\
        \hline \hline
        \verb|prod(-1, x)|         & \verb|-x|\\
        \verb|not(x)|              & \verb|!x|\\
        \verb|pow(x, y)|           & \verb|x^y|\\
        \verb|prod(x, y)|          & \verb|x * y|\\
        \verb|prod(x, y)|          & \verb|x y|\\
        \verb|prod(x, pow(y, -1))| & \verb|x / y|\\
        \verb|sum(x, y)|           & \verb|x + y|\\
        \verb|sum(x, prod(-1, y))| & \verb|x - y|\\
        \verb|cons(x, y)|          & \verb|x :: y|\\
        \verb|eq(x, y)|            & \verb|x == y|\\
        \verb|neq(x, y)|           & \verb|x != y|\\
        \verb|greater(x, y)|       & \verb|x > y|\\
        \verb|smaller(x, y)|       & \verb|x < y|\\
        \verb|greater_eq(x, y)|    & \verb|x >= y|\\
        \verb|smaller_eq(x, y)|    & \verb|x <= y|\\
        \verb|and(x, y)|           & \verb|x && y|\\
        \verb|or(x, y)|            & \verb!x || y!\\
        \verb|of_type__(x, y)|     & \verb|x :y|\\
        \hline
    \end{tabular}
    \caption{alternative Syntax für bestimmte Funktionssymbole}
\end{figure}

Die bisher genutzte Schreibweise $(f, x, y, z)$ für die Funktionsanwendung des Funktionssymbols $f$ auf die Parameter $x, y, z$ wurde in Abschnitt \ref{subsecTerm} eingeführt, um den Term syntaktisch von anderen Ideen zu differenzieren. Dies ist für die Umsetzung in C\texttt{++} nicht notwendig, da von vorne herein klar ist, ob eine Zeichenkette in einen Term übersetzt wird. Aus dem Grund werden Funktionsanwendungen in der Syntax \verb|f(x, y, z)| geparst. 
Weiter können Funktionsanwendungen bestimmter Funktionen auch mit fest definierten Infixoperatoren geschrieben werden, siehe Tabelle \ref{tabZucker}. Die Bidekraft der Operatoren nimmt in der Tabelle von oben nach unten ab.
Aus der Tabelle hervorzuheben sind zwei Dinge: Zum einen wird ein Leerzeichen zwischen zwei Termen als Multiplikation interpretiert, Funktionsanwendungen dürfen also kein Leerzeichen zwischen Funktion und Parametertupel setzen. \verb|f (x)| ist dementsprechend gleichbedeutend zu \verb|prod(f, x)|, während \verb|f (x, y)| einen Syntaxfehler darstellt. Die zweite Besonderheit ist das Fehlen von Funktionssymbolen für die Darstellung des additiven Inversen und des multiplikativen Inversen. Dies reduziert die Anzahl der Ersetzungsregeln, die benötigt werden um eine Gesetzmäßigkeit, die Summen oder Produkte involviert, abzubilden. 

\subsection{Lambdafunktionen} \label{subsubsecLambdaSyntax}
Da der ASCII Zeichensatz keine griechischen Buchstaben enthält, wird anstelle des kleinen Lambdas $\lambda$ der umgekehrte Schrägstrich \verb~\ ~genutzt. Die Identitätsfunktion $\lambda x.x$ wird dementsprechend \verb~\x .x~ geschrieben. Mehrere Parameter werden durch Leerzeichen getrennt: \verb~\x y .pow(x, y)~ ist eine Lambdafunktion mit identischem Verhalten zum Funktionssymbol \verb|pow|.

\subsection{Symbole}
Komplexe Zahlen auf der reellen Achse sind in der Syntax vergleichbar mit Darstellungen für Integer und Fließkommazahlen erlaubt in C. Möglich sind etwa \verb|42|, \verb|3.1415|, \verb|1.337e3|, \verb|1.602e-19| oder \verb|1e+10|. Komplexe Zahlen auf der imaginären Achse sind in der Struktur identisch, allerdings immer direkt gefolgt vom Zeichen \verb|i|. 

Die verschiedenen Mustervariablen werden auch unterschiedliche Weise identifiziert. Eine normale Mustervariable muss mit einem Unterstrich beginnen (\verb|_x|), eine Wert-Mustervariable mit einem Dollar (\verb|$x|) und eine Multi-Mustervariable endet in drei Punkten (\verb|xs...|). Besitzt ein Name keiner dieser besonderen Merkmale, wird daraus kontextabhängig ein Lambdaparameter oder ein Symbol aus $\Sigma^+$. Ist der umgebende Name bereits in einem umschließenden Lambda gebunden, wird die innerste solche Bindung gewählt. Als Beispiel ist im Ausdruck \verb|\x .x + y| das Symbol \verb|x| als Lambdaparameter interpretiert, während \verb|y| Als Zeichenkette erhalten bleibt.
Im Ausdruck \verb|\x .\x .2 - x| kann der Parameter \verb|x| des äußeren Lambdas in der Definition des inneren Lambdas nicht mehr referenziert werden, der Gesamtausdruck ist damit identisch zu \verb|(\x .(\y .2 - y))|.


\subsection{Ersetzungsregel}
Es gibt zwei Varianten Ersetzungsregeln zu schreiben.
\begin{align*}
	~&~&~&~&\textit{<linke Seite>}&~ ~                       &= \textit{<rechte Seite>}&~&~&~&~\\
	~&~&~&~&\textit{<linke Seite>}&~|~\textit{<Bedingungen>} &= \textit{<rechte Seite>}&~&~&~&~
\end{align*}
In der zweiten Variante ist \textit{<Bedingungen>} eine kommaseparierte Liste von Bedingungen and die vorkommenden Mustervariablen, wie in Abschnitt \ref{subsubsecBedingungen} diskutiert. 



%.........................................................................
%.........................................................................
%.........................................................................
\section{Datenstruktur}
\BFred{TODO}

%.........................................................................
%.........................................................................
%.........................................................................
\section{Algorithmen} \label{subsecCppAlgos}
\BFred{TODO}



















\section{Vereinfachen von arithmetischen Termen}
\begin{itshape}
\textcolor{red} {Anmerkung zur Anmerkung: ab hier sind es nur noch Anmerkungen, da wird Farbe gespart (in echt mag \LaTeX{}  es nicht über Kapitelgrenzen hinweg zu färben).}
Anmerkung: Ich habe begonnen das Termersetzungssystem zu entwickeln, um Arithmetische Ausdrücke zu vereinfachen (etwa $a + 2 a \rightarrow 3 a$). Wie genau ich das umsetze, wird in diesem Abschnitt erläutert.
\\Während die Datenstruktur und der Matchingalgorithmus schon benutzbar sind, ist dieser Teil von mir bisher so gut wie gar nicht implementiert worden. Der grobe Plan ist aber folgender:
\begin{enumerate}
    \item Funktionen höherer Ordnung anwenden:
    \begin{itemize}
        \item ableiten (Prototyp dafür steht schon)
        \item vielleicht integrieren (soll für den allgemeinen Fall wohl schwer sein)
        \item vielleicht fouriertransformieren
        \item vielleicht laplacetransformieren
        \item ganz ganz ganz ganz vielleicht Differentialgleichungen lösen
    \end{itemize}
    \item Normalform herstellen:
    \begin{itemize}
        \item alles ausmultiplizieren ($a\cdot (b + c) \rightarrow a\cdot b + a\cdot c$)
        \item Vorzeichen aus ungeraden Funktionen herausziehen ($\sin(-x) \rightarrow -\sin(x)$)
        \item Vorzeichen in geraden Funktionen auf plus setzen ($\cos(-x) \rightarrow \cos(x)$)
        \item Überlegen, wie man das selbe für Fälle mit Summen im Argument definiert ($\cos(a - b)$ vs. $\cos(b - a)$)
        \item bekannte Faktoren aus Potenz ziehen ($(3 x)^2 \rightarrow 9 x^2$)
        \item \dots
    \end{itemize}
    \item Vereinfachen:
    \begin{itemize}
        \item manche Transformationen sollten immer angewendet werden (etwa $\sin^2(x) + \cos^2(x) \rightarrow 1)$
        \item andere Transformationen nur ausprobieren und mit einer passenden Metrik gucken, wie gut ein Term nach Anwendung noch weiter vereinfacht werden kann (etwa, wenn man aus verschiedenen Optionen des Ausklammerns wählen kann)
        \item vielleicht Linearfaktorzerlegung von Polynomen (schätze ich für den allgemeinen Fall schwierig ein, solange ich nur exakte Operationen zulasse)
        \item vielleicht Polynomdivision (schätze ich genau so schwierig ein, zumindest wenn ich nicht vorher schon Linearfaktoren habe)
        \item \dots
    \end{itemize}
\end{enumerate}
~\\~
Anmerkung 1: Die Normalform ist notwendig, um zu garantieren, dass mehrfaches Auftreten eines Teilbaums / Teilterms auch erkannt wird. \\
Anmerkung 2: es kann sein, dass ich manche Eigenschaften der Normalform auch während des Vereinfachungsschrittes immer wieder wiederherstellen muss.

\subsection{Vergleich meiner Features mit anderen Computeralgebrasystemen}
Ich bin ja nicht der erste, der auf die Idee kommt, Terme zusammenzufassen. Wolphram Alpha und Maple sind zwar nicht Open Source, aber andere Optionen, wie etwa SymPy aus der Python Standardbibliothek soweit ich weiß schon. Da lässt sich bestimmt ein bisschen vergleichen, wie andere Leute die selben Probleme lösen.
Spannend ist mit Sicherheit auch der Vergleich zu Egison.
\end{itshape}





















\chapter{Zusammenfassung}
\begin{itshape}
Was halt in eine Zusammenfassung kommt
\end{itshape}








\printbibliography

\end{document}


































